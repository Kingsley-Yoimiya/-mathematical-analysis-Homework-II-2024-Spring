%-*-    coding: UTF-8   -*-
% !TEX program = xelatex
\ifx\allfiles\undefined
%\special{dvipdfmx:config z 0}% 取消压缩,加快编译速度
\documentclass[UTF-8]{ctexart}
\usepackage{graphicx}
\usepackage{subfigure}
\usepackage{xcolor}
\usepackage{amsmath}
\usepackage{amssymb}
\usepackage{tabularx}
\usepackage{amssymb}
\usepackage{amsthm}
%\usepackage[usenames,dvipsnames]{color}
\usepackage{hyperref}
\hypersetup{
	colorlinks=true,
	linkcolor=black,
	filecolor=black,      
	urlcolor=red,
	citecolor=black,
}
\usepackage{geometry}
\geometry{a4paper,centering,scale=0.80}
\usepackage[format=hang,font=small,textfont=it]{caption}
\usepackage[nottoc]{tocbibind}
\usepackage{algorithm}  
\usepackage{algorithmicx}  
\usepackage{algpseudocode}
\usepackage{prettyref}
\usepackage{framed}
\setlength{\parindent}{2em}
\usepackage{indentfirst}
\usepackage[framemethod=TikZ]{mdframed}
\newcounter{ques}[section]
\renewcommand{\theques}{\arabic{section}.\arabic{ques}}
\newcommand{\setParDis}{\setlength {\parskip} {0.3cm} }
\newcommand{\setParDef}{\setlength {\parskip} {0pt} }
\setParDis% 调整这一个subsection的段落间距
%\setParDef%恢复间距

\newenvironment{ques}[1][]{
	\refstepcounter{ques}
	\mdfsetup{
		frametitle={
			\tikz[baseline=(current bounding box.east), outer sep=0pt]
			\node[anchor=east,rectangle,fill=blue!20]
			{\strut Problem~\theques\ifstrempty{#1}{}{:~#1}};},
		innertopmargin=10pt,linecolor=blue!20,
		linewidth=2pt,topline=true,
		frametitleaboveskip=\dimexpr-\ht\strutbox\relax
	}
	\begin{mdframed}[]\relax
}{\end{mdframed}}

\newcounter{Thm}[section]
\renewcommand{\theThm}{\arabic{section}.\arabic{Thm}}
\newenvironment{Thm}[1][]{
	\refstepcounter{Thm}
	\mdfsetup{
		frametitle={
			\tikz[baseline=(current bounding box.east), outer sep=0pt]
			\node[anchor=east,rectangle,fill=blue!20]
			{\strut Theorem~\theThm\ifstrempty{#1}{}{:~#1}};},
		innertopmargin=10pt,linecolor=blue!20,
		linewidth=2pt,topline=true,
		frametitleaboveskip=\dimexpr-\ht\strutbox\relax
	}
	\begin{mdframed}[]\relax
}{\end{mdframed}}

\newcounter{Defi}[section]
\renewcommand{\theDefi}{\arabic{section}.\arabic{Defi}}
\newenvironment{Defi}[1][]{
	\refstepcounter{Defi}
	\mdfsetup{
		frametitle={
			\tikz[baseline=(current bounding box.east), outer sep=0pt]
			\node[anchor=east,rectangle,fill=blue!20]
			{\strut Definition~\theDefi\ifstrempty{#1}{}{:~#1}};},
		innertopmargin=10pt,linecolor=blue!20,
		linewidth=2pt,topline=true,
		frametitleaboveskip=\dimexpr-\ht\strutbox\relax
	}
	\begin{mdframed}[]\relax
}{\end{mdframed}}

\newrefformat{qlt}{\underline{性质 \ref{#1}}}
\newcommand{\tpf}[2]{\begin{ques}[#1]{\kaishu #2}\end{ques}}
\newcommand{\pf}[1]{\begin{ques}{\kaishu #1}\end{ques}}
\newcommand{\tthm}[2]{\begin{Thm}[#1]{\kaishu #2}\end{Thm}}
\newcommand{\thm}[1]{\begin{Thm}{\kaishu #1}\end{Thm}}
\newcommand{\tdefi}[2]{\begin{Defi}[#1]{\kaishu #2}\end{Defi}}
\newcommand{\defi}[1]{\begin{Defi}{\kaishu #1}\end{Defi}}
\newcommand{\opf}[1]{{\kaishu{#1}}}
\title{数学分析 I 作业(2024. Spring)}
\author{\texttt{As-The-Wind}}

\date{2024 年 2 月 19 日 $\rightarrow$ \today}

\date{}
\author{尹锦润}
\begin{document}
\maketitle
\fi

\section{2024.3.4 作业}

\begin{ques}
	确定积分的正负性 $\displaystyle \int _{0}^{2\pi } x(\sin x)^{2n+1}\mathrm{d} x$。
\end{ques}
\begin{equation*}
	\begin{aligned}
		\int _{0}^{2\pi } x(\sin x)^{2n+1}\mathrm{d} x & =\int _{0}^{\pi } x(\sin x)^{2n+1}\mathrm{d} x-\int _{2\pi }^{\pi } x(\sin x)^{2n+1}\mathrm{d} x\\
		& =\int _{0}^{\pi }( x-( 2\pi -x))(\sin x)^{2n+1}\mathrm{d} x\\
		& =\int _{0}^{\pi }( 2x-2\pi )(\sin x)^{2n+1}\mathrm{d} x\\
		& < 0
	\end{aligned}
\end{equation*}


因此是 负号。



\begin{ques}
	计算积分:$\displaystyle \int _{0}^{a}\arctan\sqrt{\frac{a-x}{a+x}}\mathrm{d} x( a >0)$。
\end{ques}
\begin{equation*}
	\begin{aligned}
		\int _{0}^{a}\arctan\sqrt{\frac{a-x}{a+x}}\mathrm{d} x & =\left. x\arctan\sqrt{\frac{a-x}{a+x}}\middle| _{0}^{a}\right. -\int _{0}^{a} x\frac{1}{1+\frac{a-x}{a+x}}\sqrt{\frac{a+x}{a-x}}\frac{-a}{( a+x)^{2}}\mathrm{d} x\\
		& =\int _{0}^{a}\frac{x}{2\sqrt{a^{2} -x^{2}}}\mathrm{d} x\\
		& =-\frac{1}{2}\int _{0}^{a}\mathrm{d}\sqrt{a^{2} -x^{2}}\\
		& =\frac{a}{2}
	\end{aligned}
\end{equation*}




\begin{ques}
	计算积分:$\displaystyle \int _{0}^{1} x^{m-1}( 1-x)^{n-1}\mathrm{d} x$。
\end{ques}



记 $\displaystyle F( n,m) =\int _{0}^{1} x^{m-1}( 1-x)^{n-1}\mathrm{d} x$。
\begin{gather*}
	\begin{aligned}
		F( n,1) & =\int _{0}^{1}( 1-x)^{n-1}\mathrm{d} x\\
		& =\left. -\frac{( 1-x)^{n}}{n}\middle| _{0}^{1}\right. \\
		& =\frac{1}{n}
	\end{aligned}\\
	\begin{aligned}
		m >1:F( n,m) & =\int _{0}^{1} x^{m-1}( 1-x)^{n-1}\mathrm{d} x\\
		& =-\frac{1}{n}\int _{0}^{1} x^{m-1}\mathrm{d}( 1-x)^{n}\\
		& =-\frac{1}{n}\left[\left. ( 1-x)^{n} x^{m-1}\middle| _{0}^{1}\right. -\int _{0}^{1}( m-1) x^{m-2}( 1-x)^{n}\mathrm{d} x\right]\\
		& =\frac{m-1}{n}\int _{0}^{1} x^{m-2}( 1-x)^{n}\mathrm{d} x\\
		& =\frac{m-1}{n} F( n+1,m-1)\\
		& =\frac{m-1}{n} \times \frac{m-2}{n+1} \times \cdots \times \frac{1}{n+m-2} =\frac{( m-1) !( n-1) !}{( n+m-2) !} =\binom{n+m-2}{n-1}
	\end{aligned}\\
	\\
	F( n,m) =\begin{cases}
		\frac{1}{n} & ,m=1\\
		\binom{n+m-2}{n-1} & ,m >1
	\end{cases}
\end{gather*}




\begin{ques}
	设函数 $\displaystyle f( x) \in D[ a,b] ,f'( x) \in R[ a,b] ,$证明:

(1)对于任意的 $\displaystyle x\in [ a,b] \ s.t.\ |f( x) |\leqslant \left| \frac{1}{b-a}\int _{a}^{b} f( x)\mathrm{d} x\right| +\int _{a}^{b} |f'( x) |\mathrm{d} x$。

(2)当 $\displaystyle f( a) \neq f( b)$,上式成立严格不等式。

\end{ques}


证明:



(1)

根据定积分第一中值定理,$\displaystyle \exists \xi \in [ a,b] \ s.t.\ \int _{a}^{b} f( x)\mathrm{d} x=( b-a) f( \xi )$。

往证 $\displaystyle \forall x\in [ a,b] \ s.t.\ |f( x) |\leqslant | f( \xi )| +\int _{a}^{b} |f'( x) |\mathrm{d} x$。

而
\begin{equation*}
	\begin{aligned}
		f( x) & =f( \xi ) +\int _{\xi }^{x} f'( t)\mathrm{d} t\\
		|f( x) | & \leqslant |f( \xi ) |+\int _{\min( \xi ,x)}^{\max( \xi ,x)} |f'( t) |\mathrm{d} t\\
		& \leqslant |f( \xi ) |+\int _{a}^{b} |f'( x) |\mathrm{d} x
	\end{aligned}
\end{equation*}
\qed 

(2)

考虑反证,不妨假设 $\displaystyle x\geqslant \varepsilon $ 时不等式可以取等,一定有:
\begin{itemize}
	\item $\displaystyle ( \varepsilon ,x)$ 上 $\displaystyle f( x)$ 单调上升,如果不满足,取 $\displaystyle c,d\in ( \varepsilon ,x) \ s.t.\ c< d\lor f( c)  >f( d)$,有 
\end{itemize}
\begin{gather*}
	\int _{\varepsilon }^{x} f'( t)\mathrm{d} t=\int _{\varepsilon }^{c} f'( t)\mathrm{d} t+\int _{c}^{d} f'( t)\mathrm{d} t+\int _{d}^{x} f'( t)\mathrm{d} t\\
	< \int _{\varepsilon }^{c} f'( t)\mathrm{d} t+\int _{c}^{d} |f'( t) |\mathrm{d} t+\int _{d}^{x} f'( t)\mathrm{d} t\leqslant \int _{\varepsilon }^{x} |f'( t) |\mathrm{d} t
\end{gather*}
	(1)问中第一个 $\displaystyle \leqslant $ 无法取等。
\begin{itemize}
	\item $\displaystyle f( t) \equiv f( a) ,\ \forall t\in [ a,\varepsilon ] ,f( t) \equiv f( b) ,\ \forall t\in [ x,b]$。类似上面一点,不妨假设是 $\displaystyle f( t) \equiv f( a) ,\ \forall t\in ( a,\varepsilon )$ 无法满足,一定会使得 (1)问中第二个 $\displaystyle \leqslant $ 无法取等。
\end{itemize}

记函数最大值为 $\displaystyle M=f( b) =f( x)$,最小值为 $\displaystyle m=f( a) =f( \varepsilon )$。
\begin{gather*}
	\int _{a}^{b} f( t)\mathrm{d} t-( b-a) f( \varepsilon ) =\int _{a}^{b}( f( t) -f( \varepsilon ))\mathrm{d} t\\
	=\int _{\varepsilon }^{x}( f( t) -f( \varepsilon ))\mathrm{d} t+\int _{x}^{b}( f( t) -f( \varepsilon ))\mathrm{d} t=\int _{\varepsilon }^{x}( f( t) -f( \varepsilon ))\mathrm{d} t+( M-m)( b-x)
\end{gather*}

\begin{itemize}
	\item 如果 $\displaystyle x=\varepsilon $,会使得 $\displaystyle M=m$,矛盾。
	\item 如果 $\displaystyle x\neq \varepsilon $取 $\displaystyle v=\frac{f( x) +f( \varepsilon )}{2}$,因为 $\displaystyle f( x) \in D[ a,b] \Rightarrow f( x) \in C[ a,b]$,那么一定存在 $\displaystyle p\in ( \varepsilon ,x) \ s.t.\ f( p) =v$。进而 
\end{itemize}
\begin{equation*}
	\int _{\varepsilon }^{x}( f( t) -f( \varepsilon ))\mathrm{d} t\geqslant ( f( p) -f( \varepsilon ))( x-p)  >0
\end{equation*}

于 $\displaystyle \varepsilon $ 定义矛盾。
\qed 



\begin{ques}
	设 $\displaystyle f( x) \in C^{1}[ a,b]$,且 $\displaystyle f( a) =f( b) =0,\int _{a}^{b} f^{2}( x)\mathrm{d} x=1$,证明


\begin{equation*}
	\int _{a}^{b}[ f'( x)]^{2}\mathrm{d} x\times \int _{a}^{b}[ xf( x)]^{2}\mathrm{d} x\geqslant \frac{1}{4}
\end{equation*}

\end{ques}
\begin{equation*}
	\begin{aligned}
		1=\int _{a}^{b} f^{2}( x)\mathrm{d} x & =\left. xf^{2}( x)\middle| _{a}^{b}\right. -2\int _{a}^{b} xf( x) f'( x)\mathrm{d} x\\
		& =-2\int _{a}^{b} xf( x) f'( x)\mathrm{d} x
	\end{aligned}
\end{equation*}


下面证明 $\displaystyle \left[\int _{a}^{b} f( x) g( x)\mathrm{d} x\right]^{2} \leqslant \int _{a}^{b} f^{2}( x)\mathrm{d} x\times \int _{a}^{b} g^{2}( x)\mathrm{d} x$。



考虑 $\displaystyle 0\leqslant \int _{a}^{b}[ f( x) -tg( x)]^{2}\mathrm{d} x=\int _{a}^{b} f^{2}( x)\mathrm{d} x+t^{2}\int _{a}^{b} g^{2}( x)\mathrm{d} x-2t\int _{a}^{b} f( x) g( x)\mathrm{d} x$,进而
\begin{gather*}
	2t\int _{a}^{b} f( x) g( x)\mathrm{d} x\leqslant \int _{a}^{b} f^{2}( x)\mathrm{d} x+t^{2}\int _{a}^{b} g^{2}( x)\mathrm{d} x\\
	2\int _{a}^{b} f( x) g( x)\mathrm{d} x\leqslant \min_{t}\left(\frac{1}{t}\int _{a}^{b} f^{2}( x)\mathrm{d} x+t\int _{a}^{b} g^{2}( x)\mathrm{d} x\right)\\
	2\int _{a}^{b} f( x) g( x)\mathrm{d} x\leqslant 2\sqrt{\int _{a}^{b} f^{2}( x)\mathrm{d} x\times \int _{a}^{b} g^{2}( x)\mathrm{d} x}\\
	\left[\int _{a}^{b} f( x) g( x)\mathrm{d} x\right]^{2} \leqslant \int _{a}^{b} f^{2}( x)\mathrm{d} x\times \int _{a}^{b} g^{2}( x)\mathrm{d} x
\end{gather*}

进而
\begin{gather*}
	\int _{a}^{b} xf( x) f'( x)\mathrm{d} x=-\frac{1}{2}\\
	\frac{1}{4} =\left[\int _{a}^{b} xf( x) f'( x)\mathrm{d} x\right]^{2} \leqslant \int _{a}^{b}[ f'( x)]^{2}\mathrm{d} x\times \int _{a}^{b}[ xf( x)]^{2}\mathrm{d} x
\end{gather*}
\qed 



\begin{ques}
	证明对于 $\displaystyle \forall x >0$ 存在唯一的 $\displaystyle \xi _{x}  >0$ 使得 $\displaystyle \int _{0}^{x} e^{t^{2}}\mathrm{d} t=xe^{\xi _{x}^{2}}$ 成立;并求 $\displaystyle \lim _{x\rightarrow +\infty }\frac{\xi _{x}}{x}$。
\end{ques}



首先有定积分第一中值定理,一定存在 $\displaystyle \xi _{x} \in ( 0,x)$ 使得 $\displaystyle \int _{0}^{x} e^{t^{2}}\mathrm{d} t=xe^{\xi _{x}^{2}}$。

对于唯一性,因为 $\displaystyle h( t) =e^{t^{2}}$ 是严格单调上升的的,如果存在 $\displaystyle \xi _{x} ,\xi '_{x}$ 满足条件,就会使得$\displaystyle \int _{0}^{x} e^{t^{2}}\mathrm{d} t$ 有两个取值,于定积分定义矛盾。\qed 



首先,$\displaystyle \int _{0}^{x} e^{t^{2}}\mathrm{d} t=xe^{\xi _{x}^{2}} \Rightarrow \xi _{x} =\sqrt{\ln\frac{\int _{0}^{x} e^{t^{2}}\mathrm{d} t}{x}}$。

考虑
\begin{equation*}
	\begin{aligned}
		\lim _{x\rightarrow +\infty }\frac{\xi _{x}^{2}}{x^{2}} & =\lim _{x\rightarrow +\infty }\frac{\ln\frac{\int _{0}^{x} e^{t^{2}}\mathrm{d} t}{x}}{x^{2}}\\
		& \xlongequal{( 1)}\lim _{x\rightarrow +\infty }\frac{\frac{x}{\int _{0}^{x} e^{t^{2}}\mathrm{d} t}\frac{e{^{x}}^{2} x-\int _{0}^{x} e^{t^{2}}\mathrm{d} t}{x^{2}}}{2x}\\
		& =\lim _{x\rightarrow +\infty }\frac{\frac{e{^{x}}^{2} x}{\int _{0}^{x} e^{t^{2}}\mathrm{d} t} -1}{2x^{2}}\\
		& =\lim _{x\rightarrow +\infty }\frac{e^{x^{2}}}{2x\int _{0}^{x} e^{t^{2}}\mathrm{d} t}\\
		& \xlongequal{( 2)}\lim _{x\rightarrow +\infty }\frac{e^{x^{2}} 2x}{2xe^{x^{2}} +2\int _{0}^{x} e^{t^{2}}\mathrm{d} t}\\
		& =\lim _{x\rightarrow +\infty }\frac{x}{x+\frac{\int _{0}^{x} e^{t^{2}}\mathrm{d} t}{e^{x^{2}}}}\\
		& \xlongequal{( 3)} 1
	\end{aligned}
\end{equation*}


上式中,(1) 是使用洛必达法则,以及 $\displaystyle e{^{t}}^{2}$连续,进而 $\displaystyle \frac{\mathrm{d}}{\mathrm{d} x}\int _{0}^{x} e^{t^{2}}\mathrm{d} t=e^{x^{2}}$。

(2)是使用洛必达法则。

(3)是因为
\begin{equation*}
	\lim _{x\rightarrow +\infty }\frac{\int _{0}^{x} e^{t^{2}}\mathrm{d} t}{e^{x^{2}}} =\lim _{x\rightarrow +\infty }\frac{e^{x^{2}}}{e^{x^{2}} 2x} =0
\end{equation*}


因为 $\displaystyle \lim _{x\rightarrow +\infty }\frac{\xi _{x}^{2}}{x^{2}} =1$,于是 $\displaystyle \lim _{x\rightarrow +\infty }\frac{\xi _{x}}{x} =1$。





\begin{ques}
	计算 $\displaystyle I=\int _{0}^{1}\frac{\ln( 1+x)}{1+x^{2}}\mathrm{d} x$。
\end{ques}



记 $\displaystyle f( a) =\int _{0}^{1}\frac{\ln( 1+ax)}{1+x^{2}}\mathrm{d} x$。

有
\begin{equation*}
	\begin{aligned}
		f'( a) & =\int _{0}^{1}\frac{x}{( 1+ax)\left( 1+x^{2}\right)}\mathrm{d} x\\
		& =\int _{0}^{1}\frac{1}{1+a^{2}}\left(\frac{-a}{1+ax} +\frac{x+a}{1+x^{2}}\right)\mathrm{d} x\\
		& =\frac{1}{1+a^{2}}\left[ -\ln( 1+a) +\frac{1}{2}\ln 2+\frac{\pi a}{4}\right]
	\end{aligned}
\end{equation*}

进而
\begin{equation*}
	\begin{aligned}
		I & =f( 1) -f( 0)\\
		& =\int _{0}^{1} f'( a)\mathrm{d} a\\
		& =\int _{0}^{1}\frac{1}{1+a^{2}}\left[ -\ln( 1+a) +\frac{1}{2}\ln 2+\frac{\pi a}{4}\right]\mathrm{d} a\\
		& =-I+\frac{\pi \ln 2}{8} +\frac{\pi \ln 2}{8}\\
		I & =\frac{\pi \ln 2}{8}
	\end{aligned}
\end{equation*}






\begin{ques}
	计算 $\displaystyle I=\int _{0}^{1}\left[\sqrt[7]{1-x^{3}} -\sqrt[3]{1-x^{7}}\right]\mathrm{d} x$。
\end{ques}

真不会啊。。







\begin{ques}
	计算极限 $\displaystyle \lim _{x\rightarrow +\infty }\frac{\int _{0}^{x} t|\sin t|\mathrm{d} t}{x^{2}}$。
\end{ques}



令 $\displaystyle x=2k\pi +v,v\in [ 0,2\pi )$
\begin{align*}
	\lim _{x\rightarrow +\infty }\frac{\int _{0}^{x} t|\sin t|\mathrm{d} t}{x^{2}} & =\lim _{2k\pi +v\rightarrow +\infty }\frac{\int _{0}^{2k\pi } t|\sin t|\mathrm{d} t+2k\pi \int _{0}^{v} |\sin t|\mathrm{d} t+\int _{0}^{v} t|\sin t|\mathrm{d} t}{( 2k\pi +v)^{2}}\\
	& =\lim _{2k\pi +v\rightarrow +\infty }\frac{\left( k^{2} -k+1\right) 2\pi +2k\pi \int _{0}^{v} |\sin t|\mathrm{d} t+\int _{0}^{v} t|\sin t|\mathrm{d} t}{( 2k\pi +v)^{2}}
\end{align*}

(1)是因为
\begin{align*}
	\int _{0}^{2k\pi } t|\sin t|\mathrm{d} t & =\int _{0}^{\pi }\left( t+2\pi -t+4\pi \sum _{i=0}^{k-1} i\right)\sin t\mathrm{d} t\\
	& =\left( k^{2} -k+1\right) 2\pi \int _{0}^{\pi }\sin t\mathrm{d} t\\
	& =\left( k^{2} -k+1\right) 2\pi 
\end{align*}

而 
\begin{gather*}
	\int _{0}^{v} |\sin t|\mathrm{d} t\leqslant \int _{0}^{2\pi } |\sin t|\mathrm{d} t=4\\
	\int _{0}^{v} t|\sin t|\mathrm{d} t\leqslant 2\pi \int _{0}^{v} |\sin t|\mathrm{d} t=8\pi 
\end{gather*}


于是
\begin{gather*}
	\frac{\left( k^{2} -k+1\right) 2\pi }{( 2k\pi +v)^{2}} \leqslant \frac{\left( k^{2} -k+1\right) 2\pi +2k\pi \int _{0}^{v} |\sin t|\mathrm{d} t+\int _{0}^{v} t|\sin t|\mathrm{d} t}{( 2k\pi +v)^{2}}\\
	\leqslant \frac{\left( k^{2} -k+1\right) 2\pi +8k\pi +8\pi }{( 2k\pi +v)^{2}}
\end{gather*}


又因为
\begin{gather*}
	\lim _{k\rightarrow +\infty }\frac{\left( k^{2} -k+1\right) 2\pi }{( 2k\pi +v)^{2}} =\frac{1}{2\pi }\\
	\lim _{k\rightarrow +\infty }\frac{\left( k^{2} -k+1\right) 2\pi +8k\pi +8\pi }{( 2k\pi +v)^{2}} =\frac{1}{2\pi }
\end{gather*}


由夹逼原理,
\begin{equation*}
	\lim _{x\rightarrow +\infty }\frac{\int _{0}^{x} t|\sin t|\mathrm{d} t}{x^{2}} =\frac{1}{2\pi }
\end{equation*}



\ifx\allfiles\undefined
\end{document}
\fi