%-*-    coding: UTF-8   -*-
% !TEX program = xelatex
\ifx\allfiles\undefined
%\special{dvipdfmx:config z 0}% 取消压缩,加快编译速度
\documentclass[UTF-8]{ctexart}
\usepackage{graphicx}
\usepackage{subfigure}
\usepackage{xcolor}
\usepackage{amsmath}
\usepackage{amssymb}
\usepackage{tabularx}
\usepackage{amssymb}
\usepackage{amsthm}
%\usepackage[usenames,dvipsnames]{color}
\usepackage{hyperref}
\hypersetup{
	colorlinks=true,
	linkcolor=black,
	filecolor=black,      
	urlcolor=red,
	citecolor=black,
}
\usepackage{geometry}
\geometry{a4paper,centering,scale=0.80}
\usepackage[format=hang,font=small,textfont=it]{caption}
\usepackage[nottoc]{tocbibind}
\usepackage{algorithm}  
\usepackage{algorithmicx}  
\usepackage{algpseudocode}
\usepackage{prettyref}
\usepackage{framed}
\setlength{\parindent}{2em}
\usepackage{indentfirst}
\usepackage[framemethod=TikZ]{mdframed}
\newcounter{ques}[section]
\renewcommand{\theques}{\arabic{section}.\arabic{ques}}
\newcommand{\setParDis}{\setlength {\parskip} {0.3cm} }
\newcommand{\setParDef}{\setlength {\parskip} {0pt} }
\setParDis% 调整这一个subsection的段落间距
%\setParDef%恢复间距

\newenvironment{ques}[1][]{
	\refstepcounter{ques}
	\mdfsetup{
		frametitle={
			\tikz[baseline=(current bounding box.east), outer sep=0pt]
			\node[anchor=east,rectangle,fill=blue!20]
			{\strut Problem~\theques\ifstrempty{#1}{}{:~#1}};},
		innertopmargin=10pt,linecolor=blue!20,
		linewidth=2pt,topline=true,
		frametitleaboveskip=\dimexpr-\ht\strutbox\relax
	}
	\begin{mdframed}[]\relax
}{\end{mdframed}}

\newcounter{Thm}[section]
\renewcommand{\theThm}{\arabic{section}.\arabic{Thm}}
\newenvironment{Thm}[1][]{
	\refstepcounter{Thm}
	\mdfsetup{
		frametitle={
			\tikz[baseline=(current bounding box.east), outer sep=0pt]
			\node[anchor=east,rectangle,fill=blue!20]
			{\strut Theorem~\theThm\ifstrempty{#1}{}{:~#1}};},
		innertopmargin=10pt,linecolor=blue!20,
		linewidth=2pt,topline=true,
		frametitleaboveskip=\dimexpr-\ht\strutbox\relax
	}
	\begin{mdframed}[]\relax
}{\end{mdframed}}

\newcounter{Defi}[section]
\renewcommand{\theDefi}{\arabic{section}.\arabic{Defi}}
\newenvironment{Defi}[1][]{
	\refstepcounter{Defi}
	\mdfsetup{
		frametitle={
			\tikz[baseline=(current bounding box.east), outer sep=0pt]
			\node[anchor=east,rectangle,fill=blue!20]
			{\strut Definition~\theDefi\ifstrempty{#1}{}{:~#1}};},
		innertopmargin=10pt,linecolor=blue!20,
		linewidth=2pt,topline=true,
		frametitleaboveskip=\dimexpr-\ht\strutbox\relax
	}
	\begin{mdframed}[]\relax
}{\end{mdframed}}

\newrefformat{qlt}{\underline{性质 \ref{#1}}}
\newcommand{\tpf}[2]{\begin{ques}[#1]{\kaishu #2}\end{ques}}
\newcommand{\pf}[1]{\begin{ques}{\kaishu #1}\end{ques}}
\newcommand{\tthm}[2]{\begin{Thm}[#1]{\kaishu #2}\end{Thm}}
\newcommand{\thm}[1]{\begin{Thm}{\kaishu #1}\end{Thm}}
\newcommand{\tdefi}[2]{\begin{Defi}[#1]{\kaishu #2}\end{Defi}}
\newcommand{\defi}[1]{\begin{Defi}{\kaishu #1}\end{Defi}}
\newcommand{\opf}[1]{{\kaishu{#1}}}
\title{数学分析 I 作业(2024. Spring)}
\author{\texttt{As-The-Wind}}

\date{2024 年 2 月 19 日 $\rightarrow$ \today}

\date{}
\author{尹锦润}
\begin{document}
\maketitle
\fi

\section{2024.5.22 作业}
\begin{ques}
	设 $\displaystyle f( x)$ 在一个无穷区间可被多项式逼近,则 $\displaystyle f( x)$ 必是一个多项式。
\end{ques}



因为 $\displaystyle f( x)$ 在无穷区间可被多项式逼近,因此存在多项式列 $\displaystyle P_{n}( x) \rightrightarrows f( x)$,$\displaystyle \forall \varepsilon  >0,\exists N >0\ s.t.|\ P_{n}( x) -P_{N}( x) |< \varepsilon ,\forall n >N$。

$\displaystyle \ |P_{n}( x) -P_{N}( x) |$ 是一个多项式并且有界,因此是一个常数 $\displaystyle c_{n}$,于是可以固定 $\displaystyle N$,存在一个递增子列 $\displaystyle c_{n_{k}}\rightarrow c,\lim _{k\rightarrow +\infty } |P_{n_{k}}( x) -P_{N}( x) |=c\ i.e.\ f( x) -P_{n}( x) =c$,于是 $\displaystyle f( x)$ 必是一个多项式。\qed 





\begin{ques}
	设 $\displaystyle f( x)$ 在 $\displaystyle [ 0,1]$ 单调递增(减),则其 Bernstein 多项式 $\displaystyle B_{n}( f,x)$ 也在 $\displaystyle [ 0,1]$ 单调递增(减)。
\end{ques}



\vspace{30\baselineskip}





\begin{ques}
	设 $\displaystyle f( x) \in C[ 0,+\infty ) ,\lim _{x\rightarrow +\infty } f( x) =\ell $。证明:$\displaystyle \forall \varepsilon  >0$,存在多项式 $\displaystyle P( x)$ 使得 $\displaystyle |f( x) -P\left( e^{-x}\right) |< \varepsilon ,x\in [ 1,+\infty )$。
\end{ques}

令
\begin{equation*}
	g( x) =\begin{cases}
		f( -\ln x) & \left( 0,\frac{1}{e}\right]\\
		\ell  & 0
	\end{cases}
\end{equation*}

则 $\displaystyle g( x) \in C\left[ 0,\frac{1}{e}\right]$ ,于是 $\displaystyle \forall \varepsilon  >0,\exists $ 多项式 $\displaystyle P( x) \ s.t.\ |g( x) -P( x) |< \varepsilon ,\forall x\in \left[ 0,\frac{1}{e}\right]$。

令 $\displaystyle t=-\ln x\in [ 1,+\infty )$,有 $\displaystyle |f( t) -P\left( e^{-x}\right) |< \varepsilon ,x\in [ 1,+\infty )$。\qed 
\ifx\allfiles\undefined
\end{document}
\fi