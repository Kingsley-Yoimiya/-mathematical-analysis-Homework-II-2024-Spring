%-*-    coding: UTF-8   -*-
% !TEX program = xelatex
\ifx\allfiles\undefined
%\special{dvipdfmx:config z 0}% 取消压缩,加快编译速度
\documentclass[UTF-8]{ctexart}
\usepackage{graphicx}
\usepackage{subfigure}
\usepackage{xcolor}
\usepackage{amsmath}
\usepackage{amssymb}
\usepackage{tabularx}
\usepackage{amssymb}
\usepackage{amsthm}
%\usepackage[usenames,dvipsnames]{color}
\usepackage{hyperref}
\hypersetup{
	colorlinks=true,
	linkcolor=black,
	filecolor=black,      
	urlcolor=red,
	citecolor=black,
}
\usepackage{geometry}
\geometry{a4paper,centering,scale=0.80}
\usepackage[format=hang,font=small,textfont=it]{caption}
\usepackage[nottoc]{tocbibind}
\usepackage{algorithm}  
\usepackage{algorithmicx}  
\usepackage{algpseudocode}
\usepackage{prettyref}
\usepackage{framed}
\setlength{\parindent}{2em}
\usepackage{indentfirst}
\usepackage[framemethod=TikZ]{mdframed}
\newcounter{ques}[section]
\renewcommand{\theques}{\arabic{section}.\arabic{ques}}
\newcommand{\setParDis}{\setlength {\parskip} {0.3cm} }
\newcommand{\setParDef}{\setlength {\parskip} {0pt} }
\setParDis% 调整这一个subsection的段落间距
%\setParDef%恢复间距

\newenvironment{ques}[1][]{
	\refstepcounter{ques}
	\mdfsetup{
		frametitle={
			\tikz[baseline=(current bounding box.east), outer sep=0pt]
			\node[anchor=east,rectangle,fill=blue!20]
			{\strut Problem~\theques\ifstrempty{#1}{}{:~#1}};},
		innertopmargin=10pt,linecolor=blue!20,
		linewidth=2pt,topline=true,
		frametitleaboveskip=\dimexpr-\ht\strutbox\relax
	}
	\begin{mdframed}[]\relax
}{\end{mdframed}}

\newcounter{Thm}[section]
\renewcommand{\theThm}{\arabic{section}.\arabic{Thm}}
\newenvironment{Thm}[1][]{
	\refstepcounter{Thm}
	\mdfsetup{
		frametitle={
			\tikz[baseline=(current bounding box.east), outer sep=0pt]
			\node[anchor=east,rectangle,fill=blue!20]
			{\strut Theorem~\theThm\ifstrempty{#1}{}{:~#1}};},
		innertopmargin=10pt,linecolor=blue!20,
		linewidth=2pt,topline=true,
		frametitleaboveskip=\dimexpr-\ht\strutbox\relax
	}
	\begin{mdframed}[]\relax
}{\end{mdframed}}

\newcounter{Defi}[section]
\renewcommand{\theDefi}{\arabic{section}.\arabic{Defi}}
\newenvironment{Defi}[1][]{
	\refstepcounter{Defi}
	\mdfsetup{
		frametitle={
			\tikz[baseline=(current bounding box.east), outer sep=0pt]
			\node[anchor=east,rectangle,fill=blue!20]
			{\strut Definition~\theDefi\ifstrempty{#1}{}{:~#1}};},
		innertopmargin=10pt,linecolor=blue!20,
		linewidth=2pt,topline=true,
		frametitleaboveskip=\dimexpr-\ht\strutbox\relax
	}
	\begin{mdframed}[]\relax
}{\end{mdframed}}

\newrefformat{qlt}{\underline{性质 \ref{#1}}}
\newcommand{\tpf}[2]{\begin{ques}[#1]{\kaishu #2}\end{ques}}
\newcommand{\pf}[1]{\begin{ques}{\kaishu #1}\end{ques}}
\newcommand{\tthm}[2]{\begin{Thm}[#1]{\kaishu #2}\end{Thm}}
\newcommand{\thm}[1]{\begin{Thm}{\kaishu #1}\end{Thm}}
\newcommand{\tdefi}[2]{\begin{Defi}[#1]{\kaishu #2}\end{Defi}}
\newcommand{\defi}[1]{\begin{Defi}{\kaishu #1}\end{Defi}}
\newcommand{\opf}[1]{{\kaishu{#1}}}
\title{数学分析 I 作业(2024. Spring)}
\author{\texttt{As-The-Wind}}

\date{2024 年 2 月 19 日 $\rightarrow$ \today}

\date{}
\author{尹锦润}
\begin{document}
\maketitle
\fi

\section{2024.3.13 作业}

\begin{ques}
	求曲线 $\displaystyle r=a\sin^{n}\left(\frac{\theta }{n}\right) ,0\leqslant \theta \leqslant n\pi $ 的弧长。
\end{ques}



弧长为


\begin{align*}
	C & =\int _{0}^{n\pi }\sqrt{\left( r^{2}\right) +( r')^{2}}\mathrm{d} \theta \\
	& =\int _{0}^{n\pi } a\sin^{n-1}\left(\frac{\theta }{n}\right)\mathrm{d} \theta \\
	& =aF( n,n-1)
\end{align*}


对于 $\displaystyle F( n,m)$,有
\begin{align*}
	F( n,m) & =\int _{0}^{n\pi }\sin^{m}\left(\frac{\theta }{n}\right)\mathrm{d} \theta \\
	& =-\int _{0}^{n\pi }\sin^{m-1}\left(\frac{\theta }{n}\right)\mathrm{d} n\cos\left(\frac{\theta }{n}\right)\\
	& =-\left(\left( n\cos\left(\frac{\theta }{n}\right)\sin^{m-1}\left(\frac{\theta }{n}\right)\middle| _{0}^{n\pi }\right) -\int _{0}^{n\pi }( m-1)\cos^{2}\left(\frac{\theta }{n}\right)\sin^{m-2}\left(\frac{\theta }{n}\right)\mathrm{d} \theta \right)\\
	& =( m-1)\left(\int _{0}^{n\pi }\sin^{m-2}\left(\frac{\theta }{n}\right)\mathrm{d} \theta -\int _{0}^{n\pi }\sin^{m}\left(\frac{\theta }{n}\right)\mathrm{d} \theta \right)\\
	& =( m-1)( F( n,m-2) -F( n,m))\\
	& =\frac{m-1}{m} F( n,m-2)\\
	F( n,0) & =n\pi \\
	F( n,1) & =2n\\
	F( n,n-1) & =\begin{cases}
		\frac{n-2}{n-1} \times \frac{n-4}{n-3} \times \cdots \times \frac{1}{2} n\pi  & ,( n-1)\bmod 2=0\\
		\frac{n-2}{n-1} \times \frac{n-4}{n-3} \times \cdots \times \frac{2}{3} 2n & ,( n-1)\bmod 2=1
	\end{cases}\\
	C & =\begin{cases}
		a\frac{n-2}{n-1} \times \frac{n-4}{n-3} \times \cdots \times \frac{1}{2} n\pi  & ,( n-1)\bmod 2=0\\
		a\frac{n-2}{n-1} \times \frac{n-4}{n-3} \times \cdots \times \frac{2}{3} 2n & ,( n-1)\bmod 2=1
	\end{cases}
\end{align*}




\begin{ques}
	求曲线 $\displaystyle \frac{x^{2}}{a^{2}} +\frac{y^{2}}{b^{2}} =1$ 绕 $\displaystyle x$ 轴旋转一周所成的旋转面的面积。
\end{ques}




\begin{align*}
	\frac{1}{2} S & =\int _{0}^{\frac{\pi }{2}} 2\pi |b|\sin \theta \sqrt{a^{2}\sin^{2} \theta +b^{2}\cos^{2} \theta }\mathrm{d} \theta \\
	& =2\pi |b|\int _{0}^{1}\sqrt{a^{2} +\left( b^{2} -a^{2}\right) t^{2}}\mathrm{d} t
\end{align*}
当 $\displaystyle b^{2}  >a^{2}$ 时:


\begin{align*}
	\frac{1}{2} S & =2\pi |b|\int _{0}^{1}\sqrt{a^{2} +\left( b^{2} -a^{2}\right) t^{2}}\mathrm{d} t\\
	& =2\pi |b|\int _{0}^{\sqrt{\frac{b^{2} -a^{2}}{a^{2}}}} |a|\frac{1}{\cos \theta }\sqrt{\frac{a^{2}}{b^{2} -a^{2}}}\mathrm{d}\tan \theta \\
	& =2\pi |b||a|\sqrt{\frac{a^{2}}{b^{2} -a^{2}}}\int _{0}^{\arctan\sqrt{\frac{b^{2} -a^{2}}{a^{2}}}}\frac{1}{\cos^{3} \theta }\mathrm{d} \theta \\
	& =2\pi |b|\frac{a^{2}}{\sqrt{b^{2} -a^{2}}}\left(\frac{1}{2}\frac{\tan \theta }{\cos \theta } +\frac{1}{4}\ln\left| \frac{1+\sin \theta }{1-\sin \theta }\right| \middle| _{\theta =0}^{\theta =\arctan\sqrt{\frac{b^{2} -a^{2}}{a^{2}}}}\right)\\
	& =\pi |b|\left( |b|+\frac{a^{2}}{\sqrt{b^{2} -a^{2}}}\ln\left| \frac{|b|+\sqrt{b^{2} -a^{2}}}{a}\right| \right)\\
	S & =2\pi |b|\left( |b|+\frac{a^{2}}{\sqrt{b^{2} -a^{2}}}\ln\left| \frac{|b|+\sqrt{b^{2} -a^{2}}}{a}\right| \right)
\end{align*}


当 $\displaystyle b^{2} \leqslant a^{2}$ 时,
\begin{align*}
	\frac{1}{2} S & =2\pi |b|\int _{0}^{1}\sqrt{a^{2} -\left( a^{2} -b^{2}\right) t^{2}}\mathrm{d} t\\
	& =2\pi |b||a|\mathrm{\sqrt{\frac{a^{2}}{a^{2} -b^{2}}}}\int _{0}^{\arcsin\sqrt{\frac{a^{2} -b^{2}}{a^{2}}}}\cos^{2} \theta \mathrm{d} \theta \\
	& =2\pi |b|\frac{a^{2}}{\sqrt{a^{2} -b^{2}}}\left(\left(\frac{\sin 2\theta }{4} +\frac{\theta }{2}\right)\middle| _{0}^{\arcsin\sqrt{\frac{a^{2} -b^{2}}{a^{2}}}}\right)\\
	& =\pi |b|\left( |b|+\frac{a^{2}}{\sqrt{a^{2} -b^{2}}}\arcsin\frac{\sqrt{a^{2} -b^{2}}}{|a|}\right)\\
	S & =2\pi |b|\left( |b|+\frac{a^{2}}{\sqrt{a^{2} -b^{2}}}\arcsin\frac{\sqrt{a^{2} -b^{2}}}{|a|}\right)
\end{align*}


\begin{ques}
	求双纽线 $\displaystyle r^{2} =a^{2}\cos 2\theta ( a >0)$ 绕 $\displaystyle \theta =\frac{\pi }{2}$ 旋转一周的面积。
\end{ques}
\begin{align*}
	S & =2\int _{0}^{\frac{\pi }{4}} 2\pi ( r\cos \theta )\sqrt{\mathrm{d} r^{2} +r^{2}\mathrm{d} \theta ^{2}}\\
	& =2\int _{0}^{\frac{\pi }{4}} 2\pi a^{2}\cos \theta \mathrm{d} \theta \\
	& =4\pi a^{2}\int _{0}^{\frac{\pi }{4}}\cos \theta \mathrm{d} \theta \\
	& =4\pi a^{2}\frac{\sqrt{2}}{2}\\
	& =2\sqrt{2} \pi a^{2}
\end{align*}




\begin{ques}
	设 $\displaystyle f'( x) \in C[ a,b]$,证明 $\displaystyle f( x) \in BV[ a,b]$ 且 $\displaystyle \bigvee _{a}^{b} f( x) =\int _{a}^{b} |f'( x) |\mathrm{d} x$。
\end{ques}



首先,因为 $\displaystyle f'( x) \in C[ a,b]$,故 $\displaystyle \forall x< y,f( y) -f( x) =\int _{x}^{y} f'( t)\mathrm{d} t$。

同时,有 $\displaystyle \left| \int _{x}^{y} f'( t)\mathrm{d} t\right| \leqslant \int _{x}^{y} |f'( t) |\mathrm{d} t$。

对于分割 $\displaystyle \Delta :a=x_{0} < x_{1} < \cdots < x_{n} =b$,有
\begin{align*}
	\sum _{i=1}^{n} |f( x_{i}) -f( x_{i-1}) | & =\sum _{i=1}^{n}\left| \int _{x_{i-1}}^{x_{i}} f'( t)\mathrm{d} t\right| \\
	& \leqslant \sum _{i=1}^{n}\int _{x_{i-1}}^{x_{i}} |f'( t) |\mathrm{d} t\\
	& =\int _{a}^{b} |f'( x) |\mathrm{d} x
\end{align*}

而 $\displaystyle \bigvee _{a}^{b} f( x) =\sup _{\Delta }\left\{\sum _{i=1}^{n} |f( x_{i}) -f( x_{i-1}) |\right\} \leqslant \int _{a}^{b} |f'( x) |\mathrm{d} x$。

因为 $\displaystyle [ a,b]$ 上 $\displaystyle f'( x)$ 连续,有界,进而 $\displaystyle \int _{a}^{b} |f'( x) |\mathrm{d} x$ 有界,$\displaystyle f( x) \in BV[ a,b]$。

同时,考虑分割 $\displaystyle \Delta _{1}$,如果 $\displaystyle f'( t) =0$,那么 $\displaystyle t\in \Delta _{1}$,可以发现在上面的等式中,唯一的不等号此时可以取等,因此 $\displaystyle \Delta _{1}$可以使得 $\displaystyle \sup _{\Delta }\left\{\sum _{i=1}^{n} |f( x_{i}) -f( x_{i-1}) |\right\} \geqslant \int _{a}^{b} |f'( x) |\mathrm{d} x$。

进而 $\displaystyle \bigvee _{a}^{b} f( x) =\int _{a}^{b} |f'( x) |\mathrm{d} x$。\qed 
\ifx\allfiles\undefined
\end{document}
\fi