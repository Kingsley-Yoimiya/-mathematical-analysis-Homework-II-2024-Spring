%-*-    coding: UTF-8   -*-
% !TEX program = xelatex
\ifx\allfiles\undefined
%\special{dvipdfmx:config z 0}% 取消压缩,加快编译速度
\documentclass[UTF-8]{ctexart}
\usepackage{graphicx}
\usepackage{subfigure}
\usepackage{xcolor}
\usepackage{amsmath}
\usepackage{amssymb}
\usepackage{tabularx}
\usepackage{amssymb}
\usepackage{amsthm}
%\usepackage[usenames,dvipsnames]{color}
\usepackage{hyperref}
\hypersetup{
	colorlinks=true,
	linkcolor=black,
	filecolor=black,      
	urlcolor=red,
	citecolor=black,
}
\usepackage{geometry}
\geometry{a4paper,centering,scale=0.80}
\usepackage[format=hang,font=small,textfont=it]{caption}
\usepackage[nottoc]{tocbibind}
\usepackage{algorithm}  
\usepackage{algorithmicx}  
\usepackage{algpseudocode}
\usepackage{prettyref}
\usepackage{framed}
\setlength{\parindent}{2em}
\usepackage{indentfirst}
\usepackage[framemethod=TikZ]{mdframed}
\newcounter{ques}[section]
\renewcommand{\theques}{\arabic{section}.\arabic{ques}}
\newcommand{\setParDis}{\setlength {\parskip} {0.3cm} }
\newcommand{\setParDef}{\setlength {\parskip} {0pt} }
\setParDis% 调整这一个subsection的段落间距
%\setParDef%恢复间距

\newenvironment{ques}[1][]{
	\refstepcounter{ques}
	\mdfsetup{
		frametitle={
			\tikz[baseline=(current bounding box.east), outer sep=0pt]
			\node[anchor=east,rectangle,fill=blue!20]
			{\strut Problem~\theques\ifstrempty{#1}{}{:~#1}};},
		innertopmargin=10pt,linecolor=blue!20,
		linewidth=2pt,topline=true,
		frametitleaboveskip=\dimexpr-\ht\strutbox\relax
	}
	\begin{mdframed}[]\relax
}{\end{mdframed}}

\newcounter{Thm}[section]
\renewcommand{\theThm}{\arabic{section}.\arabic{Thm}}
\newenvironment{Thm}[1][]{
	\refstepcounter{Thm}
	\mdfsetup{
		frametitle={
			\tikz[baseline=(current bounding box.east), outer sep=0pt]
			\node[anchor=east,rectangle,fill=blue!20]
			{\strut Theorem~\theThm\ifstrempty{#1}{}{:~#1}};},
		innertopmargin=10pt,linecolor=blue!20,
		linewidth=2pt,topline=true,
		frametitleaboveskip=\dimexpr-\ht\strutbox\relax
	}
	\begin{mdframed}[]\relax
}{\end{mdframed}}

\newcounter{Defi}[section]
\renewcommand{\theDefi}{\arabic{section}.\arabic{Defi}}
\newenvironment{Defi}[1][]{
	\refstepcounter{Defi}
	\mdfsetup{
		frametitle={
			\tikz[baseline=(current bounding box.east), outer sep=0pt]
			\node[anchor=east,rectangle,fill=blue!20]
			{\strut Definition~\theDefi\ifstrempty{#1}{}{:~#1}};},
		innertopmargin=10pt,linecolor=blue!20,
		linewidth=2pt,topline=true,
		frametitleaboveskip=\dimexpr-\ht\strutbox\relax
	}
	\begin{mdframed}[]\relax
}{\end{mdframed}}

\newrefformat{qlt}{\underline{性质 \ref{#1}}}
\newcommand{\tpf}[2]{\begin{ques}[#1]{\kaishu #2}\end{ques}}
\newcommand{\pf}[1]{\begin{ques}{\kaishu #1}\end{ques}}
\newcommand{\tthm}[2]{\begin{Thm}[#1]{\kaishu #2}\end{Thm}}
\newcommand{\thm}[1]{\begin{Thm}{\kaishu #1}\end{Thm}}
\newcommand{\tdefi}[2]{\begin{Defi}[#1]{\kaishu #2}\end{Defi}}
\newcommand{\defi}[1]{\begin{Defi}{\kaishu #1}\end{Defi}}
\newcommand{\opf}[1]{{\kaishu{#1}}}
\title{数学分析 I 作业(2024. Spring)}
\author{\texttt{As-The-Wind}}

\date{2024 年 2 月 19 日 $\rightarrow$ \today}

\date{}
\author{尹锦润}
\begin{document}
\maketitle
\fi

\section{2024.3.11 作业}

\begin{ques}
	求区域 $\displaystyle y^{2} \leqslant x,x^{2} +y^{2} \leqslant 1$ 公共部分的面积。
\end{ques}



将其分解为 $\displaystyle x\leqslant \frac{-1+\sqrt{5}}{2}$ 区域和 $\displaystyle x\geqslant \frac{-1+\sqrt{5}}{2}$(一个弓形)。
\begin{align*}
	S & =\left( 2\int _{0}^{\frac{-1+\sqrt{5}}{2}}\sqrt{x}\mathrm{d} x\right) +\arccos\left(\frac{-1+\sqrt{5}}{2}\right) -\left(\frac{-1+\sqrt{5}}{2}\right) \times \sqrt{1-\left(\frac{-1+\sqrt{5}}{2}\right)^{2}}\\
	& =\frac{\sqrt{2}\left( -1+\sqrt{5}\right)^{\frac{3}{2}}}{3} +\arccos\left(\frac{-1+\sqrt{5}}{2}\right) -\left(\frac{-1+\sqrt{5}}{2}\right)^{\frac{3}{2}}\\
	& =\arccos\left(\frac{-1+\sqrt{5}}{2}\right) +\frac{\left( -1+\sqrt{5}\right)^{\frac{3}{2}}}{6\sqrt{2}}
\end{align*}




\begin{ques}
	记四条平面曲线 $\displaystyle x=0,x=1,y=x^{3} ,y=t( t\in [ 0,1])$ 所围成的面积为 $\displaystyle S_{t}$,计算 $\displaystyle S_{t}$ 的最小值。
\end{ques}


\begin{align*}
	S_{t} & =\int _{0}^{1} |t-x^{3} |\mathrm{d} x\\
	& =\int _{0}^{\sqrt[3]{t}}\left( t-x^{3}\right)\mathrm{d} x+\int _{\sqrt[3]{t}}^{1}\left( x^{3} -t\right)\mathrm{d} x\\
	& =\left( tx-\frac{1}{4} x^{4}\middle| _{0}^{\sqrt[3]{t}}\right) +\left(\frac{1}{4} x^{4} -tx\middle| _{\sqrt[3]{t}}^{1}\right)\\
	& =2t\sqrt[3]{t} -\frac{1}{2}\left(\sqrt[3]{t}\right)^{4} +\frac{1}{4} -t\\
	& =\frac{3}{2} t^{\frac{4}{3}} -t+\frac{1}{4}
\end{align*}

考虑


\begin{align*}
	\mathrm{\frac{\mathrm{d} S_{t}}{\mathrm{d} t}} & =2t^{\frac{1}{3}} -1
\end{align*}


因此 $\displaystyle S_{t}$ 在 $\displaystyle t=\frac{1}{8}$ 处取得最小值,为 $\displaystyle \frac{7}{32}$。



\begin{ques}
	求笛卡尔叶型线 $\displaystyle r=\frac{3a\sin \varphi \cos \varphi }{\sin^{3} \varphi +\cos^{3} \varphi }( a >0)$ 围成的图形面积。
\end{ques}




\begin{align*}
	S & =\int _{0}^{\frac{\pi }{2}}\frac{1}{2} r^{2}\mathrm{d} \varphi \\
	& =\frac{9a^{2}}{2}\int _{0}^{\frac{\pi }{2}}\mathrm{\frac{\sin^{2} \varphi \cos^{2} \varphi }{\left(\sin^{3} \varphi +\cos^{3} \varphi \right)^{2}} d} \varphi \\
	& =\frac{9a^{2}}{2}\int _{0}^{+\infty }\mathrm{\frac{\tan^{2} \varphi }{\left(\tan^{3} \varphi +1\right)^{2}} d}\tan \varphi \\
	& =\frac{9a^{2}}{2}\left( -\frac{1}{3}\frac{1}{t^{3} +1}\middle| _{0}^{+\infty }\right)\\
	& =\frac{3a^{2}}{2}
\end{align*}




\begin{ques}
	求心脏线的一段 $\displaystyle r=a( 1+\cos \theta )\left( 0\leqslant \theta \leqslant \frac{\pi }{2}\right)$ 于极轴和射线 $\displaystyle \theta =\frac{\pi }{2}$ 所围成的平面区域绕极轴旋转所得立体的体积。
\end{ques}



利用球坐标系中的体积公式:
\begin{align*}
	V & =\int _{0}^{\frac{\pi }{2}}\mathrm{d} \theta \int _{0}^{2\pi }\mathrm{d} \varphi \mathrm{\int _{0}^{a( 1+\cos \theta )}} r^{2}\sin \theta \mathrm{d} r\\
	& =\int _{0}^{\frac{\pi }{2}}\mathrm{d} \theta \int _{0}^{2\pi }\mathrm{d} \varphi \frac{1}{3}( a( 1+\cos \theta ))^{3}\sin \theta \\
	& =\frac{a^{3}}{3}\int _{0}^{2\pi }\mathrm{d} \varphi \int _{0}^{\frac{\pi }{2}}\mathrm{( 1+\cos \theta )^{3}\sin \theta d} \theta \\
	& =\frac{a^{3}}{3}\int _{0}^{2\pi }\mathrm{d} \varphi \frac{15}{4}\\
	& =\frac{5a^{3} \pi }{2}
\end{align*}

\ifx\allfiles\undefined
\end{document}
\fi