%-*-    coding: UTF-8   -*-
% !TEX program = xelatex
\ifx\allfiles\undefined
%\special{dvipdfmx:config z 0}% 取消压缩,加快编译速度
\documentclass[UTF-8]{ctexart}
\usepackage{graphicx}
\usepackage{subfigure}
\usepackage{xcolor}
\usepackage{amsmath}
\usepackage{amssymb}
\usepackage{tabularx}
\usepackage{amssymb}
\usepackage{amsthm}
%\usepackage[usenames,dvipsnames]{color}
\usepackage{hyperref}
\hypersetup{
	colorlinks=true,
	linkcolor=black,
	filecolor=black,      
	urlcolor=red,
	citecolor=black,
}
\usepackage{geometry}
\geometry{a4paper,centering,scale=0.80}
\usepackage[format=hang,font=small,textfont=it]{caption}
\usepackage[nottoc]{tocbibind}
\usepackage{algorithm}  
\usepackage{algorithmicx}  
\usepackage{algpseudocode}
\usepackage{prettyref}
\usepackage{framed}
\setlength{\parindent}{2em}
\usepackage{indentfirst}
\usepackage[framemethod=TikZ]{mdframed}
\newcounter{ques}[section]
\renewcommand{\theques}{\arabic{section}.\arabic{ques}}
\newcommand{\setParDis}{\setlength {\parskip} {0.3cm} }
\newcommand{\setParDef}{\setlength {\parskip} {0pt} }
\setParDis% 调整这一个subsection的段落间距
%\setParDef%恢复间距

\newenvironment{ques}[1][]{
	\refstepcounter{ques}
	\mdfsetup{
		frametitle={
			\tikz[baseline=(current bounding box.east), outer sep=0pt]
			\node[anchor=east,rectangle,fill=blue!20]
			{\strut Problem~\theques\ifstrempty{#1}{}{:~#1}};},
		innertopmargin=10pt,linecolor=blue!20,
		linewidth=2pt,topline=true,
		frametitleaboveskip=\dimexpr-\ht\strutbox\relax
	}
	\begin{mdframed}[]\relax
}{\end{mdframed}}

\newcounter{Thm}[section]
\renewcommand{\theThm}{\arabic{section}.\arabic{Thm}}
\newenvironment{Thm}[1][]{
	\refstepcounter{Thm}
	\mdfsetup{
		frametitle={
			\tikz[baseline=(current bounding box.east), outer sep=0pt]
			\node[anchor=east,rectangle,fill=blue!20]
			{\strut Theorem~\theThm\ifstrempty{#1}{}{:~#1}};},
		innertopmargin=10pt,linecolor=blue!20,
		linewidth=2pt,topline=true,
		frametitleaboveskip=\dimexpr-\ht\strutbox\relax
	}
	\begin{mdframed}[]\relax
}{\end{mdframed}}

\newcounter{Defi}[section]
\renewcommand{\theDefi}{\arabic{section}.\arabic{Defi}}
\newenvironment{Defi}[1][]{
	\refstepcounter{Defi}
	\mdfsetup{
		frametitle={
			\tikz[baseline=(current bounding box.east), outer sep=0pt]
			\node[anchor=east,rectangle,fill=blue!20]
			{\strut Definition~\theDefi\ifstrempty{#1}{}{:~#1}};},
		innertopmargin=10pt,linecolor=blue!20,
		linewidth=2pt,topline=true,
		frametitleaboveskip=\dimexpr-\ht\strutbox\relax
	}
	\begin{mdframed}[]\relax
}{\end{mdframed}}

\newrefformat{qlt}{\underline{性质 \ref{#1}}}
\newcommand{\tpf}[2]{\begin{ques}[#1]{\kaishu #2}\end{ques}}
\newcommand{\pf}[1]{\begin{ques}{\kaishu #1}\end{ques}}
\newcommand{\tthm}[2]{\begin{Thm}[#1]{\kaishu #2}\end{Thm}}
\newcommand{\thm}[1]{\begin{Thm}{\kaishu #1}\end{Thm}}
\newcommand{\tdefi}[2]{\begin{Defi}[#1]{\kaishu #2}\end{Defi}}
\newcommand{\defi}[1]{\begin{Defi}{\kaishu #1}\end{Defi}}
\newcommand{\opf}[1]{{\kaishu{#1}}}
\title{数学分析 I 作业(2024. Spring)}
\author{\texttt{As-The-Wind}}

\date{2024 年 2 月 19 日 $\rightarrow$ \today}

\date{}
\author{尹锦润}
\begin{document}
\maketitle
\fi

\section{2024.3.20 作业}

\begin{ques}
	讨论积分的收敛性与绝对收敛性:$\displaystyle \int _{1}^{+\infty }\frac{\sin x}{x^{\alpha } +\sin x}\mathrm{d} x( \alpha  >0)$。
\end{ques}



$\displaystyle \left| \frac{\sin x}{x^{\alpha } +\sin x}\right| \leqslant \left| \frac{1}{x^{\alpha } -1}\right| $,因此 $\displaystyle \alpha  >1$ 时,该积分绝对收敛。


\begin{align*}
	\int _{1}^{+\infty }\frac{\sin x}{x^{\alpha } +\sin x}\mathrm{d} x & =\int _{1}^{+\infty }\left(\frac{\sin x}{x^{\alpha }} -\frac{\sin^{2} x}{x^{\alpha }\left( x^{\alpha } +\sin x\right)}\right)\mathrm{d} x\\
	& =I_{1} -I_{2}
\end{align*}

对于 $\displaystyle I_{1} =\int _{1}^{+\infty }\frac{\sin x}{x^{\alpha }}\mathrm{d} x$,我们知道在 $\displaystyle \alpha  >1$ 积分绝对收敛,$\displaystyle \alpha \leqslant 1$ 时条件收敛。

对于 $\displaystyle I_{2} =\int _{1}^{+\infty }\frac{\sin^{2} x}{x^{\alpha }\left( x^{\alpha } +\sin x\right)}\mathrm{d} x$,在 $\displaystyle 0< \alpha \leqslant \frac{1}{2}$,有:
\begin{align*}
	\frac{\sin^{2} x}{x^{\alpha }\left( x^{\alpha } +\sin x\right)} & \geqslant \frac{\sin^{2} x}{x^{\alpha }\left( x^{\alpha } +1\right)}
\end{align*}


同时
\begin{equation*}
	\int _{1}^{+\infty }\frac{\sin^{2} x}{x^{\alpha }\left( x^{\alpha } +1\right)}\mathrm{d} x
\end{equation*}

发散,因此此时 $\displaystyle I_{2}$ 发散。

当 $\displaystyle \alpha  >\frac{1}{2}$,有
\begin{align*}
	\frac{\sin^{2} x}{x^{\alpha }\left( x^{\alpha } +\sin x\right)} & \leqslant \frac{1}{x^{\alpha }\left( x^{\alpha } -1\right)}
\end{align*}

同时
\begin{equation*}
	\int _{1}^{+\infty }\frac{1}{x^{\alpha }\left( x^{\alpha } -1\right)}\mathrm{d} x
\end{equation*}

绝对收敛,此时 $\displaystyle I_{2}$ 绝对收敛。

原来积分在 $\displaystyle \alpha  >1$ 时绝对收敛,在 $\displaystyle \frac{1}{2} < \alpha \leqslant 1$ 时条件收敛,在 $\displaystyle \alpha \leqslant \frac{1}{2}$ 发散。\qed 





\begin{ques}
	(1)讨论积分 $\displaystyle I_{1} =\int _{0}^{+\infty }\frac{\cos x}{1+x}\mathrm{d} x$ 与 $\displaystyle I_{2} =\int _{0}^{+\infty }\frac{\sin x}{( 1+x)^{2}}\mathrm{d} x$ 的收敛性和绝对收敛性。

(2)证明 $\displaystyle I_{1} =I_{2}$。
\end{ques}



(1)

对于 $\displaystyle I_{2}$,有
\begin{align*}
	\left| \frac{\sin x}{( 1+x)^{2}}\right|  & \leqslant \frac{1}{( 1+x)^{2}}
\end{align*}


而 $\displaystyle \int _{0}^{+\infty }\mathrm{\frac{1}{( 1+x)^{2}} d} x$ 收敛,因此 $\displaystyle I_{2}$ 绝对收敛。

对于 $\displaystyle I_{1}$,有 $\displaystyle \left| \int _{0}^{+\infty }\cos x\mathrm{d} x\right| \leqslant 2,\frac{1}{1+x}$ 单调下降且 $\displaystyle \rightarrow 0\left( x\rightarrow +\infty \right)$,由狄利克雷判别法可知其收敛。对于其绝对收敛性,其与 $\displaystyle \int _{0}^{+\infty }\frac{|\cos x|}{x}\mathrm{d} x$ 收敛性相同,因此条件收敛。

故 $\displaystyle I_{1}$ 条件收敛。



(2)


\begin{align*}
	I_{1} & =\int _{0}^{+\infty }\frac{\cos x}{1+x}\mathrm{d} x\\
	& =\left(\frac{\sin x}{1+x}\middle| _{0}^{+\infty }\right) +\int _{0}^{+\infty }\frac{\sin x}{( 1+x)^{2}}\mathrm{dx}\\
	& =\int _{0}^{+\infty }\frac{\sin x}{( 1+x)^{2}}\mathrm{d} x\\
	& =I_{2}
\end{align*}
\qed 



\begin{ques}
	设两个单调下降的正函数 $\displaystyle f( x) ,g( x) \in C[ 0,+\infty ) ,\int _{0}^{+\infty } f( x)\mathrm{d} x,\int _{0}^{+\infty } g( x)\mathrm{d} x$ 都发散,记 $\displaystyle h( x) =\min\{f( x) ,g( x)\} ,x\in [ 0,+\infty )$,问 $\displaystyle \int _{0}^{+\infty } h( x)\mathrm{d} x$ 是否也必定发散。
\end{ques}



不一定,令 $\displaystyle h( x) =\begin{cases}
	1 & ,x\in [ 0,1]\\
	\frac{1}{x^{2}} & ,x\in ( 1,+\infty )
\end{cases}$,同时构造数列 $\displaystyle a_{n} =2^{2^{n}} ,s_{n} =\sum _{i=0}^{n-1} a_{i}$,定义


\begin{align*}
	f( x) & =\begin{cases}
		h( x) & ,t\bmod 2=1\\
		\frac{1}{s_{t}^{2}} & ,t\bmod 2=0
	\end{cases} ,x\in [ s_{t} ,s_{t+1}]\\
	g( x) & =\begin{cases}
		h( x) & ,t\bmod 2=0\\
		\frac{1}{s_{t}^{2}} & ,t\bmod 2=1
	\end{cases} ,x\in [ s_{t} ,s_{t+1}]
\end{align*}


以 $\displaystyle f( x)$ 为例,对于任何 $\displaystyle X >0$,都可以找到 $\displaystyle t\ s.t.\ s_{t} \geqslant X\land t\bmod 2=0$,进而有 $\displaystyle \int _{s_{t}}^{s_{t+1}} f( x)\mathrm{d} x=\frac{1}{s_{t}^{2}} a_{t} =\frac{2^{2^{t}}}{2^{2^{0}} +2^{2^{1}} +\cdots +2^{2^{t-1}}} \geqslant 1$。进而 $\displaystyle f( x)$ 发散,类似地可以推出 $\displaystyle g( x)$ 发散,但是根据定义 $\displaystyle \int _{0}^{+\infty } h( x)\mathrm{d} x$ 收敛。



\begin{ques}
	设 $\displaystyle f( x)$ 在区间 $\displaystyle [ a,+\infty )$ 上单调,并且 $\displaystyle \int _{a}^{+\infty } f( x)\mathrm{d} x$ 收敛,证明:$\displaystyle \lim _{p\rightarrow \infty }\int _{a}^{+\infty } f( x)\sin px\mathrm{d} x=0$。
\end{ques}



首先证明 $\displaystyle \lim _{x\rightarrow +\infty } f( x) =0$:

不妨假设 $\displaystyle f( x)$ 单调下降,令 $\displaystyle \lim _{x\rightarrow +\infty } f( x) =m$,如果 $\displaystyle m >0$ 则 $\displaystyle \int _{a}^{t} f( x)\mathrm{d} x\geqslant m( t-a)$,因此 $\displaystyle \int _{a}^{+\infty } f( x)\mathrm{d} x$ 不收敛。如果 $\displaystyle m< 0$,$\displaystyle \forall \varepsilon \in ( 0,m) ,$ $\displaystyle X\geqslant a$,都可以找到 $\displaystyle X< Y\ s.t.\ f( Y) < m+\varepsilon ,$进而 $\displaystyle \left| \int _{Y}^{Y+2\frac{\varepsilon }{-m-\varepsilon }} f( x)\mathrm{d} x\right| \geqslant 2\varepsilon $,进而 $\displaystyle \int _{a}^{+\infty } f( x)\mathrm{d} x$ 发散。

因此 $\displaystyle \lim _{x\rightarrow +\infty } f( x) =0$。

接着根据定积分第二中值定理,有(其中 $\displaystyle f( a+)$ 表示 $\displaystyle x\rightarrow a+$ 时 $\displaystyle f( x)$ 极限,由于收敛,该值一定存在):


\begin{align*}
	\int _{a}^{+\infty } f( x)\sin px\mathrm{d} x & =f( a+)\int _{a}^{\xi }\sin px\mathrm{d} x\\
	\left| \int _{a}^{+\infty } f( x)\sin px\mathrm{d} x\right|  & \leqslant \left| f( a+)\frac{2}{p}\right| 
\end{align*}

当 $\displaystyle p\rightarrow +\infty $,$\displaystyle \left| \int _{a}^{+\infty } f( x)\sin px\mathrm{d} x\right| \rightarrow 0$,进而$\displaystyle \lim _{p\rightarrow \infty }\int _{a}^{+\infty } f( x)\sin px\mathrm{d} x=0$。\qed 



\begin{ques}
	讨论 $\displaystyle \int _{1}^{+\infty }\frac{\sin^{2}\frac{x}{2}}{( x+\sin x)^{\alpha }}\mathrm{d} x,\alpha  >0$ 的敛散性。
\end{ques}



因为 \ $\displaystyle \frac{\sin^{2}\frac{x}{2}}{( x+\sin x)^{\alpha }} \geqslant 0,\lim _{x\rightarrow +\infty }\frac{\frac{\sin^{2}\frac{x}{2}}{( x+\sin x)^{\alpha }}}{\frac{\sin^{2}\frac{x}{2}}{( x)^{\alpha }}} =1$,因此 $\displaystyle \int _{1}^{+\infty }\frac{\sin^{2}\frac{x}{2}}{( x+\sin x)^{\alpha }}\mathrm{d} x$ 和$\displaystyle \int _{1}^{+\infty }\frac{\sin^{2}\frac{x}{2}}{x^{\alpha }}\mathrm{d} x$ 敛散性相同。

因此,当 $\displaystyle \alpha  >1$ 时 $ $$\displaystyle \int _{1}^{+\infty }\frac{\sin^{2}\frac{x}{2}}{( x+\sin x)^{\alpha }}\mathrm{d} x$ 收敛,当 $\displaystyle 0< \alpha \leqslant 1$ 时 $\displaystyle \int _{1}^{+\infty }\frac{\sin^{2}\frac{x}{2}}{( x+\sin x)^{\alpha }}\mathrm{d} x$ 发散。
\ifx\allfiles\undefined
\end{document}
\fi