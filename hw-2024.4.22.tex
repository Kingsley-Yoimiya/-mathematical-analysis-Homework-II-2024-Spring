%-*-    coding: UTF-8   -*-
% !TEX program = xelatex
\ifx\allfiles\undefined
%\special{dvipdfmx:config z 0}% 取消压缩,加快编译速度
\documentclass[UTF-8]{ctexart}
\usepackage{graphicx}
\usepackage{subfigure}
\usepackage{xcolor}
\usepackage{amsmath}
\usepackage{amssymb}
\usepackage{tabularx}
\usepackage{amssymb}
\usepackage{amsthm}
%\usepackage[usenames,dvipsnames]{color}
\usepackage{hyperref}
\hypersetup{
	colorlinks=true,
	linkcolor=black,
	filecolor=black,      
	urlcolor=red,
	citecolor=black,
}
\usepackage{geometry}
\geometry{a4paper,centering,scale=0.80}
\usepackage[format=hang,font=small,textfont=it]{caption}
\usepackage[nottoc]{tocbibind}
\usepackage{algorithm}  
\usepackage{algorithmicx}  
\usepackage{algpseudocode}
\usepackage{prettyref}
\usepackage{framed}
\setlength{\parindent}{2em}
\usepackage{indentfirst}
\usepackage[framemethod=TikZ]{mdframed}
\newcounter{ques}[section]
\renewcommand{\theques}{\arabic{section}.\arabic{ques}}
\newcommand{\setParDis}{\setlength {\parskip} {0.3cm} }
\newcommand{\setParDef}{\setlength {\parskip} {0pt} }
\setParDis% 调整这一个subsection的段落间距
%\setParDef%恢复间距

\newenvironment{ques}[1][]{
	\refstepcounter{ques}
	\mdfsetup{
		frametitle={
			\tikz[baseline=(current bounding box.east), outer sep=0pt]
			\node[anchor=east,rectangle,fill=blue!20]
			{\strut Problem~\theques\ifstrempty{#1}{}{:~#1}};},
		innertopmargin=10pt,linecolor=blue!20,
		linewidth=2pt,topline=true,
		frametitleaboveskip=\dimexpr-\ht\strutbox\relax
	}
	\begin{mdframed}[]\relax
}{\end{mdframed}}

\newcounter{Thm}[section]
\renewcommand{\theThm}{\arabic{section}.\arabic{Thm}}
\newenvironment{Thm}[1][]{
	\refstepcounter{Thm}
	\mdfsetup{
		frametitle={
			\tikz[baseline=(current bounding box.east), outer sep=0pt]
			\node[anchor=east,rectangle,fill=blue!20]
			{\strut Theorem~\theThm\ifstrempty{#1}{}{:~#1}};},
		innertopmargin=10pt,linecolor=blue!20,
		linewidth=2pt,topline=true,
		frametitleaboveskip=\dimexpr-\ht\strutbox\relax
	}
	\begin{mdframed}[]\relax
}{\end{mdframed}}

\newcounter{Defi}[section]
\renewcommand{\theDefi}{\arabic{section}.\arabic{Defi}}
\newenvironment{Defi}[1][]{
	\refstepcounter{Defi}
	\mdfsetup{
		frametitle={
			\tikz[baseline=(current bounding box.east), outer sep=0pt]
			\node[anchor=east,rectangle,fill=blue!20]
			{\strut Definition~\theDefi\ifstrempty{#1}{}{:~#1}};},
		innertopmargin=10pt,linecolor=blue!20,
		linewidth=2pt,topline=true,
		frametitleaboveskip=\dimexpr-\ht\strutbox\relax
	}
	\begin{mdframed}[]\relax
}{\end{mdframed}}

\newrefformat{qlt}{\underline{性质 \ref{#1}}}
\newcommand{\tpf}[2]{\begin{ques}[#1]{\kaishu #2}\end{ques}}
\newcommand{\pf}[1]{\begin{ques}{\kaishu #1}\end{ques}}
\newcommand{\tthm}[2]{\begin{Thm}[#1]{\kaishu #2}\end{Thm}}
\newcommand{\thm}[1]{\begin{Thm}{\kaishu #1}\end{Thm}}
\newcommand{\tdefi}[2]{\begin{Defi}[#1]{\kaishu #2}\end{Defi}}
\newcommand{\defi}[1]{\begin{Defi}{\kaishu #1}\end{Defi}}
\newcommand{\opf}[1]{{\kaishu{#1}}}
\title{数学分析 I 作业(2024. Spring)}
\author{\texttt{As-The-Wind}}

\date{2024 年 2 月 19 日 $\rightarrow$ \today}

\date{}
\author{尹锦润}
\begin{document}
\maketitle
\fi

\section{2024.4.22 作业}
\begin{ques}
	求函数列 $\displaystyle \{f_{n}( x)\}$ 在给定的区间上的极限函数,并讨论是否一致收敛:$\displaystyle f_{n}( x) =\sin\frac{x}{n^{n}} ,$(1)$\displaystyle x\in [ a,b]$,(2)$\displaystyle x\in ( -\infty ,+\infty )$。
\end{ques}



当 $\displaystyle n\rightarrow +\infty $ 时,有 $\displaystyle \forall x\in ( -\infty ,+\infty ) ,\frac{x}{n^{n}}\rightarrow 0$,进而 $\displaystyle f_{n}( x)\rightarrow 0$,于是极限函数 $\displaystyle f( x) \equiv 0$。

(1)当 $\displaystyle x\in [ a,b]$ 时候,$\displaystyle \left| \frac{x}{n^{n}}\right| \leqslant \max\left\{\left| \frac{a}{n^{n}}\right| ,\left| \frac{b}{n^{n}}\right| \right\}$,进而 $\displaystyle \forall \varepsilon  >0,\exists N >0\ s.t.\left| \frac{x}{n^{n}}\right| \leqslant \max\left\{\left| \frac{a}{n^{n}}\right| ,\left| \frac{b}{n^{n}}\right| \right\} < \varepsilon ,\forall n >N,x\in [ a,b] \ $,进而 $\displaystyle f_{n}( x) \rightrightarrows 0,x\in [ a,b]$。\qed 

(2)当 $\displaystyle x\in ( -\infty ,+\infty )$ 时候,$\displaystyle \forall n,\exists x=n^{n} \ s.t.\ f_{n}( x) =\sin 1$ 为常数,因此 $\displaystyle f_{n}( x) ,x\in ( -\infty ,+\infty )$ 不一致收敛。\qed 



\begin{ques}
	求函数列 $\displaystyle \{f_{n}( x)\}$ 在给定的区间上的极限函数,并讨论是否一致收敛:$\displaystyle f_{n}( x) =(\sin x)^{n^{\alpha }} ,( \alpha  >0) ,x\in [ 0,\pi ]$。
\end{ques}



当 $\displaystyle x=\frac{\pi }{2}$ 时 $\displaystyle f_{n}( x) \equiv 1$,当 $\displaystyle x\in \left[ 0,\frac{\pi }{2}\right) \cup \left(\frac{\pi }{2} ,\pi \right]$ 时,$\displaystyle 0\leqslant \sin x< 1$ 当 $\displaystyle n\rightarrow +\infty $ 时,$\displaystyle f_{n}( x)\rightarrow 0$,因此极限函数:
\begin{equation*}
	f( x) =\begin{cases}
		1 & x=\frac{\pi }{2}\\
		0 & x\in \left[ 0,\frac{\pi }{2}\right) \cup \left(\frac{\pi }{2} ,\pi \right]
	\end{cases}
\end{equation*}


考虑一致收敛性:

如果 $\displaystyle f_{n}( x) \rightrightarrows f( x)$,则 $\displaystyle \forall \varepsilon  >0,\exists N >0\ s.t.\ |f_{n}( x) -f( x) |< \varepsilon ,\forall n >N$,

对于 $\displaystyle x\in \left[ 0,\frac{\pi }{2}\right) \cup \left(\frac{\pi }{2} ,\pi \right] ,$此时 $\displaystyle |f_{n}( x) -0|< \varepsilon $,当 $\displaystyle x\rightarrow \frac{\pi }{2}$ 时,有 $\displaystyle |f_{n}\left(\frac{\pi }{2}\right) -0|\leqslant \varepsilon $,当 $\displaystyle \varepsilon < \frac{1}{4}$ 时,与 $\displaystyle |f_{n}\left(\frac{\pi }{2}\right) -1|< \varepsilon $ 矛盾。进而 $\displaystyle f_{n}( x)$ 不一致收敛。\qed 





\begin{ques}
	设 $\displaystyle f_{1}( x) \in R[ a,b]$,定义 $\displaystyle f_{n+1}( x) =\int _{a}^{x} f_{n}( t)\mathrm{d} t,n=1,2,3,\cdots $。证明:$\displaystyle f_{n}( x) \rightrightarrows 0,x\in [ a,b]$。
\end{ques}





因为 $\displaystyle f_{1}( x) \in R[ a,b]$,有 $\displaystyle \exists M >0,\ s.t.\ |f_{1}( x) |< M$,考虑归纳证明 $\displaystyle |f_{n}( x) |< \frac{M( x-a)^{n-1}}{( n-1) !}$:

对于 $\displaystyle n=1$ 成立。

对于 $\displaystyle n >1$,$\displaystyle |f_{n}( x) |=\left| \int _{a}^{x} f_{n-1}( t)\mathrm{d} t\right| < \left| \int _{a}^{x}\frac{M( t-a)^{n-2}}{( n-2) !}\mathrm{d} t\right| =\frac{M( x-a)^{n-1}}{( n-1) !}$。

因此有 $\displaystyle |f_{n}( x) |< \frac{M( x-a)^{n-1}}{( n-1) !}$,当 $\displaystyle n\rightarrow +\infty $ 时,有 $\displaystyle \frac{M( x-a)^{n-1}}{( n-1) !}\rightarrow 0$,由最值判别法,$\displaystyle f_{n}( x) \rightrightarrows 0,x\in [ a,b]$。\qed 









\begin{ques}
	讨论函数列在指定区间的一致收敛性:$\displaystyle f_{n}( x) =n^{\alpha } x( 1-x)^{n} ,\alpha \in \mathbb{R} ,x\in [ 0,1]$。
\end{ques}



当 $\displaystyle x=0$ 时,$\displaystyle f_{n}( x) \equiv 0$,当 $\displaystyle x\neq 0$ 时,有 $\displaystyle \lim _{n\rightarrow +\infty } f_{n}( x) =\lim _{n\rightarrow +\infty } n^{\alpha } x( 1-x)^{n} =0$,于是 $\displaystyle f_{n}( x)$ 的极限函数为 $\displaystyle f( x) \equiv 0,x\in [ 0,1]$。

接着考虑 $\displaystyle \sup _{x\in [ 0,1]}\{|f_{n}( x) -0|\}$,考虑
\begin{align*}
	\mathrm{\frac{d}{dx}}\left( n^{\alpha } x( 1-x)^{n}\right) & =n^{\alpha }\mathrm{\frac{d}{dx}}\left( x( 1-x)^{n}\right)\\
	& =n^{\alpha }\left(( 1-x)^{n}\mathrm{d} x-nx( 1-x)^{n-1}\mathrm{d} x\right)\\
	& =n^{\alpha }( 1-x)^{n-1}( 1-x-nx)\mathrm{d} x
\end{align*}


于是当 $\displaystyle x\leqslant \frac{1}{n+1}$ 时 $\displaystyle n^{\alpha } x( 1-x)^{n}$ 单调递增,当 $\displaystyle x\geqslant \frac{1}{n+1}$ 时单调递减,有
\begin{align*}
	\sup _{x\in [ 0,1]}\{|f_{n}( x) -0|\} & =f_{n}\left(\frac{1}{n+1}\right)\\
	& =n^{\alpha }\frac{1}{n+1}\frac{n^{n}}{( n+1)^{n}}\\
	& =\frac{n^{n+\alpha }}{( n+1)^{n+1}}
\end{align*}


当 $\displaystyle \alpha  >1$ 时,$\displaystyle \sup _{x\in [ 0,1]}\{|f_{n}( x) -0|\}\rightarrow +\infty \left( n\rightarrow +\infty \right)$,不一致收敛。

当 $\displaystyle \alpha =1$ 时,$\displaystyle \sup _{x\in [ 0,1]}\{|f_{n}( x) -0|\} =\left( 1-\frac{1}{n+1}\right)^{n+1}\rightarrow e^{-1}\left( n\rightarrow +\infty \right)$ 不一致收敛。

当 $\displaystyle \alpha < 1$ 时,$\displaystyle \sup _{x\in [ 0,1]}\{|f_{n}( x) -0|\}\rightarrow 0\left( n\rightarrow +\infty \right)$,一致收敛。

因此当 $\displaystyle \alpha \geqslant 1$ 时不一致收敛,当 $\displaystyle \alpha < 1$ 时一致收敛。\qed 







\begin{ques}
	设连续函数列 $\displaystyle \{f_{n}( x)\}$ 在区间 $\displaystyle [ 0,1]$ 上一致收敛,证明 $\displaystyle \left\{e^{f_{n}( x)}\right\}$ 在 $\displaystyle [ 0,1]$ 上也一致收敛。
\end{ques}



因为 $\displaystyle f_{n}( x)$ 在 $\displaystyle [ 0,1]$ 上一致收敛,因此对于收敛函数 $\displaystyle f( x)$ 有 $\displaystyle \exists M >0\ s.t.\ |f( x) |< M\ \forall x\in [ 0,1]$。

同时 $\displaystyle \forall \varepsilon  >0,\exists N >0,s.t.\ |f_{n}( x) -f( x) |< \frac{\varepsilon }{2M} ,\forall n >N$。

对于 $\displaystyle e^{f_{n}( x)}$ 其收敛函数就是 $\displaystyle e^{f( x)}$,而$\displaystyle \ |e^{f_{n}( x)} -e^{f( x)} |=e^{f( x)}\left( e^{f_{n}( x) -f( x)} -1\right)$ 。

有 $\displaystyle \forall \varepsilon  >0$,
\begin{gather*}
	\exists N_{1}  >0,\ s.t.\ |f_{n}( x) -f( x) |< \frac{\varepsilon }{2M} ,\forall n >N_{1} ;\\
	\exists N_{2}  >0,\ s.t.\ |\left( e^{f_{n}( x) -f( x)} -1\right) -( f_{n}( x) -f( x)) |< \frac{\varepsilon }{2M} ,\forall n > N_{2}\\
	\exists N=\max\{N_{1} ,N_{2}\} ,\ s.t.\ \ |e^{f_{n}( x)} -e^{f( x)} |=e^{f( x)}\left( e^{f_{n}( x) -f( x)} -1\right) < M\frac{\varepsilon }{M} =\varepsilon ,\forall n >N
\end{gather*}

因此 $\displaystyle \left\{e^{f_{n}( x)}\right\}$ 在 $\displaystyle [ 0,1]$ 上一致收敛。\qed 

\ifx\allfiles\undefined
\end{document}
\fi