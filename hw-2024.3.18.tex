%-*-    coding: UTF-8   -*-
% !TEX program = xelatex
\ifx\allfiles\undefined
%\special{dvipdfmx:config z 0}% 取消压缩,加快编译速度
\documentclass[UTF-8]{ctexart}
\usepackage{graphicx}
\usepackage{subfigure}
\usepackage{xcolor}
\usepackage{amsmath}
\usepackage{amssymb}
\usepackage{tabularx}
\usepackage{amssymb}
\usepackage{amsthm}
%\usepackage[usenames,dvipsnames]{color}
\usepackage{hyperref}
\hypersetup{
	colorlinks=true,
	linkcolor=black,
	filecolor=black,      
	urlcolor=red,
	citecolor=black,
}
\usepackage{geometry}
\geometry{a4paper,centering,scale=0.80}
\usepackage[format=hang,font=small,textfont=it]{caption}
\usepackage[nottoc]{tocbibind}
\usepackage{algorithm}  
\usepackage{algorithmicx}  
\usepackage{algpseudocode}
\usepackage{prettyref}
\usepackage{framed}
\setlength{\parindent}{2em}
\usepackage{indentfirst}
\usepackage[framemethod=TikZ]{mdframed}
\newcounter{ques}[section]
\renewcommand{\theques}{\arabic{section}.\arabic{ques}}
\newcommand{\setParDis}{\setlength {\parskip} {0.3cm} }
\newcommand{\setParDef}{\setlength {\parskip} {0pt} }
\setParDis% 调整这一个subsection的段落间距
%\setParDef%恢复间距

\newenvironment{ques}[1][]{
	\refstepcounter{ques}
	\mdfsetup{
		frametitle={
			\tikz[baseline=(current bounding box.east), outer sep=0pt]
			\node[anchor=east,rectangle,fill=blue!20]
			{\strut Problem~\theques\ifstrempty{#1}{}{:~#1}};},
		innertopmargin=10pt,linecolor=blue!20,
		linewidth=2pt,topline=true,
		frametitleaboveskip=\dimexpr-\ht\strutbox\relax
	}
	\begin{mdframed}[]\relax
}{\end{mdframed}}

\newcounter{Thm}[section]
\renewcommand{\theThm}{\arabic{section}.\arabic{Thm}}
\newenvironment{Thm}[1][]{
	\refstepcounter{Thm}
	\mdfsetup{
		frametitle={
			\tikz[baseline=(current bounding box.east), outer sep=0pt]
			\node[anchor=east,rectangle,fill=blue!20]
			{\strut Theorem~\theThm\ifstrempty{#1}{}{:~#1}};},
		innertopmargin=10pt,linecolor=blue!20,
		linewidth=2pt,topline=true,
		frametitleaboveskip=\dimexpr-\ht\strutbox\relax
	}
	\begin{mdframed}[]\relax
}{\end{mdframed}}

\newcounter{Defi}[section]
\renewcommand{\theDefi}{\arabic{section}.\arabic{Defi}}
\newenvironment{Defi}[1][]{
	\refstepcounter{Defi}
	\mdfsetup{
		frametitle={
			\tikz[baseline=(current bounding box.east), outer sep=0pt]
			\node[anchor=east,rectangle,fill=blue!20]
			{\strut Definition~\theDefi\ifstrempty{#1}{}{:~#1}};},
		innertopmargin=10pt,linecolor=blue!20,
		linewidth=2pt,topline=true,
		frametitleaboveskip=\dimexpr-\ht\strutbox\relax
	}
	\begin{mdframed}[]\relax
}{\end{mdframed}}

\newrefformat{qlt}{\underline{性质 \ref{#1}}}
\newcommand{\tpf}[2]{\begin{ques}[#1]{\kaishu #2}\end{ques}}
\newcommand{\pf}[1]{\begin{ques}{\kaishu #1}\end{ques}}
\newcommand{\tthm}[2]{\begin{Thm}[#1]{\kaishu #2}\end{Thm}}
\newcommand{\thm}[1]{\begin{Thm}{\kaishu #1}\end{Thm}}
\newcommand{\tdefi}[2]{\begin{Defi}[#1]{\kaishu #2}\end{Defi}}
\newcommand{\defi}[1]{\begin{Defi}{\kaishu #1}\end{Defi}}
\newcommand{\opf}[1]{{\kaishu{#1}}}
\title{数学分析 I 作业(2024. Spring)}
\author{\texttt{As-The-Wind}}

\date{2024 年 2 月 19 日 $\rightarrow$ \today}

\date{}
\author{尹锦润}
\begin{document}
\maketitle
\fi

\section{2024.3.18 作业}
\begin{ques}
	设 $\displaystyle f( x) =\lim _{n\rightarrow \infty }\int _{0}^{1}\frac{nt^{n-1}}{1+e^{xt}}\mathrm{d} t$,计算 $\displaystyle I=\int _{0}^{+\infty } f( x)\mathrm{d} x$。
\end{ques}




\begin{align*}
	\int _{0}^{1}\frac{nt^{n-1}}{1+e^{xt}}\mathrm{d} t & =\left(\frac{t^{n}}{1+e^{xt}}\middle| _{0}^{1}\right) -\int _{0}^{1}\frac{t^{n} e^{xt} x}{\left( 1+e^{xt}\right)^{2}}\mathrm{d} t\\
	& =\frac{1}{1+e^{x}} -\int _{0}^{1}\frac{t^{n} e^{xt} x}{\left( 1+e^{xt}\right)^{2}}\mathrm{d} t
\end{align*}


而考虑


\begin{align*}
	0\leqslant \int _{0}^{1}\frac{t^{n} e^{xt} x}{\left( 1+e^{xt}\right)^{2}}\mathrm{d} t & \leqslant \int _{0}^{1}\frac{t^{n} x}{1+e^{xt}}\mathrm{d} t\\
	& \leqslant x\int _{0}^{1} t^{n}\mathrm{d} t\\
	& =\frac{x}{n}
\end{align*}

当 $\displaystyle n\rightarrow +\infty $ 时,$\displaystyle 0\leqslant \int _{0}^{1}\frac{t^{n} e^{xt} x}{\left( 1+e^{xt}\right)^{2}}\mathrm{d} t\leqslant \frac{x}{n}\rightarrow 0$,因此,$\displaystyle f( x) =\frac{1}{1+e^{x}}$。


\begin{align*}
I & =\int _{0}^{+\infty } f( x)\mathrm{d} x\\
& =\int _{0}^{+\infty }\mathrm{\frac{1}{1+e^{x}} d} x\\
& =\int _{0}^{+\infty }\left( 1-\frac{e^{x}}{1+e^{x}}\right) \mathrm{d} x\\
& =\left(\ln\left(\frac{e^{x}}{1+e^{x}}\right)\middle| _{0}^{+\infty }\right)\\
& =\ln 2
\end{align*}




\begin{ques}
	讨论积分的敛散性:$\displaystyle I=\int _{0}^{+\infty }\frac{\sin x}{1+e^{-x}}\mathrm{d} x$。
\end{ques}




\begin{align*}
	I & =\int _{0}^{+\infty }\frac{\sin x}{1+e^{-x}}\mathrm{d} x\\
	& =\int _{0}^{+\infty }\sin x\mathrm{d} x+\int _{0}^{+\infty }\sin x\left(\frac{1}{1+e^{-x}} -1\right)\mathrm{d} x
\end{align*}

对于 $\displaystyle \int _{0}^{+\infty }\sin x\left(\frac{1}{1+e^{-x}} -1\right)\mathrm{d} x$,我们发现 $\displaystyle \left| \int _{0}^{+\infty }\sin x\mathrm{d} x\right| \leqslant 2$,同时 $\displaystyle \frac{1}{1+e^{-x}} -1$ 单调递增且 $\displaystyle \rightarrow 0\left( x\rightarrow +\infty \right)$。于是 $\displaystyle \int _{0}^{+\infty }\sin x\left(\frac{1}{1+e^{-x}} -1\right)\mathrm{d} x$ 收敛。

对于 $\displaystyle \int _{0}^{+\infty }\sin x\mathrm{d} x$,显然是发散的,进而 $\displaystyle I$ 是发散的。\qed 



\begin{ques}
	讨论积分 $\displaystyle I=\int _{0}^{+\infty }\ln\left(\cos\frac{1}{x} +\sin^{p}\frac{1}{x}\right)\mathrm{d} x( p >1)$ 的敛散性。
\end{ques}



当 $\displaystyle x\rightarrow +\infty $ 是,$\displaystyle \cos\frac{1}{x} +\sin^{p}\frac{1}{x}\rightarrow 1$ 同时 $\displaystyle \ln\left(\cos\frac{1}{x} +\sin^{p}\frac{1}{x}\right)$ 定号,运用极限形式的比较判别法,

$\displaystyle I$ 和 $\displaystyle \int _{0}^{+\infty }\left(\cos\frac{1}{x} +\sin^{p}\frac{1}{x} -1\right)\mathrm{d} x$ 的敛散性相同。

进一步运用极限形式的比较判别法,可以得到 $\displaystyle I$ 和 $\displaystyle \int _{0}^{+\infty }\left( -\frac{1}{2x^{2}} +\frac{1}{x^{p}}\right)\mathrm{d} x$ 的敛散性相同。

而 $\displaystyle p >1$ 因此 $\displaystyle I$ 收敛。\qed 


\ifx\allfiles\undefined
\end{document}
\fi