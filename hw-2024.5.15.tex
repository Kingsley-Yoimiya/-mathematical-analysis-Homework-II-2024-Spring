%-*-    coding: UTF-8   -*-
% !TEX program = xelatex
\ifx\allfiles\undefined
%\special{dvipdfmx:config z 0}% 取消压缩,加快编译速度
\documentclass[UTF-8]{ctexart}
\usepackage{graphicx}
\usepackage{subfigure}
\usepackage{xcolor}
\usepackage{amsmath}
\usepackage{amssymb}
\usepackage{tabularx}
\usepackage{amssymb}
\usepackage{amsthm}
%\usepackage[usenames,dvipsnames]{color}
\usepackage{hyperref}
\hypersetup{
	colorlinks=true,
	linkcolor=black,
	filecolor=black,      
	urlcolor=red,
	citecolor=black,
}
\usepackage{geometry}
\geometry{a4paper,centering,scale=0.80}
\usepackage[format=hang,font=small,textfont=it]{caption}
\usepackage[nottoc]{tocbibind}
\usepackage{algorithm}  
\usepackage{algorithmicx}  
\usepackage{algpseudocode}
\usepackage{prettyref}
\usepackage{framed}
\setlength{\parindent}{2em}
\usepackage{indentfirst}
\usepackage[framemethod=TikZ]{mdframed}
\newcounter{ques}[section]
\renewcommand{\theques}{\arabic{section}.\arabic{ques}}
\newcommand{\setParDis}{\setlength {\parskip} {0.3cm} }
\newcommand{\setParDef}{\setlength {\parskip} {0pt} }
\setParDis% 调整这一个subsection的段落间距
%\setParDef%恢复间距

\newenvironment{ques}[1][]{
	\refstepcounter{ques}
	\mdfsetup{
		frametitle={
			\tikz[baseline=(current bounding box.east), outer sep=0pt]
			\node[anchor=east,rectangle,fill=blue!20]
			{\strut Problem~\theques\ifstrempty{#1}{}{:~#1}};},
		innertopmargin=10pt,linecolor=blue!20,
		linewidth=2pt,topline=true,
		frametitleaboveskip=\dimexpr-\ht\strutbox\relax
	}
	\begin{mdframed}[]\relax
}{\end{mdframed}}

\newcounter{Thm}[section]
\renewcommand{\theThm}{\arabic{section}.\arabic{Thm}}
\newenvironment{Thm}[1][]{
	\refstepcounter{Thm}
	\mdfsetup{
		frametitle={
			\tikz[baseline=(current bounding box.east), outer sep=0pt]
			\node[anchor=east,rectangle,fill=blue!20]
			{\strut Theorem~\theThm\ifstrempty{#1}{}{:~#1}};},
		innertopmargin=10pt,linecolor=blue!20,
		linewidth=2pt,topline=true,
		frametitleaboveskip=\dimexpr-\ht\strutbox\relax
	}
	\begin{mdframed}[]\relax
}{\end{mdframed}}

\newcounter{Defi}[section]
\renewcommand{\theDefi}{\arabic{section}.\arabic{Defi}}
\newenvironment{Defi}[1][]{
	\refstepcounter{Defi}
	\mdfsetup{
		frametitle={
			\tikz[baseline=(current bounding box.east), outer sep=0pt]
			\node[anchor=east,rectangle,fill=blue!20]
			{\strut Definition~\theDefi\ifstrempty{#1}{}{:~#1}};},
		innertopmargin=10pt,linecolor=blue!20,
		linewidth=2pt,topline=true,
		frametitleaboveskip=\dimexpr-\ht\strutbox\relax
	}
	\begin{mdframed}[]\relax
}{\end{mdframed}}

\newrefformat{qlt}{\underline{性质 \ref{#1}}}
\newcommand{\tpf}[2]{\begin{ques}[#1]{\kaishu #2}\end{ques}}
\newcommand{\pf}[1]{\begin{ques}{\kaishu #1}\end{ques}}
\newcommand{\tthm}[2]{\begin{Thm}[#1]{\kaishu #2}\end{Thm}}
\newcommand{\thm}[1]{\begin{Thm}{\kaishu #1}\end{Thm}}
\newcommand{\tdefi}[2]{\begin{Defi}[#1]{\kaishu #2}\end{Defi}}
\newcommand{\defi}[1]{\begin{Defi}{\kaishu #1}\end{Defi}}
\newcommand{\opf}[1]{{\kaishu{#1}}}
\title{数学分析 I 作业(2024. Spring)}
\author{\texttt{As-The-Wind}}

\date{2024 年 2 月 19 日 $\rightarrow$ \today}

\date{}
\author{尹锦润}
\begin{document}
\maketitle
\fi

\section{2024.5.15 作业}

\begin{ques}
	求幂级数的收敛半径和收敛域:
\begin{equation*}
	\sum _{n=1}^{\infty }\left( 1+\frac{1}{n}\right)^{n^{2}} (x-2)^{n} .
\end{equation*}
\end{ques}

\begin{gather*}
	a_{n} =\left( 1+\frac{1}{n}\right)^{n^{2}}\\
	\varlimsup _{n\rightarrow +\infty }\sqrt[n]{|a_{n} |} =\varlimsup _{n\rightarrow +\infty }\left( 1+\frac{1}{n}\right)^{n} =e
\end{gather*}

因此收敛半径 $\displaystyle R=\frac{1}{e}$,当 $\displaystyle x=2\pm \frac{1}{e}$ 时,$\displaystyle \lim{}_{n\rightarrow +\infty }\left( 1+\frac{1}{n}\right)^{n^{2}}( x-2)^{n} =\pm 1$,幂级数不收敛,因此收敛域为 $\displaystyle \left( 2-\frac{1}{e} ,2+\frac{1}{e}\right)$。\qed 







\begin{ques}
	求幂级数的收敛半径和收敛域:
\begin{equation*}
	\sum _{n=1}^{\infty }\frac{(2n)!!}{(2n+1)!!} x^{n} .
\end{equation*}
\end{ques}

\begin{gather*}
	a_{n} =\frac{(2n)!!}{(2n+1)!!}\\
	\lim _{n\rightarrow +\infty }\frac{|a_{n+1} |}{|a_{n} |} =\lim _{n\rightarrow +\infty }\frac{2n+2}{2n+3} =1
\end{gather*}

因此收敛半径是 $\displaystyle 1$,当 $\displaystyle x=-1$ 时,为交错级数,且 $\displaystyle |a_{n} |$ 单调下降,故而收敛,当 $\displaystyle x=1$ 时,由拉贝判别法 $\displaystyle \lim _{n\rightarrow +\infty } n\left(\frac{a_{n}}{a_{n+1}} -1\right) =\frac{1}{2} < 1$,进而发散,于是收敛域为 $\displaystyle [ -1,1)$。\qed 



\begin{ques}
已知幂级数 $\sum _{n=0}^{\infty } a_{n} x^{n}$ 的收敛半径是 $r$($0< r< \infty $),给出下面幂级数的收敛半径,其中 $k$ 为正整数。

1. $\displaystyle \sum _{n=0}^{\infty } a_{n}^{k} x^{n}$ 2. $\displaystyle \sum _{n=0}^{\infty } a_{n^{2}} x^{n}$ 3. $\displaystyle \sum _{n=0}^{\infty } a_{n} x^{kn}$ 4. $\displaystyle \sum _{n=0}^{\infty } a_{n} x^{n^{2}}$
\end{ques}
\begin{equation*}
	\varlimsup _{n\rightarrow +\infty }\sqrt[n]{|a_{n} |} =\frac{1}{r}
\end{equation*}


1. 
\begin{equation*}
	\varlimsup _{n\rightarrow +\infty }\sqrt[n]{|a_{n}^{k} |} =\left(\frac{1}{r}\right)^{k}
\end{equation*}
于是收敛半径为 $\displaystyle r^{k}$。

2.
\begin{equation*}
	\varlimsup _{n\rightarrow +\infty }\sqrt[n]{|a_{n^{2}} |} =\begin{cases}
		+\infty  & r< 1\\
		0 & r >1\\
		1 & r=1
	\end{cases}
\end{equation*}
于是收敛半径为 $\displaystyle \begin{cases}
	0 & r< 1\\
	+\infty  & r >1\\
	1 & r=1
\end{cases}$。

3.
\begin{equation*}
	\varlimsup _{n\rightarrow +\infty }\sqrt[kn]{|a_{n} |} =\sqrt[k]{\frac{1}{r}}
\end{equation*}
于是收敛半径为 $\displaystyle \sqrt[k]{r}$。

4.
\begin{equation*}
	\varlimsup _{n\rightarrow +\infty }\sqrt[n^{2}]{|a_{n} |} =1
\end{equation*}
于是收敛半径为 $\displaystyle 1$。

\qed 



\begin{ques}
设 $\sum _{n=0}^{\infty } a_{n}$ 发散,且幂级数 $\sum _{n=0}^{\infty } a_{n} x^{n}$ 收敛半径为 1。是否一定有 $\lim _{x\rightarrow 1^{-}}\sum _{n=0}^{\infty } a_{n} x^{n} =\infty $。若是,证明之;若否,给出例子。
\end{ques}



否,取 $\displaystyle a_{n} =( -1)^{n}$,则 $\lim _{x\rightarrow 1^{-}}\sum _{n=0}^{\infty } a_{n} x^{n} =\lim _{x\rightarrow 1^{-}}\lim _{n\rightarrow +\infty }\frac{1-( -x)^{n}}{1+x} =\frac{1}{2}$。\qed 



\begin{ques}
设 $\sum _{n=0}^{\infty } a_{n}$ 发散,且幂级数 $\sum _{n=0}^{\infty } a_{n} x^{n}$ 收敛半径为 1。是否一定有 $\lim _{x\rightarrow 1^{-}}\sum _{n=0}^{\infty } a_{n} x^{n} \nexists $。若是,证明之;若否,给出例子。
\end{ques}



否,取 $\displaystyle a_{n} =( -1)^{n}$,则 $\lim _{x\rightarrow 1^{-}}\sum _{n=0}^{\infty } a_{n} x^{n} =\lim _{x\rightarrow 1^{-}}\lim _{n\rightarrow +\infty }\frac{1-( -x)^{n}}{1+x} =\frac{1}{2}$。\qed 



\begin{ques}
设 $\sum _{n=0}^{\infty } a_{n} x^{n}$ 的收敛半径是 $R=1$,和函数是 $S(x)=\sum _{n=0}^{\infty } a_{n} x^{n}$,$\sum _{n=0}^{\infty } a_{n}$ 发散。问 $\lim _{x\rightarrow 1^{-}} S(x)$ 是否可能存在。如可能存在,给出例子;若不可能存在,说明原因。
\end{ques}



可能存在,取 $\displaystyle a_{n} =( -1)^{n}$,则 $\lim _{x\rightarrow 1^{-}}\sum _{n=0}^{\infty } a_{n} x^{n} =\lim _{x\rightarrow 1^{-}}\lim _{n\rightarrow +\infty }\frac{1-( -x)^{n}}{1+x} =\frac{1}{2}$。\qed 





\begin{ques}
利用幂级数求级数的和(闭式):
\begin{equation*}
	\sum _{n=0}^{\infty }\frac{(-1)^{n}}{3n+1} .
\end{equation*}
\end{ques}


令



\begin{equation*}
	f( x) =\sum _{n=0}^{\infty }\frac{(-1)^{n} (x)^{3n+1}}{3n+1}
\end{equation*}则答案就是 $\displaystyle f( 1)$。




\begin{align*}
	f'( x) & =\sum _{n=0}^{\infty } (-1)^{n} (x)^{3n}\\
	& =\frac{1}{1+x^{3}}\\
	& =\frac{1}{3}\frac{1}{x+1} -\frac{1}{6}\frac{2x-1}{x^{2} -x+1} +\frac{1}{2}\frac{1}{x^{2} -x+1}\\
	f( x) =\int _{0}^{1} f'( x)\mathrm{d} x & =\int _{0}^{1}\left(\frac{1}{3}\frac{1}{x+1} -\frac{1}{6}\frac{2x-1}{x^{2} -x+1} +\frac{1}{2}\frac{1}{x^{2} -x+1}\right)\mathrm{d} x\\
	& =\left. \left(\frac{1}{3}\ln( x+1) -\frac{1}{6}\ln\left( x^{2} -x+1\right) +\frac{1}{\sqrt{3}}\arctan\left(\frac{2x}{\sqrt{3}} -\frac{1}{\sqrt{3}}\right)\right)\right| _{0}^{1}\\
	& =\frac{\ln 2}{3} +\frac{\pi }{3\sqrt{3}}
\end{align*}
\qed 







\begin{ques}
利用幂级数求级数的和(闭式):
\begin{equation*}
	\sum _{n=0}^{\infty }\left(\frac{1}{4n+1} +\frac{1}{4n+3} -\frac{1}{2n+2}\right) .
\end{equation*}
\end{ques}


记


\begin{equation*}
	f( x) =\sum _{n=0}^{\infty }\left(\frac{x^{4n+1}}{4n+1} +\frac{x^{4n+3}}{4n+3} -\frac{x^{2n+2}}{2n+2}\right)
\end{equation*}


则答案就是 $\displaystyle f( 1)$。




\begin{align*}
	f'( x) & =\sum _{n=0}^{\infty } x^{4n} +x^{4n+2} -x^{2n+1}\\
	& =\frac{1}{1-x^{4}} +\frac{x^{2}}{1-x^{4}} -\frac{x}{1-x^{2}}\\
	& =\frac{1}{1+x}\\
	f( 1) =\int _{0}^{1} f'( x)\mathrm{d} x & =\int _{0}^{1}\frac{1}{1+x}\mathrm{d} x\\
	& =\ln( 1+x) | _{0}^{1}\\
	& =\ln 2
\end{align*}
\qed 



\ifx\allfiles\undefined
\end{document}
\fi