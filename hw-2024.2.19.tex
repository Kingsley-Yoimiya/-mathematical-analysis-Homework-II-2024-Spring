%-*-    coding: UTF-8   -*-
% !TEX program = xelatex
\ifx\allfiles\undefined
%\special{dvipdfmx:config z 0}% 取消压缩,加快编译速度
\documentclass[UTF-8]{ctexart}
\usepackage{graphicx}
\usepackage{subfigure}
\usepackage{xcolor}
\usepackage{amsmath}
\usepackage{amssymb}
\usepackage{tabularx}
\usepackage{amssymb}
\usepackage{amsthm}
%\usepackage[usenames,dvipsnames]{color}
\usepackage{hyperref}
\hypersetup{
	colorlinks=true,
	linkcolor=black,
	filecolor=black,      
	urlcolor=red,
	citecolor=black,
}
\usepackage{geometry}
\geometry{a4paper,centering,scale=0.80}
\usepackage[format=hang,font=small,textfont=it]{caption}
\usepackage[nottoc]{tocbibind}
\usepackage{algorithm}  
\usepackage{algorithmicx}  
\usepackage{algpseudocode}
\usepackage{prettyref}
\usepackage{framed}
\setlength{\parindent}{2em}
\usepackage{indentfirst}
\usepackage[framemethod=TikZ]{mdframed}
\newcounter{ques}[section]
\renewcommand{\theques}{\arabic{section}.\arabic{ques}}
\newcommand{\setParDis}{\setlength {\parskip} {0.3cm} }
\newcommand{\setParDef}{\setlength {\parskip} {0pt} }
\setParDis% 调整这一个subsection的段落间距
%\setParDef%恢复间距

\newenvironment{ques}[1][]{
	\refstepcounter{ques}
	\mdfsetup{
		frametitle={
			\tikz[baseline=(current bounding box.east), outer sep=0pt]
			\node[anchor=east,rectangle,fill=blue!20]
			{\strut Problem~\theques\ifstrempty{#1}{}{:~#1}};},
		innertopmargin=10pt,linecolor=blue!20,
		linewidth=2pt,topline=true,
		frametitleaboveskip=\dimexpr-\ht\strutbox\relax
	}
	\begin{mdframed}[]\relax
}{\end{mdframed}}

\newcounter{Thm}[section]
\renewcommand{\theThm}{\arabic{section}.\arabic{Thm}}
\newenvironment{Thm}[1][]{
	\refstepcounter{Thm}
	\mdfsetup{
		frametitle={
			\tikz[baseline=(current bounding box.east), outer sep=0pt]
			\node[anchor=east,rectangle,fill=blue!20]
			{\strut Theorem~\theThm\ifstrempty{#1}{}{:~#1}};},
		innertopmargin=10pt,linecolor=blue!20,
		linewidth=2pt,topline=true,
		frametitleaboveskip=\dimexpr-\ht\strutbox\relax
	}
	\begin{mdframed}[]\relax
}{\end{mdframed}}

\newcounter{Defi}[section]
\renewcommand{\theDefi}{\arabic{section}.\arabic{Defi}}
\newenvironment{Defi}[1][]{
	\refstepcounter{Defi}
	\mdfsetup{
		frametitle={
			\tikz[baseline=(current bounding box.east), outer sep=0pt]
			\node[anchor=east,rectangle,fill=blue!20]
			{\strut Definition~\theDefi\ifstrempty{#1}{}{:~#1}};},
		innertopmargin=10pt,linecolor=blue!20,
		linewidth=2pt,topline=true,
		frametitleaboveskip=\dimexpr-\ht\strutbox\relax
	}
	\begin{mdframed}[]\relax
}{\end{mdframed}}

\newrefformat{qlt}{\underline{性质 \ref{#1}}}
\newcommand{\tpf}[2]{\begin{ques}[#1]{\kaishu #2}\end{ques}}
\newcommand{\pf}[1]{\begin{ques}{\kaishu #1}\end{ques}}
\newcommand{\tthm}[2]{\begin{Thm}[#1]{\kaishu #2}\end{Thm}}
\newcommand{\thm}[1]{\begin{Thm}{\kaishu #1}\end{Thm}}
\newcommand{\tdefi}[2]{\begin{Defi}[#1]{\kaishu #2}\end{Defi}}
\newcommand{\defi}[1]{\begin{Defi}{\kaishu #1}\end{Defi}}
\newcommand{\opf}[1]{{\kaishu{#1}}}
\title{数学分析 I 作业(2024. Spring)}
\author{\texttt{As-The-Wind}}

\date{2024 年 2 月 19 日 $\rightarrow$ \today}

\date{}
\author{尹锦润}
\begin{document}
\maketitle
\fi

\section{2024.2.19 作业}

\begin{ques}
使用积分计算: $\displaystyle \lim _{n\rightarrow \infty }\frac{\sqrt[n]{n!}}{n}$。

\end{ques}



\begin{equation*}
\begin{aligned}
\lim _{n\rightarrow \infty }\frac{\sqrt[n]{n!}}{n} = & \lim _{n\rightarrow \infty }\frac{e^{\frac{1}{n}(\ln 1+\cdots +\ln n)}}{n}\\
= & \lim _{n\rightarrow \infty }\frac{e^{\frac{1}{n}\left(\ln\frac{1}{n} +\cdots +\ln 1\right) +\ln n}}{n}\\
= & \lim _{n\rightarrow \infty }\frac{e^{\left(\int _{0}^{1}\ln x\mathrm{d} x\right) +\ln n}}{n}\\
= & \lim _{n\rightarrow \infty }\frac{e^{-1+\ln n}}{n}\\
= & e^{-1}
\end{aligned}
\end{equation*}


\begin{ques}
设序列 $\{a_{n} \}$ 满足 $\lim _{n\rightarrow \infty }\frac{a_{n}}{n^{\alpha }} =1$ $(\alpha  >0)$,使用积分计算 $\displaystyle \lim _{n\rightarrow \infty }\frac{1}{n^{1+\alpha }}( a_{1} +\cdots +a_{n})$。

\end{ques}


因为 $\lim _{n\rightarrow \infty }\frac{a_{n}}{n^{\alpha }} =1$,于是当 $\displaystyle n\rightarrow \infty $ 时,$\displaystyle a_{n} \sim n^{\alpha }$。

进而


\begin{equation*}
\begin{aligned}
\lim _{n\rightarrow \infty }\frac{1}{n^{1+\alpha }}( a_{1} +\cdots +a_{n}) = & \lim _{n\rightarrow \infty }\frac{1}{n}\frac{1}{n^{\alpha }}\sum _{i=1}^{n} i^{\alpha }\\
= & \lim _{n\rightarrow \infty }\frac{1}{n}\sum _{i=1}^{n}\left(\frac{i}{n}\right)^{\alpha }\\
= & \int _{0}^{1} x^{\alpha }\mathrm{d} x\\
= & \frac{1}{\alpha +1} x^{\alpha +1} |_{0}^{1}\\
= & \frac{1}{\alpha +1}
\end{aligned}
\end{equation*}




\begin{ques}
设 $\displaystyle f( x) \in R[ 0,1] ,\int _{0}^{1} f( x)\mathrm{d} x >0$,证明:$\displaystyle \exists ( \alpha ,\beta ) \subset [ 0,1] ,\ s.t.\ f( x)  >0,x\in ( \alpha ,\beta )$。

\end{ques}


考虑反证法,如果 $\displaystyle \forall ( \alpha ,\beta ) \subset [ 0,1] \ s.t.\ \exists \xi \in ( \alpha ,\beta ) \ s.t.\ f( \xi ) \leqslant 0$,那么

考虑任一 $\displaystyle [ 0,1]$ 分割 $\displaystyle \Delta :0=x_{0} < x_{1} < \dotsc < x_{n} =1$,对于根据假设可以选出一组 $\displaystyle \{\xi _{i}\} ,\xi _{i} \in ( x_{i-1} ,x_{i})$,满足 $\displaystyle \forall i,f( \xi _{i}) \leqslant 0$,又 $\displaystyle f( x) \in R[ 0,1]$,有


\begin{equation*}
\int _{0}^{1} f( x)\mathrm{d} x=\lim _{n\rightarrow \infty }\sum f( \xi _{i}) \Delta x_{i} \leqslant 0
\end{equation*}


与原来条件矛盾。\qed 



\begin{ques}
设 $\displaystyle f( x)$ 为 $\displaystyle \mathbb{R}$ 上的连续凸函数,$\displaystyle g( x)$ 为 $\displaystyle \mathbb{R}$ 上连续函数,证明:当 $\displaystyle a >0$ 时,$\displaystyle f\left(\frac{1}{a}\int _{0}^{a} g( t)\mathrm{d} t\right) \leqslant \frac{1}{a}\int _{0}^{a} f( g( t))\mathrm{d} t$。

\end{ques}


取任一分割 $\displaystyle \Delta $,有




\begin{equation*}
\begin{aligned}
f\left(\frac{1}{a}\int _{0}^{a} g( t)\mathrm{d} t\right) & =\lim _{n\rightarrow \infty } f\left(\frac{1}{a}\sum g( \xi _{i}) \Delta x_{i}\right)\\
 & =\lim _{n\rightarrow \infty } f\left(\sum g( \xi _{i})\frac{\Delta x_{i}}{a}\right)\\
 & \leqslant \lim _{n\rightarrow \infty }\sum \frac{\Delta x_{i}}{a} f( g( \xi _{i}))\\
 & =\frac{1}{a}\int _{0}^{a} f( g( t))\mathrm{d} t\qed 
\end{aligned}
\end{equation*}





\ifx\allfiles\undefined
\end{document}
\fi