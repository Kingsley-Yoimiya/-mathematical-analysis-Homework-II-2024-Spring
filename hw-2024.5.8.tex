%-*-    coding: UTF-8   -*-
% !TEX program = xelatex
\ifx\allfiles\undefined
%\special{dvipdfmx:config z 0}% 取消压缩,加快编译速度
\documentclass[UTF-8]{ctexart}
\usepackage{graphicx}
\usepackage{subfigure}
\usepackage{xcolor}
\usepackage{amsmath}
\usepackage{amssymb}
\usepackage{tabularx}
\usepackage{amssymb}
\usepackage{amsthm}
%\usepackage[usenames,dvipsnames]{color}
\usepackage{hyperref}
\hypersetup{
	colorlinks=true,
	linkcolor=black,
	filecolor=black,      
	urlcolor=red,
	citecolor=black,
}
\usepackage{geometry}
\geometry{a4paper,centering,scale=0.80}
\usepackage[format=hang,font=small,textfont=it]{caption}
\usepackage[nottoc]{tocbibind}
\usepackage{algorithm}  
\usepackage{algorithmicx}  
\usepackage{algpseudocode}
\usepackage{prettyref}
\usepackage{framed}
\setlength{\parindent}{2em}
\usepackage{indentfirst}
\usepackage[framemethod=TikZ]{mdframed}
\newcounter{ques}[section]
\renewcommand{\theques}{\arabic{section}.\arabic{ques}}
\newcommand{\setParDis}{\setlength {\parskip} {0.3cm} }
\newcommand{\setParDef}{\setlength {\parskip} {0pt} }
\setParDis% 调整这一个subsection的段落间距
%\setParDef%恢复间距

\newenvironment{ques}[1][]{
	\refstepcounter{ques}
	\mdfsetup{
		frametitle={
			\tikz[baseline=(current bounding box.east), outer sep=0pt]
			\node[anchor=east,rectangle,fill=blue!20]
			{\strut Problem~\theques\ifstrempty{#1}{}{:~#1}};},
		innertopmargin=10pt,linecolor=blue!20,
		linewidth=2pt,topline=true,
		frametitleaboveskip=\dimexpr-\ht\strutbox\relax
	}
	\begin{mdframed}[]\relax
}{\end{mdframed}}

\newcounter{Thm}[section]
\renewcommand{\theThm}{\arabic{section}.\arabic{Thm}}
\newenvironment{Thm}[1][]{
	\refstepcounter{Thm}
	\mdfsetup{
		frametitle={
			\tikz[baseline=(current bounding box.east), outer sep=0pt]
			\node[anchor=east,rectangle,fill=blue!20]
			{\strut Theorem~\theThm\ifstrempty{#1}{}{:~#1}};},
		innertopmargin=10pt,linecolor=blue!20,
		linewidth=2pt,topline=true,
		frametitleaboveskip=\dimexpr-\ht\strutbox\relax
	}
	\begin{mdframed}[]\relax
}{\end{mdframed}}

\newcounter{Defi}[section]
\renewcommand{\theDefi}{\arabic{section}.\arabic{Defi}}
\newenvironment{Defi}[1][]{
	\refstepcounter{Defi}
	\mdfsetup{
		frametitle={
			\tikz[baseline=(current bounding box.east), outer sep=0pt]
			\node[anchor=east,rectangle,fill=blue!20]
			{\strut Definition~\theDefi\ifstrempty{#1}{}{:~#1}};},
		innertopmargin=10pt,linecolor=blue!20,
		linewidth=2pt,topline=true,
		frametitleaboveskip=\dimexpr-\ht\strutbox\relax
	}
	\begin{mdframed}[]\relax
}{\end{mdframed}}

\newrefformat{qlt}{\underline{性质 \ref{#1}}}
\newcommand{\tpf}[2]{\begin{ques}[#1]{\kaishu #2}\end{ques}}
\newcommand{\pf}[1]{\begin{ques}{\kaishu #1}\end{ques}}
\newcommand{\tthm}[2]{\begin{Thm}[#1]{\kaishu #2}\end{Thm}}
\newcommand{\thm}[1]{\begin{Thm}{\kaishu #1}\end{Thm}}
\newcommand{\tdefi}[2]{\begin{Defi}[#1]{\kaishu #2}\end{Defi}}
\newcommand{\defi}[1]{\begin{Defi}{\kaishu #1}\end{Defi}}
\newcommand{\opf}[1]{{\kaishu{#1}}}
\title{数学分析 I 作业(2024. Spring)}
\author{\texttt{As-The-Wind}}

\date{2024 年 2 月 19 日 $\rightarrow$ \today}

\date{}
\author{尹锦润}
\begin{document}
\maketitle
\fi

\section{2024.5.8 作业}
\begin{ques}
	证明 $\displaystyle f( x) =\sum _{n=1}^{+\infty } n^{\alpha } e^{-nx} \in C^{\infty }( 0,+\infty )$,这里 $\displaystyle \alpha $ 是一实数。
\end{ques}




\begin{align*}
	f^{( 0)}( x) & =\sum _{n=1}^{+\infty } n^{\alpha } e^{-nx}\\
	f^{( 1)}( x) & =\sum _{n=1}^{+\infty }( -1) n^{\alpha +1} e^{-nx}\\
	& \vdots \\
	f^{( t)}( x) & =\sum _{n=1}^{+\infty }( -1)^{t} n^{\alpha +t} e^{-nx}
\end{align*}


因为 $\displaystyle ( -1)^{t} n^{\alpha +t} e^{-nx}$ 是连续函数,只需证明 $\displaystyle \sum _{n=1}^{+\infty } n^{\alpha +t} e^{-nx}$ 内闭一致收敛即可。

对于 $\displaystyle x\in [ a,b] \subset ( 0,+\infty ) ,\forall a,b$,有 $\displaystyle \left( n^{\alpha +t} e^{-nx}\right) '=-n^{\alpha +t+1} e^{-nx} < 0$ 于是 $\displaystyle n^{\alpha +t} e^{-nx}$ 在 $\displaystyle x=a$ 取得最大值,进而 $\displaystyle |n^{\alpha +t} e^{-nx} |\leqslant n^{\alpha +t} e^{-na}$,对于 $\displaystyle \left| \sum _{n=1}^{+\infty } n^{\alpha +t} e^{-nx}\right| \leqslant \sum _{n=1}^{+\infty } n^{\alpha +t} e^{-na}$,而


\begin{equation*}
	\lim _{n\rightarrow +\infty }\frac{n^{\alpha +t} e^{-na}}{( n+1)^{\alpha +t} e^{-( n+1) a}} =\lim _{n\rightarrow +\infty }\frac{n^{\alpha +t} e^{a}}{( n+1)^{\alpha +t}} =e^{a}  >1
\end{equation*}


因此,根据 Rabbe 判别法,$\displaystyle \sum _{n=1}^{+\infty } n^{\alpha +t} e^{-na}$ 收敛,进而$\displaystyle \left| \sum _{n=1}^{+\infty } n^{\alpha +t} e^{-nx}\right| $在 $\displaystyle x\in [ a,b]$ 上一致收敛。

于是 $\displaystyle \sum _{n=1}^{+\infty } n^{\alpha +t} e^{-nx}$ 内闭一致收敛,又因为 $\displaystyle ( -1)^{t} n^{\alpha +t} e^{-nx}$ 是连续函数,$\displaystyle f^{( t)}( x) \in C( 0,+\infty ) ,\forall t$,于是 $\displaystyle f( x) =\sum _{n=1}^{+\infty } n^{\alpha } e^{-nx} \in C^{\infty }( 0,+\infty )$。\qed 







\begin{ques}
	设函数 $\displaystyle f_{n}( x)( n=1,2,3,\cdots )$ 在任一有限区间上可积,并且在 $\displaystyle ( -\infty ,+\infty )$ 上内闭一致收敛到 $\displaystyle f( x)$。再设存在函数 $\displaystyle g( x)$ 满足 $\displaystyle |f_{n}( x) |\leqslant g( x)( n=1,2,3,\cdots ) ,\forall x\in ( -\infty ,+\infty )$,且 $\displaystyle \int _{-\infty }^{+\infty } g( x)\mathrm{d} x< +\infty $。证明广义积分 $\displaystyle \int _{-\infty }^{+\infty } f( x)\mathrm{d} x$ 收敛,并且 $\displaystyle \lim _{n\rightarrow \infty }\int _{-\infty }^{+\infty } f_{n}( x)\mathrm{d} x=\int _{-\infty }^{+\infty } f( x)\mathrm{d} x$。
\end{ques}





因为 $\displaystyle |f_{n}( x) |\leqslant g( x) ,\forall n$,于是有


\begin{gather*}
|f( x) |=\lim _{n\rightarrow +\infty } |f_{n}( x) |\leqslant g( x)\\
\int _{-\infty }^{+\infty } |f( x) |\mathrm{d} x\leqslant \int _{-\infty }^{+\infty } g( x) < +\infty 
\end{gather*}


于是 $\displaystyle \int _{-\infty }^{+\infty } f( x)\mathrm{d} x$ 绝对收敛,进而 $\displaystyle \int _{-\infty }^{+\infty } f( x)\mathrm{d} x$ 收敛。

接着考虑 $\displaystyle \forall \varepsilon  >0,\exists X >0\ s.t.\forall n$


\begin{align*}
\left| \int _{-\infty }^{-X} f_{n}( x)\mathrm{d} x\right|  & \leqslant \left| \int _{-\infty }^{-X} g( x)\mathrm{d} x\right| < \varepsilon \\
\left| \int _{X}^{+\infty } f_{n}( x)\mathrm{d} x\right|  & \leqslant \left| \int _{X}^{+\infty } g( x)\mathrm{d} x\right| < \varepsilon \\
\left| \int _{-\infty }^{-X} f( x)\mathrm{d} x\right|  & \leqslant \left| \int _{-\infty }^{-X} g( x)\mathrm{d} x\right| < \varepsilon \\
\left| \int _{X}^{+\infty } f( x)\mathrm{d} x\right|  & \leqslant \left| \int _{X}^{+\infty } g( x)\mathrm{d} x\right| < \varepsilon 
\end{align*}


而 $\displaystyle f_{n}( x)$ 在 $\displaystyle ( -\infty ,+\infty )$ 内闭一致收敛到 $\displaystyle f( x)$,在 $\displaystyle [ -X,X]$ 一致收敛到 $\displaystyle f( x)$。

于是 $\displaystyle \exists N >0\ s.t.\ |f_{n}( x) -f( x) |< \frac{\varepsilon }{2X} ,\forall n >N$,进而 $\displaystyle \forall n >N$
\begin{gather*}
\left| \int _{-\infty }^{+\infty } f_{n}( x)\mathrm{d} x-\int _{-\infty }^{+\infty } f( x)\mathrm{d} x\right| \\
\leqslant \left| \int _{-\infty }^{-X} f_{n}( x)\mathrm{d} x-\int _{-\infty }^{-X} f( x)\mathrm{d} x\right| +\left| \int _{-X}^{X} f_{n}( x)\mathrm{d} x-\int _{-X}^{X} f( x)\mathrm{d} x\right| +\left| \int _{X}^{+\infty } f_{n}( x)\mathrm{d} x-\int _{X}^{+\infty } f( x)\mathrm{d} x\right| \\
< 3\varepsilon 
\end{gather*}


于是 $\displaystyle \forall \varepsilon  >0,\exists N >0\ s.t.\left| \int _{-\infty }^{+\infty } f_{n}( x)\mathrm{d} x-\int _{-\infty }^{+\infty } f( x)\mathrm{d} x\right| < \varepsilon ,\forall n >N$,于是有 $\displaystyle \lim _{n\rightarrow \infty }\int _{-\infty }^{+\infty } f_{n}( x)\mathrm{d} x-\int _{-\infty }^{+\infty } f( x)\mathrm{d} x\ \ $。\qed 









\begin{ques}
	(1)设 $\displaystyle f( x)$ 在区间 $\displaystyle [ a,b+1]$ 可导,且 $\displaystyle f'( x)$ 在 $\displaystyle [ a,b+1]$ 一致连续。证明 $\displaystyle F_{n}( x) =n\left[ f\left( x+\frac{1}{n}\right) -f( x)\right]( n=1,2,3,\cdots )$ 在 $\displaystyle [ a,b]$ 上一致收敛。

(2) 证明 $\displaystyle f_{n}( x) =n\left[\sqrt{x+\frac{1}{n}} -\sqrt{x}\right]( n=1,2,3,\cdots )$ 在 $\displaystyle ( 0,+\infty )$ 内闭一致收敛,但在 $\displaystyle ( 0,+\infty )$ 不一致收敛。
\end{ques}





(1)

由拉格朗日微分中值定理,$\displaystyle \exists \xi \in \left( x,x+\frac{1}{n}\right) \ s.t.\ f\left( x+\frac{1}{n}\right) =f( x) +f'( \xi )\frac{1}{n}$,那么 $\displaystyle F_{n}( x) =f'( \xi ) ,\xi \in \left( x,x+\frac{1}{n}\right)$。

因为 $\displaystyle f'( x)$ 一致连续,故 $\displaystyle \forall \varepsilon  >0,\exists \delta  >0\ s.t.\ |f( x') -f( x'') |< \varepsilon ,\forall |x'-x''|< \delta $。

因此 $\displaystyle \forall \varepsilon  >0,\ \exists \delta  >0,N=\frac{1}{\delta } \ s.t.\ |f( x') -f( x'') |< \varepsilon ,\forall |x'-x''|< \delta ,|F_{n}( x) -F_{m}( x) |< \varepsilon ,\forall n,m >N$。

由柯西准则 $\displaystyle F_{n}( x)$ 在 $\displaystyle [ a,b]$ 上一致连续。



(2)

$\displaystyle \forall [ a,b] \subset ( 0,+\infty )$,有 $\displaystyle f( x) =\sqrt{x}$ 在 $\displaystyle [ a,b+1]$ 上可导,$\displaystyle f'( x) =\frac{1}{2\sqrt{x}}$ 在 $\displaystyle [ a,b+1]$ 上一致连续,进而 $\displaystyle f_{n}( x)$ 在 $\displaystyle [ a,b]$ 上收敛,在 $\displaystyle ( 0,+\infty )$ 上内闭一致收敛。

接着考虑 $\displaystyle f_{n}( x)\rightarrow \frac{1}{2\sqrt{x}}$,而令 $\displaystyle x_{n} =\frac{1}{n}$,则 $\displaystyle f_{n}( x_{n}) =\left(\sqrt{2} -1\right)\sqrt{n} ,|f_{n}( x_{n}) -f( x_{n}) |=\left(\sqrt{2} -\frac{3}{2}\right)\sqrt{n}$ 不趋向于 $\displaystyle 0$,因此 $\displaystyle f_{n}( x)$ 在 $\displaystyle ( 0,+\infty )$ 上不一致收敛。\qed 





\begin{ques}
	设在 $\displaystyle [ -1,1]$ 定义的非负函数列 $\displaystyle f_{n}( x)( n=1,2,3,\cdots )$ 满足:

(1)对 $\displaystyle \forall n,\int _{-1}^{1} f_{n}( x)\mathrm{d} x=1$;

(2)对任意的 $\displaystyle \delta  >0,f_{n}( x) \rightrightarrows 0,x\in [ -1,-\delta ] \cup [ \delta ,1]$。

再设 $\displaystyle g( x) \in R[ -1,1]$ 且在 $\displaystyle x=0$ 连续,证明 $\displaystyle \lim _{n\rightarrow \infty }\int _{-1}^{1} f_{n}( x) g( x)\mathrm{d} x=g( 0)$。

\end{ques}


因为 $\displaystyle g( x) \in R[ -1,1]$,故 $\displaystyle \exists M >0\ s.t.\ |g( x) |\leqslant M,\forall x\in [ -1,1]$。

$\displaystyle \forall \delta  >0,f_{n}( x) \rightrightarrows 0,x\in [ -1,-\delta ] \cup [ \delta ,1]$,于是 $\displaystyle \forall \delta ,\varepsilon  >0,\exists N >0\ s.t.\ |f_{n}( x) |< \varepsilon ,\forall x\in [ -1,-\delta ] \cup [ \delta ,1] ,n >N$。

同时,因为 $\displaystyle g( x)$ 在 $\displaystyle x=0$ 处连续,于是 $\displaystyle \forall \varepsilon  >0,\exists \delta  >0\ s.t.\ |g( x) -g( 0) |< \varepsilon ,\forall x\in ( -\delta ,+\delta )$。

进而,$\displaystyle \forall \varepsilon  >0,\exists \delta  >0\ s.t.\ g( 0) -\varepsilon < g( x) < g( 0) +\varepsilon ,\forall x\in ( -\delta ,+\delta ) ,$ 同时 $\displaystyle |f_{n}( x) |< \varepsilon ( \forall n >N) ,\forall x\in [ -1,-\delta ] \cup [ \delta ,1] ,1-2\varepsilon ( 1-\delta ) \leqslant \int _{-\delta }^{\delta } f( x)\mathrm{d} x\leqslant 1$,于是有:

$\displaystyle  \begin{array}{{>{\displaystyle}l}}
g( 0) -\delta -2\varepsilon ( 1-\delta )( g( 0) +\delta ) -g( 0) -\varepsilon M\leqslant \int _{-1}^{1} f_{n}( x) g( x)\mathrm{d} x-g( 0) \leqslant g( 0) +\delta -g( 0) +\varepsilon M\\
-\delta -2\varepsilon ( 1-\delta )( g( 0) +\delta ) -\varepsilon M\leqslant \int _{-1}^{1} f_{n}( x) g( x)\mathrm{d} x-g( 0) \leqslant \delta +\varepsilon M( \forall n >N)
\end{array}$

因为 $\displaystyle \delta $ 可以足够小,不妨假设 $\displaystyle \delta '=\min\{\delta ,\varepsilon \}$ 代替上式中的 $\displaystyle \delta $,进而不等式左右都是常数 $\displaystyle \times \varepsilon $,于是有$\displaystyle \lim _{n\rightarrow \infty }\int _{-1}^{1} f_{n}( x) g( x)\mathrm{d} x=g( 0)$。\qed 





\begin{ques}
	证明:$\displaystyle \sum _{n=0}^{\infty }\int _{0}^{1} t^{n}\sin \pi t\mathrm{d} t=\int _{0}^{1}\frac{\sin \pi t}{t}\mathrm{d} t$。
\end{ques}






\begin{align*}
\sum _{n=0}^{\infty }\int _{0}^{1} t^{n}\sin \pi t\mathrm{d} t & =\int _{0}^{1}\sum _{n=0}^{\infty } t^{n}\sin \pi t\mathrm{d} t & ( 1)\\
& =\int _{0}^{1}\frac{1}{1-t}\sin \pi t\mathrm{d} t & \\
& =\int _{0}^{1}\frac{1}{t}\sin \pi t\mathrm{d} t & ( 2)
\end{align*}


(2):换元法,$\displaystyle t'=1-t$。

(1):下面证明 $\displaystyle \sum _{n=0}^{\infty } t^{n} =\frac{1}{1-t}$ 在 $\displaystyle ( 0,1)$ 上内闭一致收敛,进而有$\displaystyle \sum _{n=0}^{\infty } t^{n}\sin \pi t=\frac{\sin \pi t}{1-t}$ 在 $\displaystyle ( 0,1)$ 上内闭一致收敛, $\displaystyle \sum _{n=0}^{\infty }\int _{0}^{1} t^{n}\sin \pi t\mathrm{d} t=\int _{0}^{1}\sum _{n=0}^{\infty } t^{n}\sin \pi t\mathrm{d} t$:



对于 $\displaystyle [ a,b] \subset ( 0,1)$ 记 $\displaystyle R( k) =\sum _{n=k}^{\infty } t^{n} =\frac{t^{k}}{1-t} \leqslant \frac{a^{k}}{1-b}$,则 $\displaystyle \forall \varepsilon  >0,$ 令 $\displaystyle N=\lceil \log_{a}(( 1-b) \varepsilon ) \rceil  >0,$ 则 $\displaystyle \forall k >N\ s.t.\ \left(\frac{1}{1-t} -\sum _{n=0}^{k} t^{n}\right) \leqslant \varepsilon $,于是 $\displaystyle \sum _{n=0}^{\infty } t^{n} =\frac{1}{1-t}$ 在 $\displaystyle ( 0,1)$ 上内闭一致收敛。\qed 







\begin{ques}
	证明函数 $\displaystyle f( x) =\sum _{n=1}^{\infty }\frac{\sin\left( xe^{n}\right)}{n^{n}}$ 在 $\displaystyle ( -\infty ,+\infty )$ 内具有连续的导函数。
\end{ques}



在 $\displaystyle x\in ( -\infty ,+\infty )$ 上考虑


\begin{equation*}
f( x) =\sum _{n=1}^{\infty }\frac{\cos\left( xe^{n}\right)}{\left(\frac{n}{e}\right)^{n}} =\sum _{n=1}^{\infty }\frac{\cos\left( xe^{n}\right)}{\left(\frac{n}{e}\right)^{2}}\frac{1}{\left(\frac{n}{e}\right)^{n-2}}
\end{equation*}


对于 
\begin{equation*}
\frac{1}{\left(\frac{n}{e}\right)^{n-2}}
\end{equation*}


其关于 $\displaystyle n$ 单调且 $\displaystyle \rightrightarrows 0$,而对于


\begin{equation*}
\left| \sum _{n=1}^{\infty }\frac{\cos\left( xe^{n}\right)}{\left(\frac{n}{e}\right)^{2}}\right| \leqslant e^{2}\left| \sum _{n=1}^{\infty }\frac{1}{n^{2}}\right| 
\end{equation*}


$\displaystyle \left| \sum _{n=1}^{\infty }\frac{1}{n^{2}}\right| $ 收敛,因此 $\displaystyle \sum _{n=1}^{\infty }\frac{\cos\left( xe^{n}\right)}{\left(\frac{n}{e}\right)^{n}}$ 一致有界,由狄利克雷判别法,$\displaystyle \sum _{n=1}^{\infty }\frac{\cos\left( xe^{n}\right)}{\left(\frac{n}{e}\right)^{n}}$ 一致收敛。

并且当 $\displaystyle x=0$ 时,$\displaystyle \lim _{n\rightarrow \infty }\frac{\sin\left( xe^{n}\right)}{n^{n}} =0$ 存在,因此有:


\begin{equation*}
f'( x) =\left(\sum _{n=1}^{\infty }\frac{\sin\left( xe^{n}\right)}{n^{n}}\right)^{'} =\sum _{n=1}^{\infty }\left(\frac{\sin\left( xe^{n}\right)}{n^{n}}\right)^{'} =\sum _{n=1}^{\infty }\frac{e^{n}\cos\left( xe^{n}\right)}{n^{n}}
\end{equation*}


同时,因为 $\displaystyle \sum _{n=1}^{\infty }\frac{\cos\left( xe^{n}\right)}{\left(\frac{n}{e}\right)^{n}}$ 一致收敛,且 $\displaystyle \frac{\cos\left( xe^{n}\right)}{\left(\frac{n}{e}\right)^{n}}$ 连续,因此 $\displaystyle \sum _{n=1}^{\infty }\frac{\cos\left( xe^{n}\right)}{\left(\frac{n}{e}\right)^{n}}$ 连续,函数 $\displaystyle f( x) =\sum _{n=1}^{\infty }\frac{\sin\left( xe^{n}\right)}{n^{n}}$ 在 $\displaystyle ( -\infty ,+\infty )$ 内具有连续的导函数。\qed 





\ifx\allfiles\undefined
\end{document}
\fi