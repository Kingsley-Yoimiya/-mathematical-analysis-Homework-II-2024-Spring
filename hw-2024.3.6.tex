%-*-    coding: UTF-8   -*-
% !TEX program = xelatex
\ifx\allfiles\undefined
%\special{dvipdfmx:config z 0}% 取消压缩,加快编译速度
\documentclass[UTF-8]{ctexart}
\usepackage{graphicx}
\usepackage{subfigure}
\usepackage{xcolor}
\usepackage{amsmath}
\usepackage{amssymb}
\usepackage{tabularx}
\usepackage{amssymb}
\usepackage{amsthm}
%\usepackage[usenames,dvipsnames]{color}
\usepackage{hyperref}
\hypersetup{
	colorlinks=true,
	linkcolor=black,
	filecolor=black,      
	urlcolor=red,
	citecolor=black,
}
\usepackage{geometry}
\geometry{a4paper,centering,scale=0.80}
\usepackage[format=hang,font=small,textfont=it]{caption}
\usepackage[nottoc]{tocbibind}
\usepackage{algorithm}  
\usepackage{algorithmicx}  
\usepackage{algpseudocode}
\usepackage{prettyref}
\usepackage{framed}
\setlength{\parindent}{2em}
\usepackage{indentfirst}
\usepackage[framemethod=TikZ]{mdframed}
\newcounter{ques}[section]
\renewcommand{\theques}{\arabic{section}.\arabic{ques}}
\newcommand{\setParDis}{\setlength {\parskip} {0.3cm} }
\newcommand{\setParDef}{\setlength {\parskip} {0pt} }
\setParDis% 调整这一个subsection的段落间距
%\setParDef%恢复间距

\newenvironment{ques}[1][]{
	\refstepcounter{ques}
	\mdfsetup{
		frametitle={
			\tikz[baseline=(current bounding box.east), outer sep=0pt]
			\node[anchor=east,rectangle,fill=blue!20]
			{\strut Problem~\theques\ifstrempty{#1}{}{:~#1}};},
		innertopmargin=10pt,linecolor=blue!20,
		linewidth=2pt,topline=true,
		frametitleaboveskip=\dimexpr-\ht\strutbox\relax
	}
	\begin{mdframed}[]\relax
}{\end{mdframed}}

\newcounter{Thm}[section]
\renewcommand{\theThm}{\arabic{section}.\arabic{Thm}}
\newenvironment{Thm}[1][]{
	\refstepcounter{Thm}
	\mdfsetup{
		frametitle={
			\tikz[baseline=(current bounding box.east), outer sep=0pt]
			\node[anchor=east,rectangle,fill=blue!20]
			{\strut Theorem~\theThm\ifstrempty{#1}{}{:~#1}};},
		innertopmargin=10pt,linecolor=blue!20,
		linewidth=2pt,topline=true,
		frametitleaboveskip=\dimexpr-\ht\strutbox\relax
	}
	\begin{mdframed}[]\relax
}{\end{mdframed}}

\newcounter{Defi}[section]
\renewcommand{\theDefi}{\arabic{section}.\arabic{Defi}}
\newenvironment{Defi}[1][]{
	\refstepcounter{Defi}
	\mdfsetup{
		frametitle={
			\tikz[baseline=(current bounding box.east), outer sep=0pt]
			\node[anchor=east,rectangle,fill=blue!20]
			{\strut Definition~\theDefi\ifstrempty{#1}{}{:~#1}};},
		innertopmargin=10pt,linecolor=blue!20,
		linewidth=2pt,topline=true,
		frametitleaboveskip=\dimexpr-\ht\strutbox\relax
	}
	\begin{mdframed}[]\relax
}{\end{mdframed}}

\newrefformat{qlt}{\underline{性质 \ref{#1}}}
\newcommand{\tpf}[2]{\begin{ques}[#1]{\kaishu #2}\end{ques}}
\newcommand{\pf}[1]{\begin{ques}{\kaishu #1}\end{ques}}
\newcommand{\tthm}[2]{\begin{Thm}[#1]{\kaishu #2}\end{Thm}}
\newcommand{\thm}[1]{\begin{Thm}{\kaishu #1}\end{Thm}}
\newcommand{\tdefi}[2]{\begin{Defi}[#1]{\kaishu #2}\end{Defi}}
\newcommand{\defi}[1]{\begin{Defi}{\kaishu #1}\end{Defi}}
\newcommand{\opf}[1]{{\kaishu{#1}}}
\title{数学分析 I 作业(2024. Spring)}
\author{\texttt{As-The-Wind}}

\date{2024 年 2 月 19 日 $\rightarrow$ \today}

\date{}
\author{尹锦润}
\begin{document}
\maketitle
\fi

\section{2024.3.6 作业}


\begin{ques}
	设 $\displaystyle f( x)$ 在 $\displaystyle [ -\pi ,\pi ]$ 单调下降,证明:对于正整数 $\displaystyle n$ 有 \ (1)$\displaystyle \int _{-\pi }^{\pi } f( x)\sin 2nx\mathrm{d} x\geqslant 0$;(2)$\displaystyle \int _{-\pi }^{\pi } f( x)\sin( 2n+1) x\mathrm{d} x\leqslant 0$。
\end{ques}





(1)


\begin{align*}
	\int _{-\pi }^{\pi } f( x)\sin 2nx\mathrm{d} x & =\sum _{i=0}^{n-1}\left(\int _{-\pi +\frac{i}{n} \pi }^{-\pi +\frac{2i+1}{2n} \pi } f( x)\sin 2nx\mathrm{d} x+\int _{-\pi +\frac{2i+1}{n} \pi }^{-\pi +\frac{i+1}{n} \pi } f( x)\sin 2nx\mathrm{d} x\right)\\
	& =\sum _{i=0}^{n-1}\left(\int _{-\pi +\frac{i}{n} \pi }^{-\pi +\frac{2i+1}{2n} \pi }\left( f( x) -f\left( -2\pi +\frac{2i+1}{n} \pi -x\right)\right)\sin 2nx\mathrm{d} x\right)
\end{align*}

而 $\displaystyle f( x)$ 在 $\displaystyle [ -\pi ,\pi ]$ 单调下降,$\displaystyle f( x) -f\left( -2\pi +\frac{2i+1}{n} \pi -x\right) \geqslant 0$,同时 $\displaystyle \sin 2nx\geqslant 0,\forall x\in \left[ -\pi +\frac{i}{n} \pi ,-\pi +\frac{2i+1}{2n} \pi \right]$,因此

\begin{equation*}
	\int _{-\pi }^{\pi } f( x)\sin 2nx\mathrm{d} x=\sum _{i=0}^{n-1}\left(\int _{-\pi +\frac{i}{n} \pi }^{-\pi +\frac{2i+1}{2n} \pi }\left( f( x) -f\left( -2\pi +\frac{2i+1}{n} \pi -x\right)\right)\sin 2nx\mathrm{d} x\right) \geqslant 0
\end{equation*}\qed 

(2)
\begin{align*}
	\int _{-\pi }^{\pi } f( x)\sin( 2n+1) x\mathrm{d} x & =\sum _{i=0}^{n-1}\left(\int _{-\pi +\frac{i}{n} \pi }^{-\pi +\frac{2i+1}{2n} \pi } f( x)\sin( 2n+1) x\mathrm{d} x+\int _{-\pi +\frac{2i+1}{n} \pi }^{-\pi +\frac{i+1}{n} \pi } f( x)\sin( 2n+1) x\mathrm{d} x\right)\\
	& =\sum _{i=0}^{n-1}\left(\int _{-\pi +\frac{i}{n} \pi }^{-\pi +\frac{2i+1}{2n} \pi }\left( f( x) -f\left( -2\pi +\frac{2i+1}{n} \pi -x\right)\right)\sin( 2n+1) x\mathrm{d} x\right)
\end{align*}

而 $\displaystyle f( x)$ 在 $\displaystyle [ -\pi ,\pi ]$ 单调下降,$\displaystyle f( x) -f\left( -2\pi +\frac{2i+1}{n} \pi -x\right) \geqslant 0$,同时 $\displaystyle \sin( 2n+1) x\leqslant 0,\forall x\in \left[ -\pi +\frac{i}{n} \pi ,-\pi +\frac{2i+1}{2n} \pi \right]$,因此

\begin{equation*}
	\int _{-\pi }^{\pi } f( x)\sin( 2n+1) x\mathrm{d} x=\sum _{i=0}^{n-1}\left(\int _{-\pi +\frac{i}{n} \pi }^{-\pi +\frac{2i+1}{2n} \pi }\left( f( x) -f\left( -2\pi +\frac{2i+1}{n} \pi -x\right)\right)\sin( 2n+1) x\mathrm{d} x\right) \leqslant 0
\end{equation*}\qed 





\begin{ques}
	设 $\displaystyle f( x) \in D[ a,b] ,f'( x)$ 在 $\displaystyle [ a,b]$ 上单调递增,$\displaystyle |f'( x) |\geqslant m >0,\forall x\in [ a,b]$,求证:$\displaystyle \left| \int _{a}^{b}\cos( f( x))\mathrm{d} x\right| \leqslant \frac{2}{m}$。
\end{ques}



因为 $\displaystyle |f'( x) \geqslant m >0$,因此 $\displaystyle f( x)$ 单调上升,令 $\displaystyle g( x) =f^{-1}( x)$ 存在且单调上升。


\begin{align*}
	\int _{a}^{b}\cos( f( x))\mathrm{d} x & =\int _{f( a)}^{f( b)}\cos( u)\frac{1}{f'( g( u))}\mathrm{d} u\\
	& =\frac{1}{f'( g( f( b)))}\int _{\xi }^{f( b)}\cos u\mathrm{d} u( \xi \in [ f( a) ,f( b)] &(1)\\
	\left| \int _{a}^{b}\cos( f( x))\mathrm{d} x\right|  & \leqslant \frac{2}{m} & (2)
\end{align*}


(1):定积分第二中值定理,$\displaystyle \frac{1}{f'( g( u))}$ 单调下降。

(2):$\displaystyle \left| \int _{\xi }^{f( b)}\cos u\mathrm{d} u\right| \leqslant 2$。\qed 



\begin{ques}
	设 $\displaystyle f( x) \in C^{1}[ 0,1]$,证明:$\displaystyle \lim _{n\rightarrow +\infty } n\int _{0}^{1} x^{n} f( x)\mathrm{d} x=f( 1)$。
\end{ques}
\begin{align*}
	\int _{0}^{1} x^{n} f( x)\mathrm{d} x & =\left. \frac{1}{n+1} x^{n+1} f( x)\middle| _{0}^{1}\right. -\frac{1}{n+1}\int _{0}^{1} x^{n+1} f'( x)\mathrm{d} x\\
	& =\frac{f( 1)}{n+1} -\frac{1}{n+1}\int _{0}^{1} x^{n+1} f'( x)\mathrm{d} x
\end{align*}


因为 $\displaystyle f( x) \in C^{1}[ 0,1]$,于是 $\displaystyle |f'( x) |$ 由上界,记为 $\displaystyle M$。

进而
\begin{align*}
	\left| \frac{1}{n+1}\int _{0}^{1} x^{n+1} f'( x)\mathrm{d} x\right|  & \leqslant \frac{M}{n+1}\int _{0}^{1} x^{n+1}\mathrm{d} x\\
	& =\frac{M}{n+1}\frac{1}{n+2}\\
	n\left| \frac{1}{n+1}\int _{0}^{1} x^{n+1} f'( x)\mathrm{d} x\right|  & \rightarrow 0\left( n\rightarrow +\infty \right)
\end{align*}


因此 $\displaystyle n\rightarrow +\infty $ 时,有 $\displaystyle n\int _{0}^{1} x^{n} f( x)\mathrm{d} x=\frac{f( 1)}{n+1} -\frac{1}{n+1}\int _{0}^{1} x^{n+1} f'( x)\mathrm{d} x\rightarrow \frac{nf( 1)}{n+1}\rightarrow f( 1)$。\qed 





\begin{ques}
	设 $\displaystyle f( x)$ 于 $\displaystyle [ a,b]$ 非负,连续,严格递增,则 $\displaystyle \forall p >0$,据定积分第一中值定理,$\displaystyle \exists x_{p} \in [ a,b] \ s.t.\ f^{p}( x_{p}) =\frac{1}{b-a}\int _{a}^{b} f^{p}( t)\mathrm{d} t$。证明:$\displaystyle \lim _{p\rightarrow +\infty } x_{p} =b$。
\end{ques}





$\displaystyle \forall \varepsilon \in \left( 0,\frac{b-a}{2}\right)$,有 $\displaystyle \int _{a}^{b} f^{p}( t)\mathrm{d} t\geqslant \int _{b-\varepsilon }^{b} f^{p}( t)\mathrm{d} t >f^{p}( b-\varepsilon ) \varepsilon $。

考虑不等式 $\displaystyle f^{p}( b-\varepsilon ) \varepsilon \geqslant f^{p}( b-2\varepsilon )( b-a) \Leftrightarrow \left(\frac{f( b-\varepsilon )}{f( b-2\varepsilon )}\right)^{p} \geqslant \frac{( b-a)}{\varepsilon }$。

因为 $\displaystyle \frac{f( b-\varepsilon )}{f( b-2\varepsilon )}  >1$,于是一定存在 $\displaystyle L >0$,当 $\displaystyle p >L$ 时,有 $\displaystyle \left(\frac{f( b-\varepsilon )}{f( b-2\varepsilon )}\right)^{p} \geqslant \frac{( b-a)}{\varepsilon }$,进而成立


\begin{equation*}
	\int _{a}^{b} f^{p}( t)\mathrm{d} t >f^{p}( b-\varepsilon ) \varepsilon \geqslant f^{p}( b-2\varepsilon )( b-a)
\end{equation*}


于是,$\displaystyle \exists L >0,\forall p >L,x_{p}  >b-2\varepsilon $,进而 $\displaystyle \lim _{p\rightarrow +\infty } x_{p} =b$。\qed 



\begin{ques}
	设 $\displaystyle f( x) \in C[ a,b]$,据积分中值定理,$\displaystyle \forall x\in [ a,b] ,\exists \xi \in ( a,x) \ s.t.\ \int _{a}^{x} f( t)\mathrm{d} t=f( \xi )( x-a)$。若 $\displaystyle f'( a)$ 存在且非零,证明:$\displaystyle \lim _{x\rightarrow a}\frac{\xi -a}{x-a} =\frac{1}{2}$。
\end{ques}
\begin{align*}
	\lim _{x\rightarrow a}\frac{\xi -a}{x-a} & =\lim _{x\rightarrow a}\frac{\xi -a}{f( \xi ) -f( a)} \times \frac{f( \xi ) -f( a)}{x-a}\\
	& =\frac{1}{f'( a)}\lim _{x\rightarrow a}\frac{f( \xi ) -f( a)}{x-a}\\
	& =\frac{1}{f'( a)}\lim _{x\rightarrow a}\frac{\int _{a}^{x} f( t)\mathrm{d} t-( x-a) f( a)}{( x-a)^{2}}\\
	& =\frac{1}{f'( a)}\lim _{x\rightarrow a}\frac{f( x) -f( a)}{2( x-a)} & (1)\\
	& =\frac{1}{f'( a)}\frac{f'( a)}{2}\\
	& =\frac{1}{2}
\end{align*}

(1):洛必达法则,$\displaystyle x\rightarrow a$ 时, $\displaystyle ( x-a)^{2}\rightarrow 0$ 且 $\displaystyle \int _{a}^{x} f( t)\mathrm{d} t-( x-a) f( a)\rightarrow 0-0=0$,同时 $\displaystyle \lim _{x\rightarrow a}\frac{\int _{a}^{x} f( t)\mathrm{d} t-( x-a) f( a)}{( x-a)^{2}}$ 存在为 $\displaystyle \frac{1}{2}$。\qed 

\ifx\allfiles\undefined
\end{document}
\fi