%-*-    coding: UTF-8   -*-
% !TEX program = xelatex
\ifx\allfiles\undefined
%\special{dvipdfmx:config z 0}% 取消压缩,加快编译速度
\documentclass[UTF-8]{ctexart}
\usepackage{graphicx}
\usepackage{subfigure}
\usepackage{xcolor}
\usepackage{amsmath}
\usepackage{amssymb}
\usepackage{tabularx}
\usepackage{amssymb}
\usepackage{amsthm}
%\usepackage[usenames,dvipsnames]{color}
\usepackage{hyperref}
\hypersetup{
	colorlinks=true,
	linkcolor=black,
	filecolor=black,      
	urlcolor=red,
	citecolor=black,
}
\usepackage{geometry}
\geometry{a4paper,centering,scale=0.80}
\usepackage[format=hang,font=small,textfont=it]{caption}
\usepackage[nottoc]{tocbibind}
\usepackage{algorithm}  
\usepackage{algorithmicx}  
\usepackage{algpseudocode}
\usepackage{prettyref}
\usepackage{framed}
\setlength{\parindent}{2em}
\usepackage{indentfirst}
\usepackage[framemethod=TikZ]{mdframed}
\newcounter{ques}[section]
\renewcommand{\theques}{\arabic{section}.\arabic{ques}}
\newcommand{\setParDis}{\setlength {\parskip} {0.3cm} }
\newcommand{\setParDef}{\setlength {\parskip} {0pt} }
\setParDis% 调整这一个subsection的段落间距
%\setParDef%恢复间距

\newenvironment{ques}[1][]{
	\refstepcounter{ques}
	\mdfsetup{
		frametitle={
			\tikz[baseline=(current bounding box.east), outer sep=0pt]
			\node[anchor=east,rectangle,fill=blue!20]
			{\strut Problem~\theques\ifstrempty{#1}{}{:~#1}};},
		innertopmargin=10pt,linecolor=blue!20,
		linewidth=2pt,topline=true,
		frametitleaboveskip=\dimexpr-\ht\strutbox\relax
	}
	\begin{mdframed}[]\relax
}{\end{mdframed}}

\newcounter{Thm}[section]
\renewcommand{\theThm}{\arabic{section}.\arabic{Thm}}
\newenvironment{Thm}[1][]{
	\refstepcounter{Thm}
	\mdfsetup{
		frametitle={
			\tikz[baseline=(current bounding box.east), outer sep=0pt]
			\node[anchor=east,rectangle,fill=blue!20]
			{\strut Theorem~\theThm\ifstrempty{#1}{}{:~#1}};},
		innertopmargin=10pt,linecolor=blue!20,
		linewidth=2pt,topline=true,
		frametitleaboveskip=\dimexpr-\ht\strutbox\relax
	}
	\begin{mdframed}[]\relax
}{\end{mdframed}}

\newcounter{Defi}[section]
\renewcommand{\theDefi}{\arabic{section}.\arabic{Defi}}
\newenvironment{Defi}[1][]{
	\refstepcounter{Defi}
	\mdfsetup{
		frametitle={
			\tikz[baseline=(current bounding box.east), outer sep=0pt]
			\node[anchor=east,rectangle,fill=blue!20]
			{\strut Definition~\theDefi\ifstrempty{#1}{}{:~#1}};},
		innertopmargin=10pt,linecolor=blue!20,
		linewidth=2pt,topline=true,
		frametitleaboveskip=\dimexpr-\ht\strutbox\relax
	}
	\begin{mdframed}[]\relax
}{\end{mdframed}}

\newrefformat{qlt}{\underline{性质 \ref{#1}}}
\newcommand{\tpf}[2]{\begin{ques}[#1]{\kaishu #2}\end{ques}}
\newcommand{\pf}[1]{\begin{ques}{\kaishu #1}\end{ques}}
\newcommand{\tthm}[2]{\begin{Thm}[#1]{\kaishu #2}\end{Thm}}
\newcommand{\thm}[1]{\begin{Thm}{\kaishu #1}\end{Thm}}
\newcommand{\tdefi}[2]{\begin{Defi}[#1]{\kaishu #2}\end{Defi}}
\newcommand{\defi}[1]{\begin{Defi}{\kaishu #1}\end{Defi}}
\newcommand{\opf}[1]{{\kaishu{#1}}}
\title{数学分析 I 作业(2024. Spring)}
\author{\texttt{As-The-Wind}}

\date{2024 年 2 月 19 日 $\rightarrow$ \today}

\date{}
\author{尹锦润}
\begin{document}
\maketitle
\fi

\section{2024.3.25 作业}

\begin{ques}
	讨论积分的敛散性:$\displaystyle I=\int _{0}^{1}\frac{\sin\frac{1}{x}}{x^{\frac{3}{2}}\ln\left( 1+\frac{1}{x}\right)}\mathrm{d} x$。
\end{ques}

\begin{align*}
	I & =\int _{0}^{1}\frac{\sin\frac{1}{x}}{x^{\frac{3}{2}}\ln\left( 1+\frac{1}{x}\right)}\mathrm{d} x & \\
	& =-\int _{1}^{+\infty }\frac{\sin t}{\ln( 1+t) t^{\frac{1}{2}}}\mathrm{d} t & t=\frac{1}{x}
\end{align*}


因为 $\displaystyle \frac{1}{\ln( 1+t) t^{\frac{1}{2}}}$ 单调且 $\displaystyle \rightarrow 0\left( t\rightarrow +\infty \right)$,$\displaystyle \left| \int _{1}^{+\infty }\sin t\mathrm{d} t\right| \leqslant 2$,因此 $\displaystyle I$ 收敛。\qed 





\begin{ques}
	讨论积分敛散性:$\displaystyle I=\int _{0}^{+\infty }\left(\left( 1-\frac{\sin x}{x}\right)^{-\alpha } -1\right)\mathrm{d} x,\alpha  >0$。
\end{ques}


\begin{align*}
	I & =\int _{0}^{1}\left(\left( 1-\frac{\sin x}{x}\right)^{-\alpha } -1\right)\mathrm{d} x+\int _{1}^{+\infty }\left(\left( 1-\frac{\sin x}{x}\right)^{-\alpha } -1\right)\mathrm{d} x\\
	& =I_{1} +I_{2}
\end{align*}


考虑 $\displaystyle I_{1}$,因 $\displaystyle \sin x=x-\frac{1}{6} x^{3} +o\left( x^{5}\right)\left( x\rightarrow 0\right)$,则 $\displaystyle I_{1}$ 的敛散性和:
\begin{equation*}
	\int _{0}^{1}\left(\left( -\frac{1}{6}\right)^{-\alpha } x^{-2\alpha } -1\right)\mathrm{d} x
\end{equation*}


相同,于是当 $\displaystyle -2\alpha  >-1\ i.e.\ \alpha < \frac{1}{2}$ 时 $\displaystyle I_{1}$ 收敛,当 $\displaystyle \alpha \geqslant \frac{1}{2}$ 时 $\displaystyle I_{1}$ 发散。

考虑 $\displaystyle I_{2}$,因 $\displaystyle \left( 1-\frac{\sin x}{x}\right)^{-\alpha } -1=\alpha \frac{\sin x}{x} +o\left(\left(\frac{\sin x}{x}\right)^{2}\right) =( \alpha +o( 1))\left(\frac{\sin x}{x}\right) ,x\rightarrow +\infty $,因此 $\displaystyle I_{2}$ 收敛。

因此,$\displaystyle 0< \alpha < \frac{1}{2}$ 时 $\displaystyle I$ 收敛,$\displaystyle \alpha \geqslant \frac{1}{2}$ 时 $\displaystyle I$ 发散。\qed 



\begin{ques}
	讨论 $\displaystyle I=\int _{0}^{+\infty }\frac{\sin x^{2}}{x^{p}}\mathrm{d} x$ 的敛散性和绝对收敛性,其中 $\displaystyle p\in \mathbb{R}$ 是常数。
\end{ques}

\begin{align*}
	I & =\int _{0}^{1}\frac{\sin x^{2}}{x^{p}}\mathrm{d} x+\int _{1}^{+\infty }\frac{\sin x^{2}}{x^{p}}\mathrm{d} x\\
	& =I_{1} +I_{2}
\end{align*}


对于 $\displaystyle I_{2}$,有


\begin{align*}
	I_{2} & =\int _{1}^{+\infty }\frac{\sin t}{2t^{\frac{p+1}{2}}}\mathrm{d} t
\end{align*}
因此,当 $\displaystyle \frac{p+1}{2}  >1\ i.e.\ p >1$ 时 $\displaystyle I_{2}$ 绝对收敛,当 $\displaystyle 0< \frac{p+1}{2} \leqslant 1\ i.e.\ -1< p\leqslant 1$ 时 $\displaystyle I_{2}$ 条件收敛。

对于 $\displaystyle I_{1}$,因为 $\displaystyle \sin x^{2} \sim x^{2}\left( x\rightarrow 0\right)$,于是其和 
\begin{equation*}
	\int _{0}^{1}\frac{1}{x^{p-2}}\mathrm{d} x
\end{equation*}


敛散性相同,当 $\displaystyle p\geqslant 3$ 时 $\displaystyle I_{1}$ 发散,当 $\displaystyle p< 3$ 时收敛。

综上所述,$\displaystyle I$ 在 $\displaystyle -1< p\leqslant 1$ 时条件收敛,$\displaystyle 1< p< 3$ 时绝对收敛,$\displaystyle p\geqslant 3$ 时发散。\qed 





\begin{ques}
	讨论积分的收敛性和绝对收敛性:$\displaystyle I=\int _{0}^{+\infty }\frac{x^{\alpha }\sin x}{1+x^{\beta }}\mathrm{d} x$,其中 $\displaystyle \alpha ,\beta $ 为常数。
\end{ques}


\begin{align*}
	I & =\int _{0}^{1}\frac{x^{\alpha }\sin x}{1+x^{\beta }}\mathrm{d} x+\int _{1}^{+\infty }\frac{x^{\alpha }\sin x}{1+x^{\beta }}\mathrm{d} x\\
	& =I_{1} +I_{2}
\end{align*}


对于 $\displaystyle I_{2}$ 有 $\displaystyle \frac{x^{\alpha }\sin x}{1+x^{\beta }} =\frac{\sin x}{x^{-\alpha } +x^{\beta -\alpha }}$。

当 $\displaystyle \beta -\alpha \geqslant -\alpha \ i.e.\ \beta \geqslant 0$ 时,$\displaystyle \frac{x^{\alpha }\sin x}{1+x^{\beta }} =\frac{\sin x}{x^{\beta -\alpha }}\frac{x^{\beta -\alpha }}{x^{-\alpha } +x^{\beta -\alpha }}$,$\displaystyle \frac{x^{\beta -\alpha }}{x^{-\alpha } +x^{\beta -\alpha }}$ 单调有界且定号,因此 \ $\displaystyle \frac{x^{\alpha }\sin x}{1+x^{\beta }}$ 和 $\displaystyle \frac{\sin x}{x^{\beta -\alpha }}$ 收敛性和绝对收敛性相同。

于是当 $\displaystyle \beta -\alpha  >1$ 时,$\displaystyle I_{2}$ 绝对收敛,当 $\displaystyle 0< \beta -\alpha \leqslant 1\ i.e.\ \alpha < \beta \leqslant \alpha +1$ 时,$\displaystyle I_{2}$ 条件收敛,其余时候 $\displaystyle I_{2}$ 发散。

当 $\displaystyle \beta < 0$ 时,类似地 $\displaystyle \frac{x^{\alpha }\sin x}{1+x^{\beta }}$ 和 $\displaystyle \frac{\sin x}{x^{-\alpha }}$ 收敛性和绝对收敛性相同,当 $\displaystyle -\alpha  >1\ i.e.\ \alpha < -1$ 时绝对收敛,当 $\displaystyle -1\leqslant \alpha < 0$ 时条件收敛,其余情况发散。



对于 $\displaystyle I_{1}$ 有 $\displaystyle \frac{x^{\alpha }\sin x}{1+x^{\beta }}\rightarrow \frac{1}{x^{-\alpha -1} +x^{\beta -\alpha -1}}\left( x\rightarrow 0\right)$。

当 $\displaystyle \beta \geqslant 0$ 时如果 $\displaystyle -\alpha -1< 1\ i.e.\ \alpha  >-2$ 绝对收敛,其余情况发散。

当 $\displaystyle \beta < 0$ 时如果 $\displaystyle \beta -\alpha -1< 1\ i.e.\ \beta -\alpha < 2$ 时绝对收敛,其余情况发散。

综上所述,

$\displaystyle \beta \geqslant 0$ 时,$\displaystyle \beta  >\alpha +1\land \alpha  >-2$ 时绝对收敛,$\displaystyle \alpha < \beta \leqslant \alpha +1\land \alpha  >-2$ 时条件收敛,其余情况发散。

$\displaystyle \beta < 0$ 时,$\displaystyle \alpha < -1\land \beta -\alpha < 2$ 时绝对收敛,$\displaystyle -1\leqslant \alpha < 0\land \beta -\alpha < 2$ 时条件收敛,其余情况发散。\qed 



\begin{ques}
	设 $\displaystyle f( x) \in C[ 0,+\infty ) ,\alpha =\lim _{x\rightarrow +\infty } f( x) ,\int _{0}^{1}\frac{f( x)}{x}\mathrm{d} x$ 收敛,证明:$\displaystyle \int _{0}^{+\infty }\frac{f( ax) -f( bx)}{x}\mathrm{d} x=\alpha \ln\frac{a}{b} ,\forall 0< a< b$。
\end{ques}


\begin{align*}
	I & =\int _{0}^{+\infty }\frac{f( ax) -f( bx)}{x}\mathrm{d} x & \\
	& =\int _{0}^{1}\frac{f( ax) -f( bx)}{x}\mathrm{d} x+\int _{1}^{+\infty }\frac{f( ax) -f( bx)}{x}\mathrm{d} x & \\
	& =I_{1} +I_{2} & \\
	I_{1} & =\int _{0}^{1}\frac{f( ax)}{x}\mathrm{d} x-\int _{0}^{1}\frac{f( bx)}{x}\mathrm{d} x & \\
	& =\int _{0}^{a}\frac{f( ax)}{x}\mathrm{d} x-\int _{0}^{b}\frac{f( bx)}{x}\mathrm{d} x & \\
	& =\int _{a}^{b}\frac{f( x)}{x}\mathrm{d} x & \\
	I_{2} & =\lim _{A\rightarrow +\infty }\int _{1}^{A}\frac{f( ax)}{x}\mathrm{d} x-\int _{1}^{A}\frac{f( bx)}{x}\mathrm{d} x & \\
	& =\left(\lim _{A\rightarrow +\infty }\int _{bA}^{aA}\frac{f( x)}{x}\mathrm{d} x\right) -\int _{a}^{b}\frac{f( x)}{x}\mathrm{d} x & \\
	& =\left(\lim _{A\rightarrow +\infty } f( \eta _{1})\ln\frac{a}{b}\right) -\int _{a}^{b}\frac{f( x)}{x}\mathrm{d} x & \eta _{1} \in ( aA,bA)( 1)\\
	& =\alpha \ln\frac{a}{b} -\int _{a}^{b}\frac{f( x)}{x}\mathrm{d} x & \\
	I & =I_{1} +I_{2} & \\
	& =\alpha \ln\frac{a}{b} & 
\end{align*}


(1):定积分第一中值定理。\qed 
\ifx\allfiles\undefined
\end{document}
\fi