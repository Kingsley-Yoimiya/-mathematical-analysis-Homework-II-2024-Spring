%-*-    coding: UTF-8   -*-
% !TEX program = xelatex
\ifx\allfiles\undefined
%\special{dvipdfmx:config z 0}% 取消压缩,加快编译速度
\documentclass[UTF-8]{ctexart}
\usepackage{graphicx}
\usepackage{subfigure}
\usepackage{xcolor}
\usepackage{amsmath}
\usepackage{amssymb}
\usepackage{tabularx}
\usepackage{amssymb}
\usepackage{amsthm}
%\usepackage[usenames,dvipsnames]{color}
\usepackage{hyperref}
\hypersetup{
	colorlinks=true,
	linkcolor=black,
	filecolor=black,      
	urlcolor=red,
	citecolor=black,
}
\usepackage{geometry}
\geometry{a4paper,centering,scale=0.80}
\usepackage[format=hang,font=small,textfont=it]{caption}
\usepackage[nottoc]{tocbibind}
\usepackage{algorithm}  
\usepackage{algorithmicx}  
\usepackage{algpseudocode}
\usepackage{prettyref}
\usepackage{framed}
\setlength{\parindent}{2em}
\usepackage{indentfirst}
\usepackage[framemethod=TikZ]{mdframed}
\newcounter{ques}[section]
\renewcommand{\theques}{\arabic{section}.\arabic{ques}}
\newcommand{\setParDis}{\setlength {\parskip} {0.3cm} }
\newcommand{\setParDef}{\setlength {\parskip} {0pt} }
\setParDis% 调整这一个subsection的段落间距
%\setParDef%恢复间距

\newenvironment{ques}[1][]{
	\refstepcounter{ques}
	\mdfsetup{
		frametitle={
			\tikz[baseline=(current bounding box.east), outer sep=0pt]
			\node[anchor=east,rectangle,fill=blue!20]
			{\strut Problem~\theques\ifstrempty{#1}{}{:~#1}};},
		innertopmargin=10pt,linecolor=blue!20,
		linewidth=2pt,topline=true,
		frametitleaboveskip=\dimexpr-\ht\strutbox\relax
	}
	\begin{mdframed}[]\relax
}{\end{mdframed}}

\newcounter{Thm}[section]
\renewcommand{\theThm}{\arabic{section}.\arabic{Thm}}
\newenvironment{Thm}[1][]{
	\refstepcounter{Thm}
	\mdfsetup{
		frametitle={
			\tikz[baseline=(current bounding box.east), outer sep=0pt]
			\node[anchor=east,rectangle,fill=blue!20]
			{\strut Theorem~\theThm\ifstrempty{#1}{}{:~#1}};},
		innertopmargin=10pt,linecolor=blue!20,
		linewidth=2pt,topline=true,
		frametitleaboveskip=\dimexpr-\ht\strutbox\relax
	}
	\begin{mdframed}[]\relax
}{\end{mdframed}}

\newcounter{Defi}[section]
\renewcommand{\theDefi}{\arabic{section}.\arabic{Defi}}
\newenvironment{Defi}[1][]{
	\refstepcounter{Defi}
	\mdfsetup{
		frametitle={
			\tikz[baseline=(current bounding box.east), outer sep=0pt]
			\node[anchor=east,rectangle,fill=blue!20]
			{\strut Definition~\theDefi\ifstrempty{#1}{}{:~#1}};},
		innertopmargin=10pt,linecolor=blue!20,
		linewidth=2pt,topline=true,
		frametitleaboveskip=\dimexpr-\ht\strutbox\relax
	}
	\begin{mdframed}[]\relax
}{\end{mdframed}}

\newrefformat{qlt}{\underline{性质 \ref{#1}}}
\newcommand{\tpf}[2]{\begin{ques}[#1]{\kaishu #2}\end{ques}}
\newcommand{\pf}[1]{\begin{ques}{\kaishu #1}\end{ques}}
\newcommand{\tthm}[2]{\begin{Thm}[#1]{\kaishu #2}\end{Thm}}
\newcommand{\thm}[1]{\begin{Thm}{\kaishu #1}\end{Thm}}
\newcommand{\tdefi}[2]{\begin{Defi}[#1]{\kaishu #2}\end{Defi}}
\newcommand{\defi}[1]{\begin{Defi}{\kaishu #1}\end{Defi}}
\newcommand{\opf}[1]{{\kaishu{#1}}}
\title{数学分析 I 作业(2024. Spring)}
\author{\texttt{As-The-Wind}}

\date{2024 年 2 月 19 日 $\rightarrow$ \today}

\date{}
\author{尹锦润}
\begin{document}
\maketitle
\fi

\section{2024.5.27 作业}

\begin{ques}
	求 $\displaystyle 2\pi $ 周期函数的傅里叶级数:$\displaystyle f( x) =|x|,-\pi \leqslant x< \pi $。
\end{ques}



因为是偶函数,仅需要求 $\displaystyle \cos nx$ 系数 $\displaystyle a_{n}$。


\begin{align*}
	a_{0} & =\frac{1}{\pi }\int _{-\pi }^{\pi } |x|\mathrm{d} x\\
	& =\pi \\
	a_{n} & =\frac{1}{\pi }\int _{-\pi }^{\pi } |x|\cos nx\mathrm{d} x\\
	& =\frac{2}{\pi }\int _{0}^{\pi } x\cos nx\mathrm{d} x\\
	& =\frac{2}{n\pi }\left( x\sin nx| _{0}^{\pi } -\int _{0}^{\pi }\sin nx\mathrm{d} x\right)\\
	& =\left. \frac{2}{n^{2} \pi }\cos nx\middle| _{0}^{\pi }\right. \\
	& =\begin{cases}
		-\frac{4}{n^{2} \pi } & n=2k-1\\
		0 & n=2k
	\end{cases} ,k\in N^{*}
\end{align*}

因此,$\displaystyle f( x) \sim \frac{\pi }{2} -\frac{4}{\pi }\sum _{n=1}^{+\infty }\frac{\cos( 2n-1) x}{( 2n-1)^{2}}$。\qed 



\begin{ques}
	求 $\displaystyle 2\pi $ 周期函数的傅里叶级数:$\displaystyle f( x) =\begin{cases}
	\frac{\pi }{4} & 0< x< \pi \\
	-\frac{\pi }{4} & -\pi \leqslant x\leqslant 0
\end{cases}$。
\end{ques}



因为是奇函数,仅需要求 $\displaystyle \sin nx$ 系数 $\displaystyle b_{n}$。


\begin{align*}
	b_{n} & =\frac{1}{\pi }\int _{-\pi }^{\pi } f( x)\sin nx\mathrm{d} x\\
	& =\frac{2}{\pi }\int _{0}^{\pi }\frac{\pi }{4}\sin nx\mathrm{d} x\\
	& =-\frac{1}{2n}(\cos nx | _{0}^{\pi })\\
	& =\begin{cases}
		\frac{1}{n} & ,n=2k-1\\
		0 & ,n=2k
	\end{cases} ,k\in N^{*}
\end{align*}

于是,$\displaystyle f( x) \sim \sum _{n=1}^{+\infty }\frac{\sin( 2n-1) x}{2n-1}$。\qed 





\begin{ques}
	求 $\displaystyle 2\pi $ 周期函数的正弦和余弦级数:$\displaystyle f( x) =\sin x,0\leqslant x\leqslant \pi $。
\end{ques}





对于正弦级数,$\displaystyle f( x) =\sin x$。

对于余弦级数,设 $\displaystyle f( x) =\frac{a_{0}}{2} +\sum _{n=1}^{+\infty } a_{n}\cos nx\mathrm{d} x$,则有


\begin{align*}
	a_{0} & =\frac{2}{\pi }\int _{0}^{\pi }\sin x\mathrm{d} x\\
	& =\frac{4}{\pi }\\
	a_{n} & =\frac{2}{\pi }\int _{0}^{\pi }\sin x\cos nx\mathrm{d} x\\
	& =\frac{2}{n^{2} \pi }\left[\cos x\cos nx | _{0}^{\pi } +\int _{0}^{\pi }\cos nx\sin x\mathrm{d} x\right]\\
	& =\begin{cases}
		0 & ,n=2k-1\\
		-\frac{4}{\left( n^{2} -1\right) \pi } & ,n=2k
	\end{cases} ,k\in N^{*}
\end{align*}

因此,$\displaystyle f( x) \sim \frac{2}{\pi } -\frac{4}{\pi }\sum _{n=1}^{+\infty }\frac{\cos 2nx}{\left( 4n^{2} -1\right) \pi }$。\qed 







\begin{ques}
	设 $\displaystyle f( x)$ 为 $\displaystyle \left( 0,\frac{\pi }{2}\right)$ 上的 Riemann 可积或绝对可积函数,请适当延拓 $\displaystyle f( x)$ 到区间 $\displaystyle ( -\pi ,\pi )$ 上,使 Fourier 级数具有形式 $\displaystyle \sum _{n=1}^{\infty } a_{2n-1}\cos( 2n-1) x$。
\end{ques}



考虑函数


\begin{equation*}
	g( x) =\begin{cases}
		f( \pi +x) & x\in \left( -\pi ,-\frac{\pi }{2}\right)\\
		0 & x=-\frac{\pi }{2}\\
		f( -x) & x\in \left( -\frac{\pi }{2} ,0\right)\\
		0 & x=0\\
		f( x) & x\in \left( 0,\frac{\pi }{2}\right)\\
		0 & x=\frac{\pi }{2}\\
		-f( \pi -x) & x\in \left(\frac{\pi }{2} ,\pi \right)
	\end{cases} ,x\in ( -\pi ,\pi )
\end{equation*}


有 $\displaystyle g( x)$ 是偶函数,同时因为$\displaystyle f( x)$ 为 $\displaystyle \left( 0,\frac{\pi }{2}\right)$ 上的 Riemann 可积或绝对可积函数,$\displaystyle g( x)$ 也是 \ $\displaystyle \left( 0,\frac{\pi }{2}\right)$ 上的 Riemann 可积或绝对可积函数,记 $\displaystyle G( x) =1+\int _{0}^{x} g( t)\mathrm{d} t,h( x) =G'( x)$。

首先有 $\displaystyle g( x)$ 的 Fourier 级数仅仅 $\displaystyle \cos$ 前面有系数,于是 $\displaystyle \int _{-\pi }^{\pi } g( x)\cos nx\mathrm{d} x=2\int _{0}^{\pi } g( x)\cos nx\mathrm{d} x$。

对于 $\displaystyle \int _{0}^{\pi } h( x)\cos nx\mathrm{d} x$,有:
\begin{align*}
	\int _{0}^{\pi } h( x)\cos nx\mathrm{d} x & =\int _{0}^{\pi }\cos nx\mathrm{d} G( x)\\
	& =( G( x)\cos nx | _{0}^{\pi }) -\frac{1}{n}\int _{0}^{\pi } G( x)\mathrm{d}\cos nx\\
	& =\frac{n-1}{n}( G( x)\cos nx | _{0}^{\pi }) +\frac{1}{n}\int _{0}^{\pi } h( x)\cos nx\mathrm{d} x\\
	& =G( x)\cos nx | _{0}^{\pi }\\
	& =G( x)\cos nx-G( 0)\\
	& =\begin{cases}
		-G( \pi ) -G( 0) & n=2k-1\\
		G( \pi ) -G( 0) & n=2k
	\end{cases} ,k\in N^{*}\\
	& =\begin{cases}
		-G( \pi ) -G( 0) & n=2k-1\\
		0 & n=2k
	\end{cases} ,k\in N^{*}
\end{align*}


接下来我们证明 $\displaystyle \int _{0}^{\pi } g( x)\cos nx\mathrm{d} x=\int _{0}^{\pi } h( x)\cos nx\mathrm{d} x$。

考虑 $\displaystyle \int _{0}^{\pi } g( x)\cos nx\mathrm{d} x-\int _{0}^{\pi } h( x)\cos nx\mathrm{d} x=\int _{0}^{\pi }( g( x) -h( x))\cos nx\mathrm{d} x$,因为 $\displaystyle g( x)$ 可积,由勒贝格引理,$\displaystyle g( x)$ 只有有限个间断点,于是 $\displaystyle ( g( x) -h( x))\cos nx$ 只有有限个点 $\displaystyle \neq 0$,进而 $\displaystyle \int _{0}^{\pi }( g( x) -h( x))\cos nx\mathrm{d} x$ 可以无限小 $\displaystyle =0$。

于是有 $\displaystyle \int _{0}^{\pi } g( x)\cos nx\mathrm{d} x=\begin{cases}
	-G( \pi ) -G( 0) & n=2k-1\\
	0 & n=2k
\end{cases} ,k\in N^{*}$,于是对于 $\displaystyle f( x)$ 的延拓 $\displaystyle g( x)$:


\begin{equation*}
	g( x) \sim \sum _{n=1}^{+\infty }\frac{-G( \pi ) -G( 0)}{\pi }\cos( 2n-1) x
\end{equation*}
\qed 
\ifx\allfiles\undefined
\end{document}
\fi