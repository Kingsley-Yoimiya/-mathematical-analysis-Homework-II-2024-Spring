%-*-    coding: UTF-8   -*-
% !TEX program = xelatex
\ifx\allfiles\undefined
%\special{dvipdfmx:config z 0}% 取消压缩,加快编译速度
\documentclass[UTF-8]{ctexart}
\usepackage{graphicx}
\usepackage{subfigure}
\usepackage{xcolor}
\usepackage{amsmath}
\usepackage{amssymb}
\usepackage{tabularx}
\usepackage{amssymb}
\usepackage{amsthm}
%\usepackage[usenames,dvipsnames]{color}
\usepackage{hyperref}
\hypersetup{
	colorlinks=true,
	linkcolor=black,
	filecolor=black,      
	urlcolor=red,
	citecolor=black,
}
\usepackage{geometry}
\geometry{a4paper,centering,scale=0.80}
\usepackage[format=hang,font=small,textfont=it]{caption}
\usepackage[nottoc]{tocbibind}
\usepackage{algorithm}  
\usepackage{algorithmicx}  
\usepackage{algpseudocode}
\usepackage{prettyref}
\usepackage{framed}
\setlength{\parindent}{2em}
\usepackage{indentfirst}
\usepackage[framemethod=TikZ]{mdframed}
\newcounter{ques}[section]
\renewcommand{\theques}{\arabic{section}.\arabic{ques}}
\newcommand{\setParDis}{\setlength {\parskip} {0.3cm} }
\newcommand{\setParDef}{\setlength {\parskip} {0pt} }
\setParDis% 调整这一个subsection的段落间距
%\setParDef%恢复间距

\newenvironment{ques}[1][]{
	\refstepcounter{ques}
	\mdfsetup{
		frametitle={
			\tikz[baseline=(current bounding box.east), outer sep=0pt]
			\node[anchor=east,rectangle,fill=blue!20]
			{\strut Problem~\theques\ifstrempty{#1}{}{:~#1}};},
		innertopmargin=10pt,linecolor=blue!20,
		linewidth=2pt,topline=true,
		frametitleaboveskip=\dimexpr-\ht\strutbox\relax
	}
	\begin{mdframed}[]\relax
}{\end{mdframed}}

\newcounter{Thm}[section]
\renewcommand{\theThm}{\arabic{section}.\arabic{Thm}}
\newenvironment{Thm}[1][]{
	\refstepcounter{Thm}
	\mdfsetup{
		frametitle={
			\tikz[baseline=(current bounding box.east), outer sep=0pt]
			\node[anchor=east,rectangle,fill=blue!20]
			{\strut Theorem~\theThm\ifstrempty{#1}{}{:~#1}};},
		innertopmargin=10pt,linecolor=blue!20,
		linewidth=2pt,topline=true,
		frametitleaboveskip=\dimexpr-\ht\strutbox\relax
	}
	\begin{mdframed}[]\relax
}{\end{mdframed}}

\newcounter{Defi}[section]
\renewcommand{\theDefi}{\arabic{section}.\arabic{Defi}}
\newenvironment{Defi}[1][]{
	\refstepcounter{Defi}
	\mdfsetup{
		frametitle={
			\tikz[baseline=(current bounding box.east), outer sep=0pt]
			\node[anchor=east,rectangle,fill=blue!20]
			{\strut Definition~\theDefi\ifstrempty{#1}{}{:~#1}};},
		innertopmargin=10pt,linecolor=blue!20,
		linewidth=2pt,topline=true,
		frametitleaboveskip=\dimexpr-\ht\strutbox\relax
	}
	\begin{mdframed}[]\relax
}{\end{mdframed}}

\newrefformat{qlt}{\underline{性质 \ref{#1}}}
\newcommand{\tpf}[2]{\begin{ques}[#1]{\kaishu #2}\end{ques}}
\newcommand{\pf}[1]{\begin{ques}{\kaishu #1}\end{ques}}
\newcommand{\tthm}[2]{\begin{Thm}[#1]{\kaishu #2}\end{Thm}}
\newcommand{\thm}[1]{\begin{Thm}{\kaishu #1}\end{Thm}}
\newcommand{\tdefi}[2]{\begin{Defi}[#1]{\kaishu #2}\end{Defi}}
\newcommand{\defi}[1]{\begin{Defi}{\kaishu #1}\end{Defi}}
\newcommand{\opf}[1]{{\kaishu{#1}}}
\title{数学分析 I 作业(2024. Spring)}
\author{\texttt{As-The-Wind}}

\date{2024 年 2 月 19 日 $\rightarrow$ \today}

\date{}
\author{尹锦润}
\begin{document}
\maketitle
\fi

\section{2024.5.6 作业}
\begin{ques}
	设数项级数 $\displaystyle \sum _{n=1}^{+\infty }\frac{a_{n}}{n^{x_{0}}}( x_{0} \in \mathbb{R})$ 收敛,证明 $\displaystyle \lim _{x\rightarrow x_{0} +0}\sum _{n=1}^{+\infty }\frac{a_{n}}{n^{x}} =\sum _{n=1}^{+\infty }\frac{a_{n}}{n^{x_{0}}}$。
\end{ques}



因为 $\displaystyle \sum _{n=1}^{+\infty }\frac{a_{n}}{n^{x_{0}}}$ 收敛,因为该函数和 $\displaystyle x$ 无关,因此是一致收敛的,

同时任取 $\displaystyle \delta  >0$,有 $\displaystyle \frac{1}{n^{x-x_{0}}} ,\forall x\in ( x_{0} ,\delta )$ 关于 $\displaystyle n$ 单调,(需证明一致有界)并且 $\displaystyle \forall x\in ( x_{0} ,\delta ) ,\left| \frac{1}{n^{x-x_{0}}}\right| \leqslant 1$,因此是一致有界的。

根据 Abel 判别法,$\displaystyle \sum _{n=1}^{+\infty }\frac{a_{n}}{n^{x}}$ 在 $\displaystyle x\in ( x_{0} ,x_{0} +\delta )$ 上是一致收敛的,因此在 $\displaystyle x_{0} +$ 处局部一致收敛,而 $\displaystyle \frac{a_{n}}{n^{x}}$ 是连续的,因此,$\displaystyle \lim _{x\rightarrow x_{0} +0}\sum _{n=1}^{+\infty }\frac{a_{n}}{n^{x}} =\sum _{n=1}^{+\infty }\lim _{x\rightarrow x_{0} +0}\frac{a_{n}}{n^{x}} =\sum _{n=1}^{+\infty }\frac{a_{n}}{n^{x_{0}}}$。\qed 





\begin{ques}
	证明 $\displaystyle f( x) =\sum _{n=1}^{+\infty }\frac{( -1)^{n}\sin\left(\frac{\pi }{4} +\frac{x}{n}\right)}{\sqrt{n}} \in C( -\infty ,+\infty )$。
\end{ques}





对于 $\displaystyle x\in [ a,b] \subset ( -\infty ,+\infty )$,有

$\displaystyle \frac{1}{\sqrt{n}}$ 关于 $\displaystyle n$ 单调且 $\displaystyle \rightrightarrows 0$。

对于 $\displaystyle \sum _{n=1}^{+\infty }( -1)^{n}\sin\left(\frac{\pi }{4} +\frac{x}{n}\right)$,记 $\displaystyle t=\max\{|a|,|b|\}$,有 


\begin{align*}
	\left| \sum _{n=1}^{+\infty }( -1)^{n}\sin\left(\frac{\pi }{4} +\frac{x}{n}\right)\right|  & =\left| \sum _{n=1}^{10t}( -1)^{n}\sin\left(\frac{\pi }{4} +\frac{x}{n}\right) +\sum _{n=10t+1}^{+\infty }( -1)^{n}\sin\left(\frac{\pi }{4} +\frac{x}{n}\right)\right|  & \\
	& \leqslant \left| 10t+\sum _{n=10t+1}^{+\infty }( -1)^{n}\sin\left(\frac{\pi }{4} +\frac{x}{n}\right)\right|  & ( 1)\\
	& \leqslant |10t+1| & 
\end{align*}
(1):因为 $\displaystyle \sin\left(\frac{\pi }{4} +\frac{x}{n}\right)$ 在 $\displaystyle n\geqslant 10t$ 时关于 $\displaystyle n$ 单调递减,因此
\begin{equation*}
	\left| \sum _{n=10t+1}^{+\infty }( -1)^{n}\sin\left(\frac{\pi }{4} +\frac{x}{n}\right)\right| \leqslant \left| \sin\left(\frac{\pi }{4} +\frac{x}{10t}\right)\right| \leqslant 1
\end{equation*}


因此 $\displaystyle \sum _{n=1}^{+\infty }( -1)^{n}\sin\left(\frac{\pi }{4} +\frac{x}{n}\right)$ 在 $\displaystyle [ a,b]$ 上是一致有界的。

由狄利克雷法则,因此 $\displaystyle f( x) =\sum _{n=1}^{+\infty }\frac{( -1)^{n}\sin\left(\frac{\pi }{4} +\frac{x}{n}\right)}{\sqrt{n}}$ 在 $\displaystyle [ a,b] \subset ( -\infty ,+\infty )$ 上是一致收敛的。

因此 $\displaystyle f( x) =\sum _{n=1}^{+\infty }\frac{( -1)^{n}\sin\left(\frac{\pi }{4} +\frac{x}{n}\right)}{\sqrt{n}}$ 在 $\displaystyle ( -\infty ,+\infty )$ 上是内闭一致收敛的。

而$\displaystyle \frac{( -1)^{n}\sin\left(\frac{\pi }{4} +\frac{x}{n}\right)}{\sqrt{n}}$ 关于 $\displaystyle x$ 是连续函数,因此 $\displaystyle f( x) =\sum _{n=1}^{+\infty }\frac{( -1)^{n}\sin\left(\frac{\pi }{4} +\frac{x}{n}\right)}{\sqrt{n}}$ 连续。\qed 





\begin{ques}
	求极限 $\displaystyle \lim _{x\rightarrow 3}\sum _{n=1}^{+\infty }\frac{1}{2^{n}}\left(\frac{x-4}{x-2}\right)^{n}$。
\end{ques}



取 $\displaystyle \delta =0.2,q=\max_{x\in ( 3-\delta ,3+\delta )}\left| \frac{x-4}{2( x-2)}\right| \leqslant \frac{3}{4}$,则 $\displaystyle \sum _{n=1}^{+\infty }\left| \frac{1}{2^{n}}\left(\frac{x-4}{x-2}\right)^{n}\right| \leqslant \sum _{n=1}^{+\infty } q^{n}$。

而 $\displaystyle \sum _{n=1}^{+\infty } q^{n}$ 是收敛的,由 M 判别法,$\displaystyle \sum _{n=1}^{+\infty }\frac{1}{2^{n}}\left(\frac{x-4}{x-2}\right)^{n}$ 在 $\displaystyle ( 3-\delta ,3+\delta )$ 上是绝对一致收敛的。

因为 $\displaystyle \sum _{n=1}^{+\infty }\frac{1}{2^{n}}\left(\frac{x-4}{x-2}\right)^{n}$是一致收敛的,同时对于每个 $\displaystyle n$,$\displaystyle \frac{1}{2^{n}}\left(\frac{x-4}{x-2}\right)^{n}$ 是连续的,因此有
\begin{align*}
	\lim _{x\rightarrow 3}\sum _{n=1}^{+\infty }\frac{1}{2^{n}}\left(\frac{x-4}{x-2}\right)^{n} & =\sum _{n=1}^{+\infty }\frac{1}{2^{n}}\lim _{x\rightarrow 3}\left(\frac{x-4}{x-2}\right)^{n}\\
	& =\sum _{n=1}^{+\infty }\frac{1}{2^{n}}( -1)^{n}\\
	& =-\sum _{n=1}^{+\infty }\frac{1}{4^{n}}\\
	& =-\frac{1}{3}
\end{align*}
\qed 





\begin{ques}
	求极限 $\displaystyle \lim _{x\rightarrow +\infty }\sum _{n=1}^{+\infty }\frac{1}{n}\left(\frac{1-x}{x}\right)^{n}$。
\end{ques}



对于 $\displaystyle x\in [ a,b] ,2\leqslant a< b< +\infty $,令 $\displaystyle q=\max_{x\in [ a,b]}\left| \left(\frac{1}{x} -1\right)\right| =\left| \frac{1}{b} -1\right| < 1$,则


\begin{align*}
	\left| \frac{1}{n}\left(\frac{1-x}{x}\right)^{n}\right|  & \leqslant q^{n}\\
	\sum _{n=1}^{+\infty }\left| \frac{1}{n}\left(\frac{1-x}{x}\right)^{n}\right|  & \leqslant \sum _{n=1}^{+\infty } q^{n}
\end{align*}


因为 $\displaystyle \sum _{n=1}^{+\infty } q^{n}$ 是收敛的,于是 由 M 判别法 $\displaystyle \sum _{n=1}^{+\infty }\frac{1}{n}\left(\frac{1-x}{x}\right)^{n}$ 在 $\displaystyle [ a,b]$ 绝对一致收敛,进而 $\displaystyle \sum _{n=1}^{+\infty }\frac{1}{n}\left(\frac{1-x}{x}\right)^{n}$ 在 $\displaystyle [ 2,+\infty )$ 上内闭一致收敛。

而 $\displaystyle \left(\frac{1-x}{x}\right)^{n}$ 是连续的,因此有:


\begin{align*}
	\lim _{x\rightarrow +\infty }\sum _{n=1}^{+\infty }\frac{1}{n}\left(\frac{1-x}{x}\right)^{n} & =\sum _{n=1}^{+\infty }\lim _{x\rightarrow +\infty }\frac{1}{n}\left(\frac{1-x}{x}\right)^{n}\\
	& =\sum _{n=1}^{+\infty }\frac{1}{n}( -1)^{n}\\
	& =-\sum _{n=1}^{+\infty }\frac{1}{n} +\sum _{n=1}^{+\infty }\frac{1}{2n}\\
	& =-\ln 2
\end{align*}
\qed 




\ifx\allfiles\undefined
\end{document}
\fi