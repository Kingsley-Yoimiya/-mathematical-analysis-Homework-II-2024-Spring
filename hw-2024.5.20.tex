%-*-    coding: UTF-8   -*-
% !TEX program = xelatex
\ifx\allfiles\undefined
%\special{dvipdfmx:config z 0}% 取消压缩,加快编译速度
\documentclass[UTF-8]{ctexart}
\usepackage{graphicx}
\usepackage{subfigure}
\usepackage{xcolor}
\usepackage{amsmath}
\usepackage{amssymb}
\usepackage{tabularx}
\usepackage{amssymb}
\usepackage{amsthm}
%\usepackage[usenames,dvipsnames]{color}
\usepackage{hyperref}
\hypersetup{
	colorlinks=true,
	linkcolor=black,
	filecolor=black,      
	urlcolor=red,
	citecolor=black,
}
\usepackage{geometry}
\geometry{a4paper,centering,scale=0.80}
\usepackage[format=hang,font=small,textfont=it]{caption}
\usepackage[nottoc]{tocbibind}
\usepackage{algorithm}  
\usepackage{algorithmicx}  
\usepackage{algpseudocode}
\usepackage{prettyref}
\usepackage{framed}
\setlength{\parindent}{2em}
\usepackage{indentfirst}
\usepackage[framemethod=TikZ]{mdframed}
\newcounter{ques}[section]
\renewcommand{\theques}{\arabic{section}.\arabic{ques}}
\newcommand{\setParDis}{\setlength {\parskip} {0.3cm} }
\newcommand{\setParDef}{\setlength {\parskip} {0pt} }
\setParDis% 调整这一个subsection的段落间距
%\setParDef%恢复间距

\newenvironment{ques}[1][]{
	\refstepcounter{ques}
	\mdfsetup{
		frametitle={
			\tikz[baseline=(current bounding box.east), outer sep=0pt]
			\node[anchor=east,rectangle,fill=blue!20]
			{\strut Problem~\theques\ifstrempty{#1}{}{:~#1}};},
		innertopmargin=10pt,linecolor=blue!20,
		linewidth=2pt,topline=true,
		frametitleaboveskip=\dimexpr-\ht\strutbox\relax
	}
	\begin{mdframed}[]\relax
}{\end{mdframed}}

\newcounter{Thm}[section]
\renewcommand{\theThm}{\arabic{section}.\arabic{Thm}}
\newenvironment{Thm}[1][]{
	\refstepcounter{Thm}
	\mdfsetup{
		frametitle={
			\tikz[baseline=(current bounding box.east), outer sep=0pt]
			\node[anchor=east,rectangle,fill=blue!20]
			{\strut Theorem~\theThm\ifstrempty{#1}{}{:~#1}};},
		innertopmargin=10pt,linecolor=blue!20,
		linewidth=2pt,topline=true,
		frametitleaboveskip=\dimexpr-\ht\strutbox\relax
	}
	\begin{mdframed}[]\relax
}{\end{mdframed}}

\newcounter{Defi}[section]
\renewcommand{\theDefi}{\arabic{section}.\arabic{Defi}}
\newenvironment{Defi}[1][]{
	\refstepcounter{Defi}
	\mdfsetup{
		frametitle={
			\tikz[baseline=(current bounding box.east), outer sep=0pt]
			\node[anchor=east,rectangle,fill=blue!20]
			{\strut Definition~\theDefi\ifstrempty{#1}{}{:~#1}};},
		innertopmargin=10pt,linecolor=blue!20,
		linewidth=2pt,topline=true,
		frametitleaboveskip=\dimexpr-\ht\strutbox\relax
	}
	\begin{mdframed}[]\relax
}{\end{mdframed}}

\newrefformat{qlt}{\underline{性质 \ref{#1}}}
\newcommand{\tpf}[2]{\begin{ques}[#1]{\kaishu #2}\end{ques}}
\newcommand{\pf}[1]{\begin{ques}{\kaishu #1}\end{ques}}
\newcommand{\tthm}[2]{\begin{Thm}[#1]{\kaishu #2}\end{Thm}}
\newcommand{\thm}[1]{\begin{Thm}{\kaishu #1}\end{Thm}}
\newcommand{\tdefi}[2]{\begin{Defi}[#1]{\kaishu #2}\end{Defi}}
\newcommand{\defi}[1]{\begin{Defi}{\kaishu #1}\end{Defi}}
\newcommand{\opf}[1]{{\kaishu{#1}}}
\title{数学分析 I 作业(2024. Spring)}
\author{\texttt{As-The-Wind}}

\date{2024 年 2 月 19 日 $\rightarrow$ \today}

\date{}
\author{尹锦润}
\begin{document}
\maketitle
\fi

\section{2024.5.20 作业}
\begin{ques}
	求函数 $\displaystyle \cos( \alpha +\beta x)$ 的麦克劳林展式,其中 $\displaystyle \alpha ,\beta $ 为非零常数。
\end{ques}
\begin{align*}
	\cos( \alpha +\beta x) & =\cos \alpha \cos \beta x-\sin \alpha \sin \beta x\\
	& =\sum _{n=0}^{+\infty } c_{n}\frac{\beta ^{n} x^{n}}{n!}\\
	c_{n} & =\begin{cases}
		( -1)^{k}\cos \alpha  & n=2k\\
		( -1)^{k+1}\sin \alpha  & n=2k+1
	\end{cases}
\end{align*}
\qed 





\begin{ques}
	求函数 $\displaystyle \arctan\frac{2( 1-x)}{1+4x}$的麦克劳林展式。
\end{ques}
\begin{align*}
	\arctan\frac{2( 1-x)}{1+4x} & =\arctan 2-\arctan 2x\\
	& =\arctan 2-\sum _{n=0}^{+\infty }\frac{( -1)^{n} 2^{2n+1} x^{2n+1}}{2n+1}
\end{align*}\qed 



\begin{ques}
	求函数 $\displaystyle f( x) =\sum _{n=1}^{+\infty }\frac{1}{n( n+1)}\left(\frac{x+1}{2}\right)^{n}$ 的麦克劳林展式。
\end{ques}
\begin{align*}
	f( x) & =\sum _{n=1}^{+\infty }\frac{1}{n( n+1)}\left(\frac{x+1}{2}\right)^{n}\\
	& =\sum _{n=1}^{+\infty }\frac{1}{n}\left(\frac{x+1}{2}\right)^{n} -\sum _{n=1}^{+\infty }\frac{1}{n+1}\left(\frac{x+1}{2}\right)^{n}\\
	& =-\ln\left( 1-\frac{x+1}{2}\right) -\frac{-\ln\left( 1-\frac{x+1}{2}\right)}{\frac{x+1}{2}} +1\\
	& =-\ln( 1-x) +\ln 2+1-2\frac{\ln( 1-x)}{x+1} -2\frac{1}{x+1}\ln 2\\
	& =\sum _{n=1}^{+\infty }\frac{1}{n} x^{n} +\ln 2+1-2\sum _{n=1}^{+\infty }\sum _{k=1}^{n}\frac{( -1)^{n-k}}{k} x^{n} -( 2\ln 2)\sum _{n=0}^{+\infty }( -1)^{n} x^{n}\\
	& =\sum _{n=1}^{+\infty }\left(\frac{1}{n} -2\sum _{k=1}^{n}\frac{( -1)^{n-k}}{k} +( -1)^{n+1} 2\ln 2\right) x^{n} +1-\ln 2
\end{align*}
\qed



\begin{ques}
	设在闭区间 $\displaystyle [ a,b]$ 上的函数 $\displaystyle f( x)$ 的各阶导数存在且非负,证明 $\displaystyle f( x) =\sum _{n=0}^{+\infty }\frac{f^{( n)}( a)}{n!}( x-a)^{n} ,\forall x\in [ a,b]$。
\end{ques}

使用泰勒展开的积分余项:
\begin{gather*}
	f( x) =\sum _{k=0}^{n}\frac{f^{( k)}( a)}{k!}( x-a)^{k} +\int _{a}^{x}\frac{f^{( n+1)}( t)}{( n+1) !}( x-t)^{n}\mathrm{d} t\\
	f( x) =\sum _{k=0}^{n}\frac{f^{( k)}( a)}{k!}( x-a)^{k} +\frac{( x-a) !}{( n+1) !}\int _{a}^{x} f^{( n+1)}( t)\left(\frac{x-t}{x-a}\right)^{n}\mathrm{d} t\\
	R_{n}( x) =\frac{( x-a) !}{( n+1) !}\int _{a}^{x} f^{( n+1)}( t)\left(\frac{x-t}{x-a}\right)^{n}\mathrm{d} t
\end{gather*}
\begin{equation*}
	\frac{x-t}{x-a} \leqslant \frac{b-x+x-t}{b-x+x-a} =\frac{b-t}{b-a}
\end{equation*}


又因为 $\displaystyle f( x)$ 的各阶导数存在且非负,于是 $\displaystyle f^{( n+1)}( t)$ 非负且单调递增,有:

\begin{align*}
	0\leqslant R_{n}( x) & =\frac{( x-a) !}{( n+1) !}\int _{a}^{x} f^{( n+1)}( t)\left(\frac{x-t}{x-a}\right)^{n}\mathrm{d} t\\
	& \leqslant \frac{( x-a) !}{( n+1) !}\int _{a}^{x} f^{( n+1)}( b)\left(\frac{b-t}{b-a}\right)^{n}\mathrm{d} t\\
	& =\frac{( x-a) !}{( b-a) !} R_{n}( b)
\end{align*}

而
\begin{equation*}
	R_{n}( b) =R_{n-1}( b) -\frac{f^{( n)}( a)}{n!}( b-a)^{n} \leqslant R_{n-1}( b) \leqslant R_{n-2}( b) \leqslant \cdots \leqslant R_{0}( b) =f( b) -f( a) =M
\end{equation*}


进而


\begin{equation*}
	R_{n}( x) \leqslant \frac{( x-a) !}{( b-a) !} M\rightarrow 0\left( n\rightarrow +\infty ,x\neq b\right)
\end{equation*}


当 $\displaystyle x=b$ 时,因为 $\displaystyle f( x) =\sum _{n=0}^{+\infty }\frac{f^{( n)}( a)}{n!}( x-a)^{n} ,\forall x\in [ a,b)$,由阿贝尔定理推论 $\displaystyle f( b) =\lim _{x\rightarrow b^{-}}\sum _{n=0}^{+\infty }\frac{f^{( n)}( a)}{n!}( x-a)^{n} =\sum _{n=0}^{+\infty }\frac{f^{( n)}( a)}{n!}( b-a)^{n}$。\qed 





\begin{ques}
	设非常数函数 $\displaystyle f( x)$ 在 $\displaystyle ( a,b)$ 内每一点都可以展成幂级数。试证明 $\displaystyle f( x)$ 的零点集在 $\displaystyle ( a,b)$ 内没有聚点。
\end{ques}



对于 $\displaystyle f( x)$ 的任一零点 $\displaystyle x_{0}$,在 $\displaystyle x_{0}$ 展开为幂级数 $\displaystyle f( x) =\sum _{n=1}^{+\infty } a_{n}( x-x_{0})^{n} =( x-x_{0})^{m} g( x)$,取最小的 $\displaystyle m$ 使得 $\displaystyle g( x_{0}) \neq 0$(因为 $\displaystyle f( x)$ 非常数,一定取得到。

$\displaystyle g( x)$ 也是幂级数,同时 $\displaystyle g( x) ,( x-x_{0})^{m}$ 都是连续的,因此 $\displaystyle \exists \delta  >0\ s.t.\ f( x) \neq 0,\forall x\in ( x_{0} -\delta ,x_{0} +\delta )$。因此在任意零点附近都没有别的零点,$\displaystyle f( x)$ 的零点集在 $\displaystyle ( a,b)$ 内没有聚点。\qed 



\ifx\allfiles\undefined
\end{document}
\fi