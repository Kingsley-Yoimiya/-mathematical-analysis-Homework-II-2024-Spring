%-*-    coding: UTF-8   -*-
% !TEX program = xelatex
\ifx\allfiles\undefined
%\special{dvipdfmx:config z 0}% 取消压缩,加快编译速度
\documentclass[UTF-8]{ctexart}
\usepackage{graphicx}
\usepackage{subfigure}
\usepackage{xcolor}
\usepackage{amsmath}
\usepackage{amssymb}
\usepackage{tabularx}
\usepackage{amssymb}
\usepackage{amsthm}
%\usepackage[usenames,dvipsnames]{color}
\usepackage{hyperref}
\hypersetup{
	colorlinks=true,
	linkcolor=black,
	filecolor=black,      
	urlcolor=red,
	citecolor=black,
}
\usepackage{geometry}
\geometry{a4paper,centering,scale=0.80}
\usepackage[format=hang,font=small,textfont=it]{caption}
\usepackage[nottoc]{tocbibind}
\usepackage{algorithm}  
\usepackage{algorithmicx}  
\usepackage{algpseudocode}
\usepackage{prettyref}
\usepackage{framed}
\setlength{\parindent}{2em}
\usepackage{indentfirst}
\usepackage[framemethod=TikZ]{mdframed}
\newcounter{ques}[section]
\renewcommand{\theques}{\arabic{section}.\arabic{ques}}
\newcommand{\setParDis}{\setlength {\parskip} {0.3cm} }
\newcommand{\setParDef}{\setlength {\parskip} {0pt} }
\setParDis% 调整这一个subsection的段落间距
%\setParDef%恢复间距

\newenvironment{ques}[1][]{
	\refstepcounter{ques}
	\mdfsetup{
		frametitle={
			\tikz[baseline=(current bounding box.east), outer sep=0pt]
			\node[anchor=east,rectangle,fill=blue!20]
			{\strut Problem~\theques\ifstrempty{#1}{}{:~#1}};},
		innertopmargin=10pt,linecolor=blue!20,
		linewidth=2pt,topline=true,
		frametitleaboveskip=\dimexpr-\ht\strutbox\relax
	}
	\begin{mdframed}[]\relax
}{\end{mdframed}}

\newcounter{Thm}[section]
\renewcommand{\theThm}{\arabic{section}.\arabic{Thm}}
\newenvironment{Thm}[1][]{
	\refstepcounter{Thm}
	\mdfsetup{
		frametitle={
			\tikz[baseline=(current bounding box.east), outer sep=0pt]
			\node[anchor=east,rectangle,fill=blue!20]
			{\strut Theorem~\theThm\ifstrempty{#1}{}{:~#1}};},
		innertopmargin=10pt,linecolor=blue!20,
		linewidth=2pt,topline=true,
		frametitleaboveskip=\dimexpr-\ht\strutbox\relax
	}
	\begin{mdframed}[]\relax
}{\end{mdframed}}

\newcounter{Defi}[section]
\renewcommand{\theDefi}{\arabic{section}.\arabic{Defi}}
\newenvironment{Defi}[1][]{
	\refstepcounter{Defi}
	\mdfsetup{
		frametitle={
			\tikz[baseline=(current bounding box.east), outer sep=0pt]
			\node[anchor=east,rectangle,fill=blue!20]
			{\strut Definition~\theDefi\ifstrempty{#1}{}{:~#1}};},
		innertopmargin=10pt,linecolor=blue!20,
		linewidth=2pt,topline=true,
		frametitleaboveskip=\dimexpr-\ht\strutbox\relax
	}
	\begin{mdframed}[]\relax
}{\end{mdframed}}

\newrefformat{qlt}{\underline{性质 \ref{#1}}}
\newcommand{\tpf}[2]{\begin{ques}[#1]{\kaishu #2}\end{ques}}
\newcommand{\pf}[1]{\begin{ques}{\kaishu #1}\end{ques}}
\newcommand{\tthm}[2]{\begin{Thm}[#1]{\kaishu #2}\end{Thm}}
\newcommand{\thm}[1]{\begin{Thm}{\kaishu #1}\end{Thm}}
\newcommand{\tdefi}[2]{\begin{Defi}[#1]{\kaishu #2}\end{Defi}}
\newcommand{\defi}[1]{\begin{Defi}{\kaishu #1}\end{Defi}}
\newcommand{\opf}[1]{{\kaishu{#1}}}
\title{数学分析 I 作业(2024. Spring)}
\author{\texttt{As-The-Wind}}

\date{2024 年 2 月 19 日 $\rightarrow$ \today}

\date{}
\author{尹锦润}
\begin{document}
\maketitle
\fi

\section{2024.4.3 作业}
\begin{ques}
	讨论级数的敛散性:$\displaystyle \sum _{n=1}^{+\infty }\frac{( -1)^{\left[\sqrt{n}\right]}}{n^{p}}( p >0)$。
\end{ques}




\begin{align*}
	\sum _{n=1}^{+\infty }\frac{( -1)^{\left[\sqrt{n}\right]}}{n^{p}} & =\sum _{k=1}^{+\infty }( -1)^{k}\left(\frac{1}{k^{2p}} +\frac{1}{\left( k^{2} +1\right)^{p}} +\cdots +\frac{1}{\left( k^{2} +2k\right)^{p}}\right)
\end{align*}


对于
\begin{gather*}
	\left(\frac{1}{k^{2p}} +\frac{1}{\left( k^{2} +1\right)^{p}} +\cdots +\frac{1}{\left( k^{2} +2k\right)^{p}}\right) \geqslant \frac{2k+1}{( k+1)^{2p}}\\
	\lim _{k\rightarrow +\infty }\frac{2k+1}{( k+1)^{2p}} =\lim _{k\rightarrow +\infty }\frac{2}{k^{2p-1}}
\end{gather*}


当 $\displaystyle 2p-1\leqslant 0\ i.e.\ p\leqslant \frac{1}{2}$ 时,$\displaystyle \lim _{k\rightarrow +\infty }\frac{2}{k^{2p-1}} \neq 0$,进而原级数发散。

当 $\displaystyle p >\frac{1}{2}$ 时,令 $\displaystyle t=\frac{p+\frac{1}{2}}{2}$,则
\begin{equation*}
	\sum _{n=1}^{+\infty }\frac{( -1)^{\left[\sqrt{n}\right]}}{n^{t}}
\end{equation*}


绝对收敛,进而有界,而 $\displaystyle \frac{1}{n^{p-t}}$ 单调且 $\displaystyle \rightarrow 0\left( n\rightarrow +\infty \right)$,由狄利克雷判别法,级数收敛。

综上,$\displaystyle p\leqslant \frac{1}{2}$ 时发散,$\displaystyle p >\frac{1}{2}$ 时收敛。\qed 







\begin{ques}
	讨论级数的敛散性:$\displaystyle \sum _{n=1}^{+\infty }( -1)^{n}\left( 1-n\ln\frac{n+1}{n}\right)$。
\end{ques}


\begin{align*}
	\sum _{n=1}^{+\infty }( -1)^{n}\left( 1-n\ln\frac{n+1}{n}\right) & =\sum _{n=1}^{+\infty }( -1)^{n}\left( 1-n\left(\frac{1}{n} -\frac{1}{2n^{2}} +\frac{1}{12n^{3}} +O\left(\frac{1}{n^{4}}\right)\right)\right)\\
	& =\sum _{n=1}^{+\infty }( -1)^{n}\left(\frac{1}{2n} -\frac{1}{12n^{2}} +O\left(\frac{1}{n^{3}}\right)\right)
\end{align*}

令 $\displaystyle a_{n} =\frac{1}{2n} -\frac{1}{12n^{2}} +O\left(\frac{1}{n^{3}}\right)$,则
\begin{align*}
	a_{n} -a_{n+1} & =\frac{1}{n( 2n+2)} +O\left(\frac{1}{n^{3}}\right)  >0
\end{align*}

于是 $\displaystyle a_{n}$ 单调递减且趋向于 0,进而级数收敛。\qed 





\begin{ques}
	讨论级数的敛散性和绝对敛散性 $\displaystyle \sum _{n=2}^{+\infty }\frac{( -1)^{n}}{\left( n+( -1)^{n}\right)^{p}} ,p\in \mathbb{R}$ 是常数。
\end{ques}



对于绝对敛散性,其和 $\displaystyle \sum _{n=2}^{+\infty }\frac{1}{n^{p}}$ 敛散性相同,于是当 $\displaystyle p >1$ 是绝对收敛,$\displaystyle p\leqslant 1$ 时不绝对收敛。

对于敛散性,考虑


\begin{align*}
	\sum _{n=2}^{+\infty }\frac{( -1)^{n}}{\left( n+( -1)^{n}\right)^{p}} & =\sum _{k=1}^{+\infty }\left(\frac{1}{( 2k+1)^{p}} -\frac{1}{( 2k)^{p}}\right) & \\
	& =\sum _{k=1}^{+\infty }\frac{-p( 2k)^{p-1}}{( 2k)^{p}( 2k+1)^{p}} & ( 1)\\
	& =\sum _{k=1}^{+\infty }\frac{-p}{( 2k)^{p+1}} & 
\end{align*}

(1):定号,运用极限形式的比较判别法。



因此,当 $\displaystyle p+1 >1\ i.e.\ p >0$ 时收敛,$\displaystyle p\leqslant 0$ 时发散。

综上所述,$\displaystyle p\leqslant 0$ 时发散,$\displaystyle 0< p\leqslant 1$ 时条件收敛,$\displaystyle p >1$ 时绝对收敛。\qed 





\begin{ques}
	设级数 $\displaystyle \sum _{n=1}^{+\infty } a_{n}$ 的部分和序列 $\displaystyle S_{n} =\sum _{k=1}^{n} a_{k}$ 是有界序列,再设 $\displaystyle \sum _{n=1}^{+\infty }( b_{n} -b_{n+1})$ 绝对收敛并且 $\displaystyle \lim _{n\rightarrow \infty } b_{n} =0$,证明级数 $\displaystyle \sum _{n=1}^{+\infty } a_{n} b_{n}$ 收敛。
\end{ques}



根据 Abel 变换:
\begin{align*}
	\sum _{k=n}^{n+p} a_{n} b_{n} & =b_{n+p}( S_{n+p} -S_{n-1}) +\sum _{k=n}^{n+p-1}( b_{k} -b_{k+1})( S_{k} -S_{n-1})
\end{align*}

记 $\displaystyle S_{n}$ 界为 $\displaystyle M$,则 $\displaystyle \forall \varepsilon  >0,\exists N_{1} \ s.t.\ \forall n >N_{1} \ b_{n} < \frac{\varepsilon }{2M} ,\ \exists N_{2} \ s.t.\ \forall n >N_{2} \ |b_{n} -b_{n+1} |< \frac{\varepsilon }{2M}$,令 $\displaystyle N=\max( N_{1} ,N_{2})$,于是
\begin{align*}
	\left| \sum _{k=n}^{n+p} a_{n} b_{n}\right|  & \leqslant \varepsilon +\varepsilon =2\varepsilon 
\end{align*}

由柯西判别法可知级数收敛。\qed 
\ifx\allfiles\undefined
\end{document}
\fi