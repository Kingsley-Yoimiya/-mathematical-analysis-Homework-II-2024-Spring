%-*-    coding: UTF-8   -*-
% !TEX program = xelatex
\ifx\allfiles\undefined
%\special{dvipdfmx:config z 0}% 取消压缩,加快编译速度
\documentclass[UTF-8]{ctexart}
\usepackage{graphicx}
\usepackage{subfigure}
\usepackage{xcolor}
\usepackage{amsmath}
\usepackage{amssymb}
\usepackage{tabularx}
\usepackage{amssymb}
\usepackage{amsthm}
%\usepackage[usenames,dvipsnames]{color}
\usepackage{hyperref}
\hypersetup{
	colorlinks=true,
	linkcolor=black,
	filecolor=black,      
	urlcolor=red,
	citecolor=black,
}
\usepackage{geometry}
\geometry{a4paper,centering,scale=0.80}
\usepackage[format=hang,font=small,textfont=it]{caption}
\usepackage[nottoc]{tocbibind}
\usepackage{algorithm}  
\usepackage{algorithmicx}  
\usepackage{algpseudocode}
\usepackage{prettyref}
\usepackage{framed}
\setlength{\parindent}{2em}
\usepackage{indentfirst}
\usepackage[framemethod=TikZ]{mdframed}
\newcounter{ques}[section]
\renewcommand{\theques}{\arabic{section}.\arabic{ques}}
\newcommand{\setParDis}{\setlength {\parskip} {0.3cm} }
\newcommand{\setParDef}{\setlength {\parskip} {0pt} }
\setParDis% 调整这一个subsection的段落间距
%\setParDef%恢复间距

\newenvironment{ques}[1][]{
	\refstepcounter{ques}
	\mdfsetup{
		frametitle={
			\tikz[baseline=(current bounding box.east), outer sep=0pt]
			\node[anchor=east,rectangle,fill=blue!20]
			{\strut Problem~\theques\ifstrempty{#1}{}{:~#1}};},
		innertopmargin=10pt,linecolor=blue!20,
		linewidth=2pt,topline=true,
		frametitleaboveskip=\dimexpr-\ht\strutbox\relax
	}
	\begin{mdframed}[]\relax
}{\end{mdframed}}

\newcounter{Thm}[section]
\renewcommand{\theThm}{\arabic{section}.\arabic{Thm}}
\newenvironment{Thm}[1][]{
	\refstepcounter{Thm}
	\mdfsetup{
		frametitle={
			\tikz[baseline=(current bounding box.east), outer sep=0pt]
			\node[anchor=east,rectangle,fill=blue!20]
			{\strut Theorem~\theThm\ifstrempty{#1}{}{:~#1}};},
		innertopmargin=10pt,linecolor=blue!20,
		linewidth=2pt,topline=true,
		frametitleaboveskip=\dimexpr-\ht\strutbox\relax
	}
	\begin{mdframed}[]\relax
}{\end{mdframed}}

\newcounter{Defi}[section]
\renewcommand{\theDefi}{\arabic{section}.\arabic{Defi}}
\newenvironment{Defi}[1][]{
	\refstepcounter{Defi}
	\mdfsetup{
		frametitle={
			\tikz[baseline=(current bounding box.east), outer sep=0pt]
			\node[anchor=east,rectangle,fill=blue!20]
			{\strut Definition~\theDefi\ifstrempty{#1}{}{:~#1}};},
		innertopmargin=10pt,linecolor=blue!20,
		linewidth=2pt,topline=true,
		frametitleaboveskip=\dimexpr-\ht\strutbox\relax
	}
	\begin{mdframed}[]\relax
}{\end{mdframed}}

\newrefformat{qlt}{\underline{性质 \ref{#1}}}
\newcommand{\tpf}[2]{\begin{ques}[#1]{\kaishu #2}\end{ques}}
\newcommand{\pf}[1]{\begin{ques}{\kaishu #1}\end{ques}}
\newcommand{\tthm}[2]{\begin{Thm}[#1]{\kaishu #2}\end{Thm}}
\newcommand{\thm}[1]{\begin{Thm}{\kaishu #1}\end{Thm}}
\newcommand{\tdefi}[2]{\begin{Defi}[#1]{\kaishu #2}\end{Defi}}
\newcommand{\defi}[1]{\begin{Defi}{\kaishu #1}\end{Defi}}
\newcommand{\opf}[1]{{\kaishu{#1}}}
\title{数学分析 I 作业(2024. Spring)}
\author{\texttt{As-The-Wind}}

\date{2024 年 2 月 19 日 $\rightarrow$ \today}

\date{}
\author{尹锦润}
\begin{document}
\maketitle
\fi

\section{2024.2.28 作业}

\begin{ques}
	求极限 $\displaystyle \lim _{x\rightarrow 0}\frac{\int _{0}^{x}(\sin t)^{\alpha }\mathrm{d} t}{x^{1+\alpha }}( \alpha  >0)$。
\end{ques}




\begin{equation*}
	\begin{aligned}
		\lim _{x\rightarrow 0}\frac{\int _{0}^{x}(\sin t)^{\alpha }\mathrm{d} t}{x^{1+\alpha }} & =\lim _{x\rightarrow 0}\frac{(\sin x)^{\alpha }}{( 1+\alpha ) x^{\alpha }}\\
		& =\frac{1}{1+\alpha }\lim _{x\rightarrow 0}\frac{(\sin x)^{\alpha }}{x^{\alpha }}\\
		& =\frac{1}{1+\alpha }
	\end{aligned}
\end{equation*}




\begin{ques}
	设函数 $\displaystyle f( x) \in R[ a,b]$ 并且对 $\displaystyle \forall x\in [ a,b]$ 有 $\displaystyle f( x)  >0$。证明 $\displaystyle \int _{a}^{b} f( x)\mathrm{d} x >0$。
\end{ques}



对于任意 $\displaystyle [ a,b]$ 上的分割 $\displaystyle \Delta :a=x_{0} < x_{1} < x_{2} < \cdots < x_{n} =b$,记 $\displaystyle \xi _{i} \in [ x_{i-1} ,x_{i}]$,有


\begin{equation*}
	\sum _{i=1}^{n} f( \xi _{i}) \Delta x_{i}  >0
\end{equation*}

进而,$\displaystyle \int _{a}^{b} f( x)\mathrm{d} x=\lim _{\lambda ( \Delta )\rightarrow 0}\sum _{i=1}^{n} f( \xi _{i}) \Delta x_{i}  >0$。\qed 





\begin{ques}
	证明下面极限成立:\\(1)$\displaystyle \lim _{n\rightarrow +\infty }\int _{-1}^{1}\left( 1-x^{2}\right)^{n}\mathrm{d} x=0$;\\(2)设函数 $\displaystyle f( x) \in C[ -1,1] ,\lim _{n\rightarrow +\infty }\frac{\int _{-1}^{1} f( x)\left( 1-x^{2}\right)^{n}\mathrm{d} x}{\int _{-1}^{1}\left( 1-x^{2}\right)^{n}\mathrm{d} x} =f( 0)$。

\end{ques}


(1)

对于任意 $\displaystyle t >0$,$\displaystyle \forall |x| >t,\lim _{n\rightarrow +\infty }\left( 1-x^{2}\right)^{n} =0$,因此 $\displaystyle \lim _{n\rightarrow +\infty }\int _{-1}^{-t}\left( 1-x^{2}\right)^{n} =\lim _{n\rightarrow +\infty }\int _{t}^{1}\left( 1-x^{2}\right)^{n} =0$。

进而:


\begin{equation*}
	\lim _{n\rightarrow +\infty }\int _{-1}^{1}\left( 1-x^{2}\right)^{n}\mathrm{d} x=\lim _{t\rightarrow 0}\lim _{n\rightarrow +\infty }\int _{-1}^{-t}\left( 1-x^{2}\right)^{n} +\int _{t}^{1}\left( 1-x^{2}\right)^{n} +2t=0
\end{equation*}
\qed 

(2)

类似 (1),有$\displaystyle \lim _{n\rightarrow +\infty }\int _{-1}^{-t} f( x)\left( 1-x^{2}\right)^{n} =\lim _{n\rightarrow +\infty }\int _{t}^{1} f( x)\left( 1-x^{2}\right)^{n} =0,\forall t >0$。

于是


\begin{equation*}
	\lim _{n\rightarrow +\infty }\frac{\int _{-1}^{1} f( x)\left( 1-x^{2}\right)^{n}\mathrm{d} x}{\int _{-1}^{1}\left( 1-x^{2}\right)^{n}\mathrm{d} x} =\lim _{n\rightarrow +\infty }\lim _{t\rightarrow 0}\frac{f( 0) \times 2t}{2t} =1
\end{equation*}
\qed 



\begin{ques}
	设 $\displaystyle f( x) \in C[ a,b]$ 满足:对于任意的 $\displaystyle \varphi ( x) \in C[ a,b]$,只要 $\displaystyle \int _{a}^{b} \varphi ( x)\mathrm{d} x=0$,就有 $\displaystyle \int _{a}^{b} f( x) \varphi ( x)\mathrm{d} x=0$。证明:$\displaystyle f( x) =const,\ x\in [ a,b]$。
\end{ques}


记 $\displaystyle \int _{a}^{b} f( x)\mathrm{d} x=A$,令 $\displaystyle \varphi ( x) =f( x) -\frac{A}{b-a}$,则 $\displaystyle \int _{a}^{b} \varphi ( x)\mathrm{d} x=0$。

那么 $\displaystyle 0=\int _{a}^{b} f( x) \varphi ( x)\mathrm{d} x-\frac{A}{b-a}\int _{a}^{b} \varphi ( x)\mathrm{d} x=\int _{a}^{b}\left( f( x) -\frac{A}{b-a}\right)^{2}\mathrm{d} x$,因此 $\displaystyle f( x) \equiv \frac{A}{b-a}$。\qed 





\begin{ques}
设 $\displaystyle f( x) \in R[ 0,1] ,0< m\leqslant f( x) \leqslant M$,求证:$\displaystyle \int _{0}^{1} f( x)\mathrm{d} x\int _{0}^{1}\frac{\mathrm{d} x}{f( x)} \leqslant \frac{( m+M)^{2}}{4mM}$。
\end{ques}



对于 $\displaystyle [ 0,1]$ 上的分割


\begin{equation*}
	\Delta :0=x_{0} < x_{1} < \cdots < x_{n} =1,\lambda ( \Delta ) =\min( x_{i} -x_{i-1}) ,\xi _{i} \in ( x_{i-1} ,x_{i}) ,\Delta x_{i} =x_{i} -x_{i-1}
\end{equation*}


有:


\begin{equation*}
	\begin{aligned}
		\int _{0}^{1} f( x)\mathrm{d} x\int _{0}^{1}\frac{\mathrm{d} x}{f( x)} & =\lim _{\lambda ( \Delta )\rightarrow 0}\sum f( \xi _{i}) \Delta x_{i}\sum \frac{1}{f( \xi _{j})} \Delta x_{j}\\
		& =\lim _{\lambda ( \Delta )\rightarrow 0}\sum _{i< j}\left(\frac{f( \xi _{i})}{f( \xi _{j})} +\frac{f( \xi _{j})}{f( \xi _{i})}\right) \Delta x_{i} \Delta x_{j} +\sum \Delta x_{i}^{2}\\
		& =\lim _{\lambda ( \Delta )\rightarrow 0}\sum _{i< j}\left(\frac{f( \xi _{i})}{f( \xi _{j})} +\frac{f( \xi _{j})}{f( \xi _{i})}\right) \Delta x_{i} \Delta x_{j}
	\end{aligned}
\end{equation*}


考虑 $\displaystyle f( x) =x+\frac{1}{x} ,0< x< 1$,容易验证其在 $\displaystyle ( 0,1)$ 上单调下降。于是
\begin{equation*}
	\frac{f( \xi _{i})}{f( \xi _{j})} +\frac{f( \xi _{j})}{f( \xi _{i})} \leqslant \frac{m}{M} +\frac{M}{m}
\end{equation*}


那么
\begin{equation*}
	\begin{aligned}
		\int _{0}^{1} f( x)\mathrm{d} x\int _{0}^{1}\frac{\mathrm{d} x}{f( x)} & =\lim _{\lambda ( \Delta )\rightarrow 0}\sum _{i< j}\left(\frac{f( \xi _{i})}{f( \xi _{j})} +\frac{f( \xi _{j})}{f( \xi _{i})}\right) \Delta x_{i} \Delta x_{j}\\
		& \leqslant \lim _{\lambda ( \Delta )\rightarrow 0}\sum _{i< j}\left(\frac{m}{M} +\frac{M}{m}\right) \Delta x_{i} \Delta x_{j}\\
		& \leqslant \max_{0\leqslant x\leqslant 1}\left(( mx+M( 1-x))\left(\frac{m}{x} +\frac{M}{1-x}\right)\right)\\
		& =\max_{0\leqslant x\leqslant 1}\left( m+M+\frac{mM( 1-x)}{x} +\frac{mMx}{1-x}\right)\\
		& =\frac{( m+M)^{2}}{4mM}
	\end{aligned}
\end{equation*}
\qed 




\begin{ques}
	
【连续性 Jensen 不等式】设 $\displaystyle f( x) ,p( x) \in R[ a,b] ,m\leqslant f( x) \leqslant M,p( x) \geqslant 0,x\in [ a,b] ,\int _{a}^{b} p( x)\mathrm{d} x >0,\varphi ( x) \in D[ m,M]$ 且 $\displaystyle \varphi ( x)$ 在 $\displaystyle [ m,M]$ 凸,证明:
\begin{equation*}
	\varphi \left(\frac{\int _{a}^{b} p( x) f( x)\mathrm{d} x}{\int _{a}^{b} p( x)\mathrm{d} x}\right) \leqslant \frac{\int _{a}^{b} p( x) \varphi ( f( x))\mathrm{d} x}{\int _{a}^{b} p( x)\mathrm{d} x}
\end{equation*}


\end{ques}

考虑 $\displaystyle [ a,b]$ 上的分割:


\begin{equation*}
	\Delta :a=x_{0} < x_{1} < \cdots < x_{n} =b,\lambda ( \Delta ) =\max( x_{i} -x_{i-1}) ,\Delta x_{i} =x_{i} -x_{i-1} ,\xi _{i} \in ( x_{i-1} ,x_{i})
\end{equation*}


那么


\begin{equation*}
	\begin{aligned}
		\varphi \left(\frac{\int _{a}^{b} p( x) f( x)\mathrm{d} x}{\int _{a}^{b} p( x)\mathrm{d} x}\right) & =\lim _{\lambda ( \Delta )\rightarrow 0} \varphi \left(\frac{\sum _{i=1}^{n} p( \xi _{i}) f( \xi _{i}) \Delta x_{i}}{\sum _{i=1}^{n} p( \xi _{i}) \Delta x_{i}}\right)\\
		& =\lim _{\lambda ( \Delta )\rightarrow 0} \varphi \left(\sum _{i=1}^{n}\frac{p( \xi _{i}) \Delta x_{i}}{\left(\sum _{j=1}^{n} p( \xi _{j}) \Delta x_{j}\right)} f( \xi _{i})\right)
	\end{aligned}
\end{equation*}

因为 
\begin{equation*}
	\sum _{i=1}^{n}\frac{p( \xi _{i}) \Delta x_{i}}{\left(\sum _{j=1}^{n} p( \xi _{j}) \Delta x_{j}\right)} =1
\end{equation*}


根据琴生不等式,


\begin{equation*}
	\begin{aligned}
		\begin{array}{l}
			\varphi \left(\sum _{i=1}^{n}\frac{p( \xi _{i}) \Delta x_{i}}{\left(\sum _{j=1}^{n} p( \xi _{j}) \Delta x_{j}\right)} f( \xi _{i})\right)\\
		\end{array} & \leqslant \sum _{i=1}^{n}\frac{p( \xi _{i}) \Delta x_{i}}{\left(\sum _{j=1}^{n} p( \xi _{j}) \Delta x_{j}\right)} \varphi ( f( \xi _{i}))\\
		& =\frac{\sum _{i=1}^{n} p( \xi _{i}) \varphi ( f( \xi _{i})) \Delta x_{i}}{\left(\sum _{j=1}^{n} p( \xi _{j}) \Delta x_{j}\right)}
	\end{aligned}
\end{equation*}


进而根据积分的定义:

\begin{gather*}
	\varphi \left(\frac{\int _{a}^{b} p( x) f( x)\mathrm{d} x}{\int _{a}^{b} p( x)\mathrm{d} x}\right) =\lim _{\lambda ( \Delta )\rightarrow 0} \varphi \left(\sum _{i=1}^{n}\frac{p( \xi _{i}) \Delta x_{i}}{\left(\sum _{j=1}^{n} p( \xi _{j}) \Delta x_{j}\right)} f( \xi _{i})\right) \leqslant \\
	\lim _{\lambda ( \Delta )\rightarrow 0}\frac{\sum _{i=1}^{n} p( \xi _{i}) \varphi ( f( \xi _{i})) \Delta x_{i}}{\left(\sum _{j=1}^{n} p( \xi _{j}) \Delta x_{j}\right)} =\frac{\int _{a}^{b} p( x) \varphi ( f( x))\mathrm{d} x}{\int _{a}^{b} p( x)\mathrm{d} x}
\end{gather*}
\qed 



\ifx\allfiles\undefined
\end{document}
\fi