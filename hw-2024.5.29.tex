%-*-    coding: UTF-8   -*-
% !TEX program = xelatex
\ifx\allfiles\undefined
%\special{dvipdfmx:config z 0}% 取消压缩,加快编译速度
\documentclass[UTF-8]{ctexart}
\usepackage{graphicx}
\usepackage{subfigure}
\usepackage{xcolor}
\usepackage{amsmath}
\usepackage{amssymb}
\usepackage{tabularx}
\usepackage{amssymb}
\usepackage{amsthm}
%\usepackage[usenames,dvipsnames]{color}
\usepackage{hyperref}
\hypersetup{
	colorlinks=true,
	linkcolor=black,
	filecolor=black,      
	urlcolor=red,
	citecolor=black,
}
\usepackage{geometry}
\geometry{a4paper,centering,scale=0.80}
\usepackage[format=hang,font=small,textfont=it]{caption}
\usepackage[nottoc]{tocbibind}
\usepackage{algorithm}  
\usepackage{algorithmicx}  
\usepackage{algpseudocode}
\usepackage{prettyref}
\usepackage{framed}
\setlength{\parindent}{2em}
\usepackage{indentfirst}
\usepackage[framemethod=TikZ]{mdframed}
\newcounter{ques}[section]
\renewcommand{\theques}{\arabic{section}.\arabic{ques}}
\newcommand{\setParDis}{\setlength {\parskip} {0.3cm} }
\newcommand{\setParDef}{\setlength {\parskip} {0pt} }
\setParDis% 调整这一个subsection的段落间距
%\setParDef%恢复间距

\newenvironment{ques}[1][]{
	\refstepcounter{ques}
	\mdfsetup{
		frametitle={
			\tikz[baseline=(current bounding box.east), outer sep=0pt]
			\node[anchor=east,rectangle,fill=blue!20]
			{\strut Problem~\theques\ifstrempty{#1}{}{:~#1}};},
		innertopmargin=10pt,linecolor=blue!20,
		linewidth=2pt,topline=true,
		frametitleaboveskip=\dimexpr-\ht\strutbox\relax
	}
	\begin{mdframed}[]\relax
}{\end{mdframed}}

\newcounter{Thm}[section]
\renewcommand{\theThm}{\arabic{section}.\arabic{Thm}}
\newenvironment{Thm}[1][]{
	\refstepcounter{Thm}
	\mdfsetup{
		frametitle={
			\tikz[baseline=(current bounding box.east), outer sep=0pt]
			\node[anchor=east,rectangle,fill=blue!20]
			{\strut Theorem~\theThm\ifstrempty{#1}{}{:~#1}};},
		innertopmargin=10pt,linecolor=blue!20,
		linewidth=2pt,topline=true,
		frametitleaboveskip=\dimexpr-\ht\strutbox\relax
	}
	\begin{mdframed}[]\relax
}{\end{mdframed}}

\newcounter{Defi}[section]
\renewcommand{\theDefi}{\arabic{section}.\arabic{Defi}}
\newenvironment{Defi}[1][]{
	\refstepcounter{Defi}
	\mdfsetup{
		frametitle={
			\tikz[baseline=(current bounding box.east), outer sep=0pt]
			\node[anchor=east,rectangle,fill=blue!20]
			{\strut Definition~\theDefi\ifstrempty{#1}{}{:~#1}};},
		innertopmargin=10pt,linecolor=blue!20,
		linewidth=2pt,topline=true,
		frametitleaboveskip=\dimexpr-\ht\strutbox\relax
	}
	\begin{mdframed}[]\relax
}{\end{mdframed}}

\newrefformat{qlt}{\underline{性质 \ref{#1}}}
\newcommand{\tpf}[2]{\begin{ques}[#1]{\kaishu #2}\end{ques}}
\newcommand{\pf}[1]{\begin{ques}{\kaishu #1}\end{ques}}
\newcommand{\tthm}[2]{\begin{Thm}[#1]{\kaishu #2}\end{Thm}}
\newcommand{\thm}[1]{\begin{Thm}{\kaishu #1}\end{Thm}}
\newcommand{\tdefi}[2]{\begin{Defi}[#1]{\kaishu #2}\end{Defi}}
\newcommand{\defi}[1]{\begin{Defi}{\kaishu #1}\end{Defi}}
\newcommand{\opf}[1]{{\kaishu{#1}}}
\title{数学分析 I 作业(2024. Spring)}
\author{\texttt{As-The-Wind}}

\date{2024 年 2 月 19 日 $\rightarrow$ \today}

\date{}
\author{尹锦润}
\begin{document}
\maketitle
\fi

\section{2024.5.29 作业}
\begin{ques}
	证明:(1)$\displaystyle \ln\left| 2\sin\frac{x}{2}\right| =-\sum _{n=1}^{\infty }\frac{\cos nx}{n} ,x\neq 2k\pi ,k\in \mathbb{Z} ;$(2)$\displaystyle \ln 2=\sum _{n=1}^{+\infty }\frac{( -1)^{n-1}}{n}$。
\end{ques}



对于 (1),先求 $\displaystyle \ln\left| \sin\frac{x}{2}\right| $ 的傅里叶级数:


\begin{align*}
	b_{n} & =0,( n=1,2,\cdots )\\
	a_{0} & =\frac{1}{\pi }\int _{-\pi }^{\pi }\ln\left| \sin\frac{x}{2}\right| \mathrm{d} x\\
	& =\frac{2}{\pi }\int _{0}^{\pi }\ln\left(\sin\frac{x}{2}\right)\mathrm{d} x
\end{align*}

\begin{align*}
	I & =\int _{0}^{\pi }\ln\left(\sin\frac{x}{2}\right)\mathrm{d} x\\
	2I & =\int _{0}^{\pi }\ln\left(\sin\frac{x}{2}\right)\mathrm{d} x+\int _{0}^{\pi }\ln\left(\cos\frac{x}{2}\right)\mathrm{d} x\\
	& =\int _{0}^{\pi }\ln\left(\sin\frac{x}{2}\cos\frac{x}{2}\right)\mathrm{d} x\\
	& =\int _{0}^{\pi }\ln\left(\frac{\sin x}{2}\right)\mathrm{d} x\\
	& =\int _{0}^{\pi }\ln\sin x\mathrm{d} x-\int _{0}^{\pi }\ln 2\mathrm{d} x\\
	& =\int _{0}^{\frac{\pi }{2}}\ln\sin x\mathrm{d} x+\int _{0}^{\frac{\pi }{2}}\ln\sin\left( x+\frac{\pi }{2}\right)\mathrm{d} x-\int _{0}^{\pi }\ln 2\mathrm{d} x\\
	& =\frac{I}{2} +\frac{I}{2} -\int _{0}^{\pi }\ln 2\mathrm{d} x\\
	I & =-\pi \ln 2
\end{align*}


因此:


\begin{equation*}
	a_{0} =-2\ln 2
\end{equation*}


对于


\begin{align*}
	a_{n} & =\frac{1}{\pi }\int _{-\pi }^{\pi }\ln\left| \sin\frac{x}{2}\right| \cos nx\mathrm{d} x & \\
	& =\frac{2}{\pi }\int _{0}^{\pi }\ln\left(\sin\frac{x}{2}\right)\cos nx\mathrm{d} x & \\
	& =\frac{4}{\pi }\int _{0}^{\frac{\pi }{2}}\ln(\sin x)\cos 2nx\mathrm{d} x & \\
	& =\frac{4}{\pi }\frac{1}{2n}\int _{0}^{\frac{\pi }{2}}\ln(\sin x)\mathrm{d}\sin 2nx & \\
	& =-\frac{2}{n\pi }\int _{0}^{\frac{\pi }{2}}\frac{\sin 2nx\cos x}{\sin x}\mathrm{d} x & \\
	& =-\frac{2}{n\pi }\int _{0}^{\frac{\pi }{2}}\frac{\sin( 2n+1) x+\sin( 2n-1) x}{2\sin x}\mathrm{d} x & ( 1)\\
	& =-\frac{1}{n} & 
\end{align*}
(1):$\displaystyle \forall n,\int _{0}^{\frac{\pi }{2}}\frac{\sin\left( n+\frac{1}{2}\right) x}{2\sin\frac{x}{2}} =\frac{\pi }{2}$。

因此有:


\begin{gather*}
	\ln\left| \sin\frac{x}{2}\right| \sim -\ln 2-\sum _{n=1}^{+\infty }\frac{\cos nx}{n}\\
	\ln\left| 2\sin\frac{x}{2}\right| \sim -\sum _{n=1}^{+\infty }\frac{\cos nx}{n}
\end{gather*}


同时,我们注意到 $\displaystyle \ln\left| 2\sin\frac{x}{2}\right| $ 在 $\displaystyle ( -\pi ,0)$ 和 $\displaystyle ( 0,\pi )$ 上单调,在 $\displaystyle ( -\pi ,\pi )$ 上分段单调,因此 $\displaystyle \ln\left| 2\sin\frac{x}{2}\right| =-\sum _{n=1}^{+\infty }\frac{\cos nx}{n} ,x\in ( -\pi ,\pi )$,又因为 $\displaystyle \ln\left| 2\sin\frac{x}{2}\right| $ 是 $\displaystyle T=2\pi $ 的周期函数,于是 $\displaystyle \ln\left| 2\sin\frac{x}{2}\right| =-\sum _{n=1}^{+\infty }\frac{\cos nx}{n} ,x\in ( -\pi ,\pi ) ,x\neq 2k\pi ,k\in \mathbb{Z}$。\qed 

对于 (2):令 $\displaystyle x=\pi $,带入原式得 $\displaystyle \ln 2=\sum _{n=1}^{+\infty }\frac{( -1)^{n-1}}{n}$。\qed 





\begin{ques}
	决定出使下式成立的 $\displaystyle x$ 范围:$\displaystyle x^{2} =\frac{\pi ^{2}}{3} +4\sum _{n=1}^{\infty }( -1)^{n}\frac{\cos nx}{n^{2}}$。由此求出 $\displaystyle \sum _{n=1}^{+\infty }\frac{( -1)^{n-1}}{n^{2}}$ 的闭式。
\end{ques}



不难发现 $\displaystyle \frac{\pi ^{2}}{3} +4\sum _{n=1}^{\infty }( -1)^{n}\frac{\cos nx}{n^{2}}$ 是函数 $\displaystyle x^{2} ,x\in [ -\pi ,\pi ]$ 的傅里叶级数。

因为 $\displaystyle x^{2}$ 在 $\displaystyle [ -\pi ,\pi ]$ 上分段单调,于是有:


\begin{equation*}
	\frac{\pi ^{2}}{3} +4\sum _{n=1}^{\infty }( -1)^{n}\frac{\cos nx}{n^{2}} =\begin{cases}
		x^{2} & ,x\in ( -\pi ,\pi )\\
		\pi ^{2} & ,x=\pm \pi 
	\end{cases}
\end{equation*}


而对于 $\displaystyle x\in ( -\infty ,-\pi ) \cup ( \pi ,+\infty )$,因为傅里叶级数周期为 $\displaystyle 2\pi $,于是 $ $$\displaystyle \frac{\pi ^{2}}{3} +4\sum _{n=1}^{\infty }( -1)^{n}\frac{\cos nx}{n^{2}} \neq x^{2}$。

因此成立的范围是 $\displaystyle [ -\pi ,\pi ]$。\qed 

将 $\displaystyle x=0$ 带入,得到 $\displaystyle 0=\frac{\pi ^{2}}{3} +4\sum _{n=1}^{\infty }( -1)^{n}\frac{1}{n^{2}} \Rightarrow \sum _{n=1}^{+\infty }\frac{( -1)^{n-1}}{n^{2}} =\frac{\pi ^{2}}{12}$。\qed 
\ifx\allfiles\undefined
\end{document}
\fi