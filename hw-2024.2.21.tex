%-*-    coding: UTF-8   -*-
% !TEX program = xelatex
\ifx\allfiles\undefined
%\special{dvipdfmx:config z 0}% 取消压缩,加快编译速度
\documentclass[UTF-8]{ctexart}
\usepackage{graphicx}
\usepackage{subfigure}
\usepackage{xcolor}
\usepackage{amsmath}
\usepackage{amssymb}
\usepackage{tabularx}
\usepackage{amssymb}
\usepackage{amsthm}
%\usepackage[usenames,dvipsnames]{color}
\usepackage{hyperref}
\hypersetup{
	colorlinks=true,
	linkcolor=black,
	filecolor=black,      
	urlcolor=red,
	citecolor=black,
}
\usepackage{geometry}
\geometry{a4paper,centering,scale=0.80}
\usepackage[format=hang,font=small,textfont=it]{caption}
\usepackage[nottoc]{tocbibind}
\usepackage{algorithm}  
\usepackage{algorithmicx}  
\usepackage{algpseudocode}
\usepackage{prettyref}
\usepackage{framed}
\setlength{\parindent}{2em}
\usepackage{indentfirst}
\usepackage[framemethod=TikZ]{mdframed}
\newcounter{ques}[section]
\renewcommand{\theques}{\arabic{section}.\arabic{ques}}
\newcommand{\setParDis}{\setlength {\parskip} {0.3cm} }
\newcommand{\setParDef}{\setlength {\parskip} {0pt} }
\setParDis% 调整这一个subsection的段落间距
%\setParDef%恢复间距

\newenvironment{ques}[1][]{
	\refstepcounter{ques}
	\mdfsetup{
		frametitle={
			\tikz[baseline=(current bounding box.east), outer sep=0pt]
			\node[anchor=east,rectangle,fill=blue!20]
			{\strut Problem~\theques\ifstrempty{#1}{}{:~#1}};},
		innertopmargin=10pt,linecolor=blue!20,
		linewidth=2pt,topline=true,
		frametitleaboveskip=\dimexpr-\ht\strutbox\relax
	}
	\begin{mdframed}[]\relax
}{\end{mdframed}}

\newcounter{Thm}[section]
\renewcommand{\theThm}{\arabic{section}.\arabic{Thm}}
\newenvironment{Thm}[1][]{
	\refstepcounter{Thm}
	\mdfsetup{
		frametitle={
			\tikz[baseline=(current bounding box.east), outer sep=0pt]
			\node[anchor=east,rectangle,fill=blue!20]
			{\strut Theorem~\theThm\ifstrempty{#1}{}{:~#1}};},
		innertopmargin=10pt,linecolor=blue!20,
		linewidth=2pt,topline=true,
		frametitleaboveskip=\dimexpr-\ht\strutbox\relax
	}
	\begin{mdframed}[]\relax
}{\end{mdframed}}

\newcounter{Defi}[section]
\renewcommand{\theDefi}{\arabic{section}.\arabic{Defi}}
\newenvironment{Defi}[1][]{
	\refstepcounter{Defi}
	\mdfsetup{
		frametitle={
			\tikz[baseline=(current bounding box.east), outer sep=0pt]
			\node[anchor=east,rectangle,fill=blue!20]
			{\strut Definition~\theDefi\ifstrempty{#1}{}{:~#1}};},
		innertopmargin=10pt,linecolor=blue!20,
		linewidth=2pt,topline=true,
		frametitleaboveskip=\dimexpr-\ht\strutbox\relax
	}
	\begin{mdframed}[]\relax
}{\end{mdframed}}

\newrefformat{qlt}{\underline{性质 \ref{#1}}}
\newcommand{\tpf}[2]{\begin{ques}[#1]{\kaishu #2}\end{ques}}
\newcommand{\pf}[1]{\begin{ques}{\kaishu #1}\end{ques}}
\newcommand{\tthm}[2]{\begin{Thm}[#1]{\kaishu #2}\end{Thm}}
\newcommand{\thm}[1]{\begin{Thm}{\kaishu #1}\end{Thm}}
\newcommand{\tdefi}[2]{\begin{Defi}[#1]{\kaishu #2}\end{Defi}}
\newcommand{\defi}[1]{\begin{Defi}{\kaishu #1}\end{Defi}}
\newcommand{\opf}[1]{{\kaishu{#1}}}
\title{数学分析 I 作业(2024. Spring)}
\author{\texttt{As-The-Wind}}

\date{2024 年 2 月 19 日 $\rightarrow$ \today}

\date{}
\author{尹锦润}
\begin{document}
\maketitle
\fi

\section{2024.2.21 作业}

\begin{ques}
	设函数 $\displaystyle f( x)$ 在 $\displaystyle [ a,b]$ 有定义,记 $\displaystyle f^{+}( x) =\max( f( x) ,0) ;f^{-}( x) =-\min( f( x) ,0)$。证明:$\displaystyle f( x) \in R[ a,b]$ 的充分必要条件是 $\displaystyle f^{+}( x)$ 和 $\displaystyle f^{-}( x)$ 在 $\displaystyle [ a,b]$ 可积。
\end{ques}



记 $\displaystyle f( x)$ 的零点构成集合为 $\displaystyle \Delta _{0}$。

先证 $\displaystyle \Leftarrow $:

$\displaystyle f^{+}( x)$ 和 $\displaystyle f^{-}( x)$ 在 $\displaystyle [ a,b]$ 可积,那么 $\displaystyle f( x) =f^{+}( x) -f^{-}( x)$ 也是可积的。

证明 $\displaystyle \Rightarrow $:

考虑 $\displaystyle f( x)$ 的任一分割 $\displaystyle \Delta :a=x_{0} < x_{1} < \cdots < x_{n} =b$,对于 $\displaystyle [ x_{i-1} ,x_{i}]$ 的振幅 $\displaystyle \omega _{i}$ 有 $\displaystyle \omega _{i}\left( f^{+}\right) \leqslant \omega _{i}( f) ,\omega _{i}\left( f^{-}\right) \leqslant \omega _{i}( f)$,当 $\displaystyle f( x) \in R[ a,b]$ 时,有 $\displaystyle \forall \varepsilon  >0,\exists \Delta $ 使得 $\displaystyle \sum \omega _{i}( f) \Delta x_{i} < \varepsilon $,也就是$ $$\displaystyle \sum \omega _{i}\left( f^{+}\right) \Delta x_{i} < \varepsilon ,\sum \omega _{i}\left( f^{-}\right) \Delta x_{i} < \varepsilon $,于是 $\displaystyle f^{+} ,f^{-} \in R[ a,b]$。\qed 





\begin{ques}
	(1)若 $\displaystyle f( x) \in R[ a,b]$,是否有 $\displaystyle |f( x) |\in R[ a,b]$?反之如何?\\(2)若 $\displaystyle f( x) \in R[ a,b]$,是否有 $\displaystyle f( x)^{2} \in R[ a,b]$?反之如何?

\end{ques}


(1)是;反之否。

考虑 $\displaystyle [ a,b]$ 上一组分割 $\displaystyle \Delta $,对于 $\displaystyle [ x_{i-1} ,x_{i}]$ 上任意 $\displaystyle c,d$,有 $\displaystyle ||f( c) |-|f( d) ||\leqslant |f( c) -f( d) |$。

因为 $\displaystyle f( x) \in R[ a,b]$,于是 $\displaystyle \forall \varepsilon  >0,\exists \Delta \ s.t.\ \sum \omega _{i}( f) \Delta x_{i} \leqslant \varepsilon $,进而 $\displaystyle \sum \omega _{i}( |f|) \Delta x_{i} \leqslant \varepsilon $。



考虑 $\displaystyle f( x) =\begin{cases}
1 & ,x\in \mathbb{Q}\\
-1 & ,x\notin \mathbb{Q}
\end{cases} ,x\in [ 0,1]$,显然 $\displaystyle |f( x) |\in R[ 0,1] ,$但是 $\displaystyle f( x)$ 在 $\displaystyle [ 0,1]$ 处处间断,不可积。





(2)是;反之否。

当 $\displaystyle f( x) \in R[ a,b]$,$\displaystyle f( x)$ 有界,记 $\displaystyle |f( x) |\leqslant M$。同时 $\displaystyle \forall \varepsilon  >0,\exists \Delta \ s.t.\ \sum \omega _{i}( f) \Delta x_{i} \leqslant \varepsilon $。

而 $\displaystyle \omega _{i}\left( f^{2}\right) \leqslant 2\omega _{i} M,\sum \omega _{i}\left( f^{2}\right) \Delta x_{i} \leqslant 2M\varepsilon $,因为 $\displaystyle 2M$ 是常数,于是 $\displaystyle f^{2}( x) \in R[ a,b]$。



反之,同样考虑 $\displaystyle f( x) =\begin{cases}
1 & ,x\in \mathbb{Q}\\
-1 & ,x\notin \mathbb{Q}
\end{cases} ,x\in [ 0,1]$ 即可。





\begin{ques}
	设函数 $\displaystyle f( x) ,g( x) \in R[ a,b]$,记 $\displaystyle \Delta :a=x_{0} < x_{1} < \cdots < x_{n} =b$,为 $\displaystyle [ a,b]$ 的分割;并取 $\displaystyle \xi _{i} ,\eta _{i} \in [ x_{i-1} ,x_{i}] ,i=1,2,\cdots ,n$,试证明:$\displaystyle \lim _{\lambda ( \Delta )\rightarrow 0}\sum _{i=1}^{n} f( \xi _{i}) g( \eta _{i}) \Delta x_{i} =\int _{a}^{b} f( x) g( x)\mathrm{d} x$。

\end{ques}


首先 $\displaystyle \int _{a}^{b} f( x) g( x)\mathrm{d} x=\lim _{\lambda ( \Delta )\rightarrow 0}\sum f( \xi _{i}) g( \xi _{i}) \Delta x_{i}$,因为 $\displaystyle f( x) \in R[ a,b]$,于是存在 $\displaystyle M$ 使得 $\displaystyle |f( x) |\leqslant M$。

进而


\begin{equation*}
\begin{aligned}
	\left| \sum _{i=1}^{n} f( \xi _{i}) g( \eta _{i}) \Delta x_{i} -\int _{a}^{b} f( x) g( x)\mathrm{d} x\right|  & =\left| \sum f( \xi _{i})( g( \xi _{i}) -g( \eta _{i})) \Delta x_{i}\right| \\
	& \leqslant M\left| \sum ( g( \xi _{i}) -g( \eta _{i})) \Delta x_{i}\right| \\
	& \leqslant M\omega _{i}( g) \Delta x_{i}
\end{aligned}
\end{equation*}

因为 $\displaystyle g( x) \in R[ a,b]$ 以及 $\displaystyle \varepsilon $ 任意性,可得 $\displaystyle \forall \varepsilon  >0,\exists \Delta \ s.t.\left| \sum _{i=1}^{n} f( \xi _{i}) g( \eta _{i}) \Delta x_{i} -\int _{a}^{b} f( x) g( x)\mathrm{d} x\right| < \varepsilon \ $。\qed 





\begin{ques}
	设函数 $\displaystyle f( x)$ 在 $\displaystyle [ a,b]$ 可积且存在 $\displaystyle \alpha  >0$ 使得对 $\displaystyle \forall x\in [ a,b] \ s.t.\ f( x) \geqslant \alpha $,试证明:(1)$\displaystyle \frac{1}{f( x)} \in R[ a,b]$;(2)$\displaystyle \ln f( x) \in R[ a,b]$。

\end{ques}




(1)

考虑分割 $\displaystyle \Delta :a=x_{0} < x_{1} < \cdots < x_{n} =b$,对于 $\displaystyle c,d\in [ x_{i-1} ,x_{i}]$,有


\begin{equation*}
\left| \frac{1}{f( c)} -\frac{1}{f( d)}\right| =\left| \frac{f( d) -f( c)}{f( c) f( d)}\right| \leqslant \frac{\omega _{i}}{\alpha ^{2}}
\end{equation*}


$\displaystyle f( x) \in R[ a,b] \Rightarrow \forall \varepsilon  >0,\exists \Delta \ s.t.\ \sum \omega _{i}( f( x)) \Delta x_{i} \leqslant \varepsilon \Rightarrow \sum \omega _{i}\left(\frac{1}{f( x)}\right) \Delta x_{i} \leqslant \frac{\varepsilon }{\alpha ^{2}}$。

因为 $\displaystyle \frac{1}{\alpha ^{2}}$ 是常数,所以 $\displaystyle \frac{1}{f( x)} \in R[ a,b]$。





(2)

考虑分割 $\displaystyle \Delta :a=x_{0} < x_{1} < \cdots < x_{n} =b$,对于 $\displaystyle c,d\in [ x_{i-1} ,x_{i}]$,有


\begin{equation*}
| \ln f( c) -\ln f( d)| =\left| \ln\frac{f( c)}{f( d)}\right| \leqslant \left| \frac{f( c) -f( d)}{f( d)}\right| \leqslant \frac{\omega _{i}}{\alpha }
\end{equation*}


$\displaystyle f( x) \in R[ a,b] \Rightarrow \forall \varepsilon  >0,\exists \Delta \ s.t.\ \sum \omega _{i}( f( x)) \Delta x_{i} \leqslant \varepsilon \Rightarrow \sum \omega _{i}(\ln f( x)) \Delta x_{i} \leqslant \frac{\varepsilon }{\alpha }$。

因为 $\displaystyle \frac{1}{\alpha }$ 是常数,所以 $\displaystyle \ln f( x) \in R[ a,b]$。










\ifx\allfiles\undefined
\end{document}
\fi