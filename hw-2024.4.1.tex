%-*-    coding: UTF-8   -*-
% !TEX program = xelatex
\ifx\allfiles\undefined
%\special{dvipdfmx:config z 0}% 取消压缩,加快编译速度
\documentclass[UTF-8]{ctexart}
\usepackage{graphicx}
\usepackage{subfigure}
\usepackage{xcolor}
\usepackage{amsmath}
\usepackage{amssymb}
\usepackage{tabularx}
\usepackage{amssymb}
\usepackage{amsthm}
%\usepackage[usenames,dvipsnames]{color}
\usepackage{hyperref}
\hypersetup{
	colorlinks=true,
	linkcolor=black,
	filecolor=black,      
	urlcolor=red,
	citecolor=black,
}
\usepackage{geometry}
\geometry{a4paper,centering,scale=0.80}
\usepackage[format=hang,font=small,textfont=it]{caption}
\usepackage[nottoc]{tocbibind}
\usepackage{algorithm}  
\usepackage{algorithmicx}  
\usepackage{algpseudocode}
\usepackage{prettyref}
\usepackage{framed}
\setlength{\parindent}{2em}
\usepackage{indentfirst}
\usepackage[framemethod=TikZ]{mdframed}
\newcounter{ques}[section]
\renewcommand{\theques}{\arabic{section}.\arabic{ques}}
\newcommand{\setParDis}{\setlength {\parskip} {0.3cm} }
\newcommand{\setParDef}{\setlength {\parskip} {0pt} }
\setParDis% 调整这一个subsection的段落间距
%\setParDef%恢复间距

\newenvironment{ques}[1][]{
	\refstepcounter{ques}
	\mdfsetup{
		frametitle={
			\tikz[baseline=(current bounding box.east), outer sep=0pt]
			\node[anchor=east,rectangle,fill=blue!20]
			{\strut Problem~\theques\ifstrempty{#1}{}{:~#1}};},
		innertopmargin=10pt,linecolor=blue!20,
		linewidth=2pt,topline=true,
		frametitleaboveskip=\dimexpr-\ht\strutbox\relax
	}
	\begin{mdframed}[]\relax
}{\end{mdframed}}

\newcounter{Thm}[section]
\renewcommand{\theThm}{\arabic{section}.\arabic{Thm}}
\newenvironment{Thm}[1][]{
	\refstepcounter{Thm}
	\mdfsetup{
		frametitle={
			\tikz[baseline=(current bounding box.east), outer sep=0pt]
			\node[anchor=east,rectangle,fill=blue!20]
			{\strut Theorem~\theThm\ifstrempty{#1}{}{:~#1}};},
		innertopmargin=10pt,linecolor=blue!20,
		linewidth=2pt,topline=true,
		frametitleaboveskip=\dimexpr-\ht\strutbox\relax
	}
	\begin{mdframed}[]\relax
}{\end{mdframed}}

\newcounter{Defi}[section]
\renewcommand{\theDefi}{\arabic{section}.\arabic{Defi}}
\newenvironment{Defi}[1][]{
	\refstepcounter{Defi}
	\mdfsetup{
		frametitle={
			\tikz[baseline=(current bounding box.east), outer sep=0pt]
			\node[anchor=east,rectangle,fill=blue!20]
			{\strut Definition~\theDefi\ifstrempty{#1}{}{:~#1}};},
		innertopmargin=10pt,linecolor=blue!20,
		linewidth=2pt,topline=true,
		frametitleaboveskip=\dimexpr-\ht\strutbox\relax
	}
	\begin{mdframed}[]\relax
}{\end{mdframed}}

\newrefformat{qlt}{\underline{性质 \ref{#1}}}
\newcommand{\tpf}[2]{\begin{ques}[#1]{\kaishu #2}\end{ques}}
\newcommand{\pf}[1]{\begin{ques}{\kaishu #1}\end{ques}}
\newcommand{\tthm}[2]{\begin{Thm}[#1]{\kaishu #2}\end{Thm}}
\newcommand{\thm}[1]{\begin{Thm}{\kaishu #1}\end{Thm}}
\newcommand{\tdefi}[2]{\begin{Defi}[#1]{\kaishu #2}\end{Defi}}
\newcommand{\defi}[1]{\begin{Defi}{\kaishu #1}\end{Defi}}
\newcommand{\opf}[1]{{\kaishu{#1}}}
\title{数学分析 I 作业(2024. Spring)}
\author{\texttt{As-The-Wind}}

\date{2024 年 2 月 19 日 $\rightarrow$ \today}

\date{}
\author{尹锦润}
\begin{document}
\maketitle
\fi

\section{2024.4.1 作业}
\begin{ques}
	判断级数的敛散性 $\displaystyle \sum _{n=1}^{+\infty }\left(\frac{( 2n) !!}{( 2n+3) !!}\right)^{p}( p >0)$。
\end{ques}



对于 $\displaystyle a_{n} =\left(\frac{( 2n) !!}{( 2n+3) !!}\right)^{p} ,\frac{a_{n}}{a_{n+1}} =\left(\frac{2n+5}{2n+2}\right)^{p}$,有


\begin{align*}
	\lim _{n\rightarrow +\infty } n\left(\frac{a_{n}}{a_{n+1}} -1\right) & =\lim _{n\rightarrow +\infty } n\left(\left(\frac{2n+5}{2n+2}\right)^{p} -1\right) & \\
	& =\lim _{t\rightarrow 0+}\frac{\left(\frac{2+5t}{2+2t}\right)^{p} -1}{t} & ( 1)\\
	& =\lim _{t\rightarrow 0+}\frac{p\left(\frac{2+5t}{2+2t}\right)^{p-1}\frac{6}{( 2+2t)^{2}}}{1} & ( 2)\\
	& =\frac{3}{2} p & 
\end{align*}


(1):令 $\displaystyle t=\frac{1}{n}$。

(2):因为 $\displaystyle t\rightarrow 0,\left(\frac{2+5t}{2+2t}\right)^{p} -1\rightarrow 0$ 且 $\displaystyle \lim _{t\rightarrow 0+}\frac{p\left(\frac{2+5t}{2+2t}\right)^{p-1}\frac{6}{( 2+2t)^{2}}}{1}$ 存在,因此使用洛必达法则。

考虑拉贝判别法,当 $\displaystyle \lim _{n\rightarrow +\infty } n\left(\frac{a_{n}}{a_{n+1}} -1\right)  >1\ i.e.\ p >\frac{2}{3}$ 时,有原级数收敛,当 $\displaystyle p< \frac{2}{3}$ 时原级数发散,当 $\displaystyle p=\frac{2}{3}$ 时,$\displaystyle \frac{a_{n}}{a_{n+1}} =1+\frac{1}{n} +o\left(\frac{1}{n}\right) =1+\frac{1}{n} +O\left(\frac{1}{n^{1+\varepsilon }}\right)\left( n\rightarrow +\infty ,\exists \varepsilon  >0\right)$,由高斯判别法,发散。

综上,当 $\displaystyle p >\frac{2}{3}$ 时原级数收敛,当 $\displaystyle p\leqslant \frac{2}{3}$ 时原级数发散。\qed 



\begin{ques}
	判断级数的敛散性:$\displaystyle \sum _{n=1}^{+\infty }\frac{1-n\sin\frac{1}{n}}{n^{\alpha }}( \alpha  >0)$。
\end{ques}



$\displaystyle n\rightarrow +\infty $ 时,$\displaystyle \sin\frac{1}{n} =\frac{1}{n} -\frac{1}{6n^{3}} +o\left(\frac{1}{n^{3}}\right)$。

因此,
\begin{align*}
	\sum _{n=1}^{+\infty }\frac{1-n\sin\frac{1}{n}}{n^{\alpha }} & =\sum _{n=1}^{+\infty }\frac{\frac{1}{6n^{2}} +o\left(\frac{1}{n^{2}}\right)}{n^{\alpha }} & \\
	& =\sum _{n=1}^{+\infty }\frac{1}{6n^{\alpha +2}} & ( 1)
\end{align*}

(1):该级数为正,使用极限形式的比较判别法。

因为 $\displaystyle \alpha  >0$,显然该级数收敛。\qed 





\begin{ques}
	设对于 $\displaystyle \forall n\in \mathbb{N}$ 有 $\displaystyle a_{n}  >0$ 并且 $\displaystyle \sum _{n=1}^{+\infty } a_{n}$ 发散,记 $\displaystyle S_{n} =\sum _{k=1}^{n} a_{k}$,证明:

(1)级数 $\displaystyle \sum _{n=1}^{+\infty }\frac{a_{n}}{S_{n}}$ 发散;(2)级数 $\displaystyle \sum _{n=1}^{+\infty }\frac{a_{n}}{S_{n}^{2}}$收敛;(3)级数 $\displaystyle \sum _{n=1}^{+\infty }\frac{a_{n}}{1+a_{n}}$ 发散;(4)$\displaystyle \sum _{n=1}^{+\infty }\frac{a_{n}}{1+n^{2} a_{n}}$收敛。
\end{ques}





(1)考虑反证,如果 $\displaystyle \sum _{n=1}^{+\infty }\frac{a_{n}}{S_{n}}$ 收敛,那么 $\displaystyle \frac{a_{n}}{S_{n}} =\frac{S_{n} -S_{n-1}}{S_{n}}\rightarrow 0$ 因为 $\displaystyle S_{n}$ 发散且 $\displaystyle  >0$,所以 $\displaystyle S_{n} -S_{n-1}\rightarrow 0$,这与发散矛盾。\qed 



(2)
\begin{align*}
	\sum _{n=1}^{+\infty }\frac{a_{n}}{S_{n}^{2}} & \leqslant 1+\sum _{n=2}^{+\infty }\frac{a_{n}}{S_{n} S_{n-1}}\\
	& =1-\sum _{n=2}^{+\infty }\int _{S_{n-1}}^{S_{n}}\frac{\mathrm{d} x}{x^{2}}\\
	& =1+\left(\frac{1}{x}\middle| _{a_{1}}^{S_{+\infty }}\right)
\end{align*}


因为 $\displaystyle S_{n}$ 单调递增且发散,于是 $\displaystyle S_{+\infty } =+\infty $,进而 $\displaystyle \sum _{n=1}^{+\infty }\frac{a_{n}}{S_{n}^{2}}$ 收敛。\qed 



(3)

如果 $\displaystyle a_{n}$ 上界存在,记为 $\displaystyle M$,则 $\displaystyle \sum _{n=1}^{+\infty }\frac{a_{n}}{1+a_{n}} \geqslant \sum _{n=1}^{+\infty }\frac{a_{n}}{1+M} =\frac{1}{1+M}\sum _{n=1}^{+\infty } a_{n}$ 发散。

如果 $\displaystyle a_{n}$ 上界不存在,那么 $\displaystyle \varlimsup _{n\rightarrow +\infty } a_{n} =+\infty ,\varlimsup _{n\rightarrow +\infty }\frac{a_{n}}{1+a_{n}} =1$,于是 $\displaystyle \sum _{n=1}^{+\infty }\frac{a_{n}}{1+a_{n}}$ 发散。\qed 



(4)
\begin{align*}
	\sum _{n=1}^{+\infty }\frac{a_{n}}{1+n^{2} a_{n}} & =\sum _{n=1}^{+\infty }\frac{1}{\frac{1}{a_{n}} +n^{2}}\\
	& \leqslant \sum _{n=1}^{+\infty }\frac{1}{n^{2}}
\end{align*}


而 $\displaystyle \sum _{n=1}^{+\infty }\frac{1}{n^{2}}$ 收敛,于是原级数收敛。\qed 



\begin{ques}
	设 $\displaystyle a_{n}  >0,n\in \mathbb{N} ,\{a_{n} -a_{n+1}\}$ 为严格递减数列,若 $\displaystyle \sum _{n=1}^{+\infty } a_{n}$ 收敛,证明:

(1)$\displaystyle \{a_{n}\}$ 严格递减;(2)$\displaystyle \lim _{n\rightarrow \infty }\left(\frac{1}{a_{n+1}} -\frac{1}{a_{n}}\right) =+\infty $。
\end{ques}



(1)考虑反证,如果 $\displaystyle \exists N >0\ s.t.\ a_{N} \leqslant a_{N+1}$,那么 $\displaystyle \forall n >N\ a_{n} -a_{n+1} < a_{N} -a_{N+1} \leqslant 0$,进而 $\displaystyle a_{n}$ 在 $\displaystyle n >N$ 后单调上升,进而 $\displaystyle \lim _{n\rightarrow +\infty } a_{n} \neq 0$,和 $\displaystyle \sum _{n=1}^{+\infty } a_{n}$ 收敛 矛盾。\qed 

(2)


\begin{align*}
	\frac{a_{n} a_{n+1}}{a_{n} -a_{n+1}} & \leqslant \frac{a_{n}^{2}}{a_{n} -a_{n+1}}\\
	& =\frac{\sum _{k=n}^{+\infty } a_{k}^{2} -a_{k+1}^{2}}{a_{n} -a_{n+1}}\\
	& =\sum _{k=n}^{+\infty } a_{k} +a_{k+1}\\
	& =\sum _{k=n}^{+\infty } a_{k} +\sum _{k=n+1}^{+\infty } a_{k}
\end{align*}


当 $\displaystyle n\rightarrow +\infty $,因为 $\displaystyle \sum _{n=1}^{+\infty } a_{n}$ 收敛,于是 $\displaystyle \sum _{k=n}^{+\infty } a_{k}\rightarrow 0$,进而 $\displaystyle \lim _{n\rightarrow +\infty }\frac{a_{n} a_{n+1}}{a_{n} -a_{n+1}} =0$,于是
\begin{equation*}
	\lim _{n\rightarrow \infty }\left(\frac{1}{a_{n+1}} -\frac{1}{a_{n}}\right) =\lim _{n\rightarrow +\infty }\frac{1}{\frac{a_{n} a_{n+1}}{a_{n} -a_{n+1}}} =+\infty 
\end{equation*}
\qed 



\begin{ques}
	设 $\displaystyle a_{n}  >0,n\in \mathbb{N} ,\exists \alpha \in ( 0,1) \ s.t.\ \lim _{n\rightarrow \infty } n^{\alpha }\left(\frac{a_{n}}{a_{n+1}} -1\right) =\lambda ( \lambda  >0\ or\ =+\infty )$。证明:$\displaystyle \forall k\in \mathbb{N}^{*} =\{0,1,2,\cdots \} ,\sum _{n=1}^{+\infty } n^{k} a_{n}$ 收敛。
\end{ques}



$\displaystyle \lim _{n\rightarrow \infty } n^{\alpha }\left(\frac{a_{n}}{a_{n+1}} -1\right) =\lambda $,进而 $\displaystyle \lim _{n\rightarrow \infty } n\left(\frac{a_{n}}{a_{n+1}} -1\right) =\lim _{n\rightarrow \infty } n^{1-\alpha } \times \lim _{n\rightarrow \infty } n^{\alpha }\left(\frac{a_{n}}{a_{n+1}} -1\right) =+\infty  >1$。

考虑
\begin{align*}
	\lim _{n\rightarrow \infty } n\left(\frac{n^{k} a_{n}}{( n+1)^{k} a_{n+1}} -1\right) & =\lim _{n\rightarrow \infty }\frac{n^{k+1}( a_{n} -a_{n+1})}{n^{k} a_{n+1}}\\
	& =\lim _{n\rightarrow \infty } n\left(\frac{a_{n}}{a_{n+1}} -1\right)  >1
\end{align*}

于是 $\displaystyle \sum _{n=1}^{+\infty } n^{k} a_{n}$ 收敛。\qed 
\ifx\allfiles\undefined
\end{document}
\fi