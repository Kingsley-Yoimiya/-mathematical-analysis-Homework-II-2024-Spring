%-*-    coding: UTF-8   -*-
% !TEX program = xelatex
\ifx\allfiles\undefined
%\special{dvipdfmx:config z 0}% 取消压缩,加快编译速度
\documentclass[UTF-8]{ctexart}
\usepackage{graphicx}
\usepackage{subfigure}
\usepackage{xcolor}
\usepackage{amsmath}
\usepackage{amssymb}
\usepackage{tabularx}
\usepackage{amssymb}
\usepackage{amsthm}
%\usepackage[usenames,dvipsnames]{color}
\usepackage{hyperref}
\hypersetup{
	colorlinks=true,
	linkcolor=black,
	filecolor=black,      
	urlcolor=red,
	citecolor=black,
}
\usepackage{geometry}
\geometry{a4paper,centering,scale=0.80}
\usepackage[format=hang,font=small,textfont=it]{caption}
\usepackage[nottoc]{tocbibind}
\usepackage{algorithm}  
\usepackage{algorithmicx}  
\usepackage{algpseudocode}
\usepackage{prettyref}
\usepackage{framed}
\setlength{\parindent}{2em}
\usepackage{indentfirst}
\usepackage[framemethod=TikZ]{mdframed}
\newcounter{ques}[section]
\renewcommand{\theques}{\arabic{section}.\arabic{ques}}
\newcommand{\setParDis}{\setlength {\parskip} {0.3cm} }
\newcommand{\setParDef}{\setlength {\parskip} {0pt} }
\setParDis% 调整这一个subsection的段落间距
%\setParDef%恢复间距

\newenvironment{ques}[1][]{
	\refstepcounter{ques}
	\mdfsetup{
		frametitle={
			\tikz[baseline=(current bounding box.east), outer sep=0pt]
			\node[anchor=east,rectangle,fill=blue!20]
			{\strut Problem~\theques\ifstrempty{#1}{}{:~#1}};},
		innertopmargin=10pt,linecolor=blue!20,
		linewidth=2pt,topline=true,
		frametitleaboveskip=\dimexpr-\ht\strutbox\relax
	}
	\begin{mdframed}[]\relax
}{\end{mdframed}}

\newcounter{Thm}[section]
\renewcommand{\theThm}{\arabic{section}.\arabic{Thm}}
\newenvironment{Thm}[1][]{
	\refstepcounter{Thm}
	\mdfsetup{
		frametitle={
			\tikz[baseline=(current bounding box.east), outer sep=0pt]
			\node[anchor=east,rectangle,fill=blue!20]
			{\strut Theorem~\theThm\ifstrempty{#1}{}{:~#1}};},
		innertopmargin=10pt,linecolor=blue!20,
		linewidth=2pt,topline=true,
		frametitleaboveskip=\dimexpr-\ht\strutbox\relax
	}
	\begin{mdframed}[]\relax
}{\end{mdframed}}

\newcounter{Defi}[section]
\renewcommand{\theDefi}{\arabic{section}.\arabic{Defi}}
\newenvironment{Defi}[1][]{
	\refstepcounter{Defi}
	\mdfsetup{
		frametitle={
			\tikz[baseline=(current bounding box.east), outer sep=0pt]
			\node[anchor=east,rectangle,fill=blue!20]
			{\strut Definition~\theDefi\ifstrempty{#1}{}{:~#1}};},
		innertopmargin=10pt,linecolor=blue!20,
		linewidth=2pt,topline=true,
		frametitleaboveskip=\dimexpr-\ht\strutbox\relax
	}
	\begin{mdframed}[]\relax
}{\end{mdframed}}

\newrefformat{qlt}{\underline{性质 \ref{#1}}}
\newcommand{\tpf}[2]{\begin{ques}[#1]{\kaishu #2}\end{ques}}
\newcommand{\pf}[1]{\begin{ques}{\kaishu #1}\end{ques}}
\newcommand{\tthm}[2]{\begin{Thm}[#1]{\kaishu #2}\end{Thm}}
\newcommand{\thm}[1]{\begin{Thm}{\kaishu #1}\end{Thm}}
\newcommand{\tdefi}[2]{\begin{Defi}[#1]{\kaishu #2}\end{Defi}}
\newcommand{\defi}[1]{\begin{Defi}{\kaishu #1}\end{Defi}}
\newcommand{\opf}[1]{{\kaishu{#1}}}
\title{数学分析 I 作业(2024. Spring)}
\author{\texttt{As-The-Wind}}

\date{2024 年 2 月 19 日 $\rightarrow$ \today}

\date{}
\author{尹锦润}
\begin{document}
\maketitle
\fi

\section{2024.3.27 作业}
\begin{ques}
判断级数是否收敛,若收敛,请求和:$\displaystyle \sum _{n=1}^{+\infty }\frac{n}{3^{n}}$。
\end{ques}


\begin{align*}
	\sum _{n=1}^{+\infty }\frac{n}{3^{n}} & =\lim _{N\rightarrow \infty }\sum _{n=1}^{+\infty }\frac{1}{3^{n}} +\sum _{n=2}^{+\infty }\frac{1}{3^{n}} +\cdots +\sum _{n=N}^{+\infty }\frac{1}{3^{n}}\\
	& =\lim _{N\rightarrow \infty }\frac{\frac{1}{3} +\frac{1}{9} +\cdots +\frac{1}{3^{N}}}{\frac{2}{3}}\\
	& =\frac{\frac{1}{2}}{\frac{2}{3}} =\frac{4}{3}
\end{align*}


因此,该级数收敛,并且和为 $\displaystyle \frac{4}{3}$。\qed 





\begin{ques}
	判断级数的敛散性:$\displaystyle \sum _{n=1}^{+\infty }\left(\frac{\left( 1+\frac{1}{n}\right)^{n}}{e}\right)^{n}$。
\end{ques}





考虑


\begin{align*}
	\left(\frac{\left( 1+\frac{1}{n}\right)^{n}}{e}\right)^{n} & \geqslant \left(\frac{\left( 1+\frac{1}{n}\right)^{n}}{\left( 1+\frac{1}{n}\right)^{n+1}}\right)^{n}\\
	& =\frac{1}{\left( 1+\frac{1}{n}\right)^{n}}\\
	\lim _{n\rightarrow +\infty }\frac{1}{\left( 1+\frac{1}{n}\right)^{n}} & =\frac{1}{e}
\end{align*}


因此


\begin{equation*}
	\lim _{n\rightarrow +\infty }\left(\frac{\left( 1+\frac{1}{n}\right)^{n}}{e}\right)^{n} \geqslant \frac{1}{e}  >0
\end{equation*}


故级数发散。\qed 



\begin{ques}
	判断级数的敛散性:$\displaystyle \sum _{n=1}^{+\infty }\frac{a^{n}}{1+a^{kn}}( a >0,k >1)$。
\end{ques}


\begin{align*}
	\frac{a^{n}}{1+a^{kn}} & =\frac{1}{a^{-n} +a^{( k-1) n}}
\end{align*}


当 $\displaystyle a=1$ 时,级数发散。

当 $\displaystyle a >1$ 时,级数和 $\displaystyle \sum _{n=1}^{+\infty }\frac{1}{a^{( k-1) n}}$ 同敛散,因此收敛。

当 $\displaystyle a< 1$ 时,级数和 $\displaystyle \sum _{n=1}^{+\infty } a^{n}$ 同敛散,因此收敛。\qed 





\begin{ques}
	判断级数的敛散性:$\displaystyle \sum _{n=1}^{+\infty } n^{p}\left(\sqrt{n+1} -2\sqrt{n} +\sqrt{n-1}\right)$。
\end{ques}




\begin{align*}
	\sum _{n=1}^{+\infty } n^{p}\left(\sqrt{n+1} -2\sqrt{n} +\sqrt{n-1}\right) & =\sum _{n=1}^{+\infty } n^{p+\frac{1}{2}}\left(\sqrt{\frac{n+1}{n}} +\sqrt{\frac{n-1}{n}} -2\right) & \\
	& =\sum _{n=1}^{+\infty } n^{p+\frac{1}{2}}\left( -\frac{1}{2}\frac{1}{n^{2}} +o\left(\frac{1}{n^{3}}\right)\right) & ( 1)
\end{align*}


(1):利用 $\displaystyle ( 1+x)^{\frac{1}{2}} =1+\frac{1}{2} x-\frac{1}{4} x^{2} +o\left( x^{3}\right)$。

因此,原级数和 $\displaystyle \sum _{n=1}^{+\infty } n^{p-\frac{3}{2}}$ 相同,当 $\displaystyle p-\frac{3}{2} < -1\ i.e.\ p< \frac{1}{2}$ 时收敛,当 $\displaystyle p\geqslant \frac{1}{2}$ 时发散。\qed 
\ifx\allfiles\undefined
\end{document}
\fi