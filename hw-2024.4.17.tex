%-*-    coding: UTF-8   -*-
% !TEX program = xelatex
\ifx\allfiles\undefined
%\special{dvipdfmx:config z 0}% 取消压缩,加快编译速度
\documentclass[UTF-8]{ctexart}
\usepackage{graphicx}
\usepackage{subfigure}
\usepackage{xcolor}
\usepackage{amsmath}
\usepackage{amssymb}
\usepackage{tabularx}
\usepackage{amssymb}
\usepackage{amsthm}
%\usepackage[usenames,dvipsnames]{color}
\usepackage{hyperref}
\hypersetup{
	colorlinks=true,
	linkcolor=black,
	filecolor=black,      
	urlcolor=red,
	citecolor=black,
}
\usepackage{geometry}
\geometry{a4paper,centering,scale=0.80}
\usepackage[format=hang,font=small,textfont=it]{caption}
\usepackage[nottoc]{tocbibind}
\usepackage{algorithm}  
\usepackage{algorithmicx}  
\usepackage{algpseudocode}
\usepackage{prettyref}
\usepackage{framed}
\setlength{\parindent}{2em}
\usepackage{indentfirst}
\usepackage[framemethod=TikZ]{mdframed}
\newcounter{ques}[section]
\renewcommand{\theques}{\arabic{section}.\arabic{ques}}
\newcommand{\setParDis}{\setlength {\parskip} {0.3cm} }
\newcommand{\setParDef}{\setlength {\parskip} {0pt} }
\setParDis% 调整这一个subsection的段落间距
%\setParDef%恢复间距

\newenvironment{ques}[1][]{
	\refstepcounter{ques}
	\mdfsetup{
		frametitle={
			\tikz[baseline=(current bounding box.east), outer sep=0pt]
			\node[anchor=east,rectangle,fill=blue!20]
			{\strut Problem~\theques\ifstrempty{#1}{}{:~#1}};},
		innertopmargin=10pt,linecolor=blue!20,
		linewidth=2pt,topline=true,
		frametitleaboveskip=\dimexpr-\ht\strutbox\relax
	}
	\begin{mdframed}[]\relax
}{\end{mdframed}}

\newcounter{Thm}[section]
\renewcommand{\theThm}{\arabic{section}.\arabic{Thm}}
\newenvironment{Thm}[1][]{
	\refstepcounter{Thm}
	\mdfsetup{
		frametitle={
			\tikz[baseline=(current bounding box.east), outer sep=0pt]
			\node[anchor=east,rectangle,fill=blue!20]
			{\strut Theorem~\theThm\ifstrempty{#1}{}{:~#1}};},
		innertopmargin=10pt,linecolor=blue!20,
		linewidth=2pt,topline=true,
		frametitleaboveskip=\dimexpr-\ht\strutbox\relax
	}
	\begin{mdframed}[]\relax
}{\end{mdframed}}

\newcounter{Defi}[section]
\renewcommand{\theDefi}{\arabic{section}.\arabic{Defi}}
\newenvironment{Defi}[1][]{
	\refstepcounter{Defi}
	\mdfsetup{
		frametitle={
			\tikz[baseline=(current bounding box.east), outer sep=0pt]
			\node[anchor=east,rectangle,fill=blue!20]
			{\strut Definition~\theDefi\ifstrempty{#1}{}{:~#1}};},
		innertopmargin=10pt,linecolor=blue!20,
		linewidth=2pt,topline=true,
		frametitleaboveskip=\dimexpr-\ht\strutbox\relax
	}
	\begin{mdframed}[]\relax
}{\end{mdframed}}

\newrefformat{qlt}{\underline{性质 \ref{#1}}}
\newcommand{\tpf}[2]{\begin{ques}[#1]{\kaishu #2}\end{ques}}
\newcommand{\pf}[1]{\begin{ques}{\kaishu #1}\end{ques}}
\newcommand{\tthm}[2]{\begin{Thm}[#1]{\kaishu #2}\end{Thm}}
\newcommand{\thm}[1]{\begin{Thm}{\kaishu #1}\end{Thm}}
\newcommand{\tdefi}[2]{\begin{Defi}[#1]{\kaishu #2}\end{Defi}}
\newcommand{\defi}[1]{\begin{Defi}{\kaishu #1}\end{Defi}}
\newcommand{\opf}[1]{{\kaishu{#1}}}
\title{数学分析 I 作业(2024. Spring)}
\author{\texttt{As-The-Wind}}

\date{2024 年 2 月 19 日 $\rightarrow$ \today}

\date{}
\author{尹锦润}
\begin{document}
\maketitle
\fi

\section{2024.4.17 作业}

\begin{ques}
	求函数项级数的收敛区域:$\displaystyle \sum _{n=1}^{+\infty }\frac{n}{n+1}\left(\frac{x}{2x+1}\right)^{n}$。
\end{ques}



因为 $\displaystyle \frac{n}{n+1}$ 单调有界并且趋向于 $\displaystyle 1\left( n\rightarrow +\infty \right)$,由 Abel 判别法,该函数项级数收敛 $\displaystyle \Leftrightarrow $$\displaystyle \sum _{n=1}^{+\infty }\left(\frac{x}{2x+1}\right)^{n}$ 收敛。

因此 $\displaystyle 0\leqslant \left| \frac{x}{2x+1}\right| < 1$ 即所求区域也就是 $\displaystyle x\in ( -\infty ,-1) \cup \left( -\frac{1}{3} ,+\infty \right)$。

于是函数项级数收敛区域为 $\displaystyle ( -\infty ,-1) \cup \left( -\frac{1}{3} ,+\infty \right)$。



\begin{ques}
	求函数列 $\displaystyle \{f_{n}( x)\}$ 在给定区间上的极限函数,并讨论是否一致收敛:$\displaystyle f_{n}( x) =x( 1-x)^{n} ,x\in [ 0,1]$。
\end{ques}



$\displaystyle x=0$ 时,$\displaystyle f_{n}( x) \equiv 0,x\neq 0$ 时 $\displaystyle f_{n}( x)\rightarrow 0$,因此 $\displaystyle \lim _{n\rightarrow +\infty } f_{n}( x) =0$。

$\displaystyle \forall \varepsilon  >0$,对于 $\displaystyle x\in [ 0,\varepsilon )$,有 $\displaystyle f_{n}( x) < \varepsilon $,对于 $\displaystyle x\in [ \varepsilon ,1]$,有 $\displaystyle f_{n}( x) < ( 1-\varepsilon )^{n}$,$\displaystyle \exists N >0\ s.t.\ ( 1-\varepsilon )^{n} < \varepsilon ,\forall n >N$,则 $\displaystyle f_{n}( x) < \varepsilon ,\forall n >N$,进而 $\displaystyle f_{n}( x)$ 一致收敛。\qed 





\begin{ques}
	求函数列 $\displaystyle \{f_{n}( x)\}$ 在给定区间上的极限函数,并讨论是否一致收敛:$\displaystyle f_{n}( x) =nxe^{-nx^{2}} ,x\in [ -\infty ,+\infty ]$。
\end{ques}



当 $\displaystyle x=0$ 时,$\displaystyle f_{n}( x) =0$,当 $\displaystyle x\neq 0$ 时,当 $\displaystyle n\rightarrow +\infty $ 时,$\displaystyle ne^{-nx^{2}}\rightarrow 0$,进而 $\displaystyle f_{n}( x)\rightarrow 0$。

于是 $\displaystyle \lim _{n\rightarrow +\infty } f_{n}( x) =0$。

而对于一致收敛性,$\displaystyle \forall n >0,\exists x=\frac{1}{n} \ s.t.\ f_{n}( x) =e^{-\frac{1}{n}}$,当 $\displaystyle n\rightarrow +\infty $ 时 $\displaystyle e^{-\frac{1}{n}}\rightarrow 1 >0$,因此 $\displaystyle f_{n}( x)$ 不一致收敛。\qed 



\begin{ques}
	求函数列 $\displaystyle \{f_{n}( x)\}$ 在给定区间上的极限函数,并讨论是否一致收敛:$\displaystyle f_{n}( x) =n^{2} e^{-nx^{2}} ,x\in [ a,+\infty ) ,$ 这里 $\displaystyle a >0$。
\end{ques}



$\displaystyle x\neq 0$ ,$\displaystyle f_{n}( x)\rightarrow 0\left( n\rightarrow +\infty \right)$,因此 $\displaystyle \lim _{n\rightarrow +\infty } f_{n}( x) =0$。

而 $\displaystyle \forall n >0,f_{n}( x) \leqslant n^{2} e^{-na^{2}} ,\forall x\in [ a,+\infty )$,

对于 $\displaystyle n^{2} e^{-na^{2}}$,有 $\displaystyle \lim _{n\rightarrow \infty } n^{2} e^{-na^{2}} =0$,因此 $\displaystyle \forall \varepsilon  >0,\exists N >0\ s.t.\ n^{2} e^{-na^{2}} < \varepsilon ,\forall n >N$,

进而 $\displaystyle \forall \varepsilon  >0,\exists N >0\ s.t.\ f_{n}( x) \leqslant n^{2} e^{-na^{2}} < \varepsilon ,\forall n >N$,于是 $\displaystyle f_{n}( x)$ 一致收敛。\qed 



\begin{ques}
	设 $\displaystyle u_{n}( x) \geqslant 0,\ \forall x\in [ a,b] ;u_{n}( x) \in C[ a,b] ,\forall n\in \mathbb{N} ,S( x) =\sum _{n=1}^{+\infty } u_{n}( x) ,x\in [ a,b]$。证明:$\displaystyle S( x)$ 在 $\displaystyle [ a,b]$ 上取到最小值。
\end{ques}


考虑 $\displaystyle S( x)$ 在 $\displaystyle [ a,b]$ 上的下确界是 $\displaystyle c$,则存在一系列 $\displaystyle x_{m} \in [ a,b] \ s.t.\ c\leqslant S( x_{m}) < c+\frac{1}{m} ,\forall m\in N^{*}$。

因为 $\displaystyle x_{m} \in [ a,b]$,因此一定存在一个聚点 $\displaystyle x_{0}$。

同时有 $\displaystyle S_{n}( x_{m}) =\sum _{t=1}^{n} u_{t}( x_{m}) < S( x_{m}) \ \forall n,m$,且 $\displaystyle \forall m,\ S_{n}( x_{m})\rightarrow S( x_{m})\left( n\rightarrow +\infty \right)$,则 $\displaystyle S_{n}( x_{m}) < c+\frac{1}{m} ,\forall n$。

于是 $\displaystyle \forall n$,当 $\displaystyle m\rightarrow +\infty $ 时,$\displaystyle S_{n}( x_{0}) \leqslant c$,进而 $\displaystyle n\rightarrow +\infty $ 时,有 $\displaystyle S( x_{0}) \leqslant c$,进而 $\displaystyle S( x)$ 在 $\displaystyle x_{0}$ 处取到最小值。


\ifx\allfiles\undefined
\end{document}
\fi